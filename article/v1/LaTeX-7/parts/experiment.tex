\section{Experiment}
\subsection{Data}
To experiment our algoritm we use two corpus $C_1$ and $C_2$. $C_1$ contains a
set of documents related to Finance and $C_2$ contains a set of documents
related to Sport. With these corpus we generate $KW_1$ the set of Key Words
for $C_1$ and $KW_2$ the set of Key Words. Then we denote $KW$ the set of Key
Words for clustering such that $KW = KW_1 \cup KW_2$.\\
To generate $KW_1$ (resp $KW_2$) we extract Key Words from $C_1$ (resp $C_2$)
with RAKE algorithm \cite{rake}.\\
For must-link we extract a set of pair (X, X') from  each corpus such that
$(X, X') \in C_1 \vee (X, X') \in C_2$. And for cannot-link constraints we
extract a set of pair (X, X') from  each corpus such that
$(X \in C_1 \wedge  X' \in C_2)\vee (X \in C_2 \wedge  X' \in C_1)$. 
\subsubsection{Reference Algorithm}
We use an algorithm that doesn't use pairwise constraints to perform K-Means
with lexical constraints. For this algorithm : 
\\We compare our algorithm's results with reference algorithm's result.    
