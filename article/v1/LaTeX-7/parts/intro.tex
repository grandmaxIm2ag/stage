
\section{Introduction}\label{sec:intro}

Clustering is one of the most fundamental tasks in data mining and machine
learning. $K$-Means algorithm is a clustering method using centroid models,
it represents each cluster by a single mean vector. $K$-Means clustering sorts
n objects into k clusters in which each observation belongs to
the cluster with the nearest centroid. This problem is computationally
difficult (NP-hard).
\\In real application domains, users may want to introduce constaints to find
usefull properties for clustering data. The difficulty with integration of
constraints to $K$-Means algorithm  is to find a good representation for data
taking account constraint. The Deep Learning and Auto-Encoder can be used to
learn this representation. With Auto-Encoder we have to perform the $K$-Means in
the latent space learned.
\\The major contribution to this work, is to propose a constrained Deep $K$-Means
taking account into ML and CL constraints and lexical constraints.
\\In next section, we provide some background on the $K$-Means alorithm and deep
learning. In section 3, we proposed a method to introduce constraints to the
$K$-Means algorithm. And we experiment our method in section 4.  
