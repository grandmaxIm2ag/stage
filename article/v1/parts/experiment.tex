\section{Experiment}

\subsection{Data}
To experiment our algoritm we use the dataset 20NewsGroup \cite{Newsgroups20}.
The 20 Newsgroups data set is a collection of approximately 20,000 newsgroup 
documents, partitioned evenly across 20 different newsgroups.\\
Each document are represented by a vector using frequency-inverse document 
frequency (TFIDF) representation.
The term frequency-inverse document frequency is a method of weghting depicting 
the significiance of each word in a document rather a corpus.
\\To compute the Term 
Frequency (TF) of a term t in document X we use the double normalisation K, 
with k=0.5 :
\begin{equation}
TF(t, X) = 0.5 + 0.5\frac{f_{t, X}}{max_{t' \in C}f_{t', X}} 
\end{equation}
\begin{equation}
IDF(t, C) = log(\frac{N}{|X \in C : t \in X|})
\end{equation}
\begin{equation}
TFIDF(t,c,C) = TF(t, X) . IDF(t, C)   
\end{equation}
\subsection{Generate Constraint}
\subsubsection{Lexical Constraints}
To generate the set of keywords $KW$ we rank each word of each 
document of each class using TFIDF~\ref{algo:gen_kw}
\begin{algorithm}
  \SetKwInOut{Input}{input}
  \SetKwInOut{Output}{output}
  \Input{Corpus C, The number of keywords per classes $P$}
  \Output{KW}
  $KW \gets \{\}$\\
  \ForEach{Class $c_i \in C$}{
    $rank \gets [0 ... 0]$\\
    \ForEach{Document $X \in c_i$}{
      \ForEach{Word $w \in X$}{
        \If{$TFIDF(w) \ge rank_w$}{
          $rank_w \gets w$\\
        }
      }
    }
    $KW \gets KW \cup \{\{w_1, w_2 ... w_P\} : \not\exists (v_1, v_2) | v_1 \not\in 
    \{w_1, w_2 ... w_P\}, v_2 \in \{w_1, w_2 ... w_P\}, rank_{v_1} \ge rank_{v_2}\}$\\
  }
  \Return{KW}
  \caption{\label{algo:gen_kw}Extract Keywords}
\end{algorithm}
\subsubsection{Background Knowledge}
We generate pairwise randomly~\ref{algo:gen_pair}
\begin{algorithm}[!h]
  \SetKwInOut{Input}{input}
  \SetKwInOut{Output}{output}
  \Input{Corpus C, The set of labels L, The number of pair $N_p$}
  \Output{Must-Link Pair ML, Cannot-Link Pair CL}
  \For{i = 1 : $N_p$}{
    Coose randomly ($X_i, X_j$)\\
    \If{$L_i == L_j$}{
      Insert ($X_i, X_j$) in ML 
    }
    \Else{
      Insert ($X_i, X_j$) in CL
    }
  }
  \Return{ML, CL}
  \caption{\label{algo:gen_pair}Extract Pair}
\end{algorithm}
\subsection{Evaluation}
\subsubsection{Reference Algorithm}
We evaluate our algorithm with four algorithms :
\begin{itemize}
\item \textbf{COPKmeans} proposed by Wagstaff \cite{Wagstaff:2001:CKC:645530.655669}.
\item \textbf{COPKmeans with lexical constraints}, we learn a $k$-means
  friendly space taking into account lexical constraints. The loss
  function for this model is :
  $$
  Min~L(KW, C, K; \theta) = \lambda(L_{rec}(C, \theta) + \omega_{KW} )+
  (1-\lambda)L_{clust}(C,K)
  $$
\item \textbf{Thematic kmeans}, the loss function for this algorithm is
  \begin{equation*}
    argmin \sum_{i=1}^{N}L(x_i) + \beta\sum_{k=1}^{K}||r_k - \sum_{\omega \in KW}
    \alpha_{k,\omega}h(\omega, \theta)||_2^2 + \delta\sum_{k=1}^{K}||\alpha_k|| 
  \end{equation*}
  with $L(x_i) = L_{rec}(x_i) + \lambda L_{clust}(x_i)$
\end{itemize}
\subsubsection{Metric}
To evaluate our algorithm and compare results with reference algorithms we can
use the NMI Metric\cite{measure}, Accuracy Metric \cite{measure}, and Adjusted
Rand index\cite{measure}. 
\begin{itemize}
\item The NMI Metric is defined by
$$
NMI(S,C) = \frac{I(S,C)}{[H(S)+H(C)]/2}
$$ 
with
$
I(S,C) =\sum_k \sum_f\frac{|s_k \cap c_f|}{N}log\frac{N|s_k \cap c_f|}{|s_k| |c_f|}
$ and $
H(S) = -\sum_k\frac{|s_k|}{N}log\frac{N|s_k|}{|s_k|}
$
\item The Accuracy metric is defined by :
$$
purity(S,C) = \frac{1}{N}\sum_k {max}_j|s_k \cap c_j|
$$
\item The Adjusted Rand index are defined by :
  $$ARI = \frac{a+b}{\binom{N}{2}}$$
  where a is the number of pairs of document in C
  that are in the same cluster in the predicted partition and in the
  same cluster in the real partition, and b he number of pairs of
  documents in C that are in different clusters in predicted partition
  and in different cluster in real partition.
\end{itemize}
\subsection{Experimental Setup}
\subsubsection{Autoencoder Architecture}
We use the same architecture used in~\cite{Deap-K-Means}. The encoder is a fully-connected 
multilayer perceptron formed by 3 hidden layers (with dimensions 500, 500, 2000) 
 and an embedding layer (with dimension K, the number of cluster). 
The decoder is a mirrored version of the encoder~\ref{fig:archi}.
The ReLu activation function is used on layers, except for the third
layer of encoder and decoder part.  
The auto-encoder weights are initialized following the Xavier scheme.
\begin{figure}[!h]
  \centering
  \tikzset{every picture/.style={scale=1.6}}
  % Graphic for TeX using PGF
% Title: /media/maxence/SD_MAXENCE/cours/m1Informatique/S2/stage/article/v0/ressources/neural_network_autoencoder.dia
% Creator: Dia v0.97.3
% CreationDate: Wed Mar 28 15:41:42 2018
% For: maxence
% \usepackage{tikz}
% The following commands are not supported in PSTricks at present
% We define them conditionally, so when they are implemented,
% this pgf file will use them.
\ifx\du\undefined
  \newlength{\du}
\fi
\setlength{\du}{15\unitlength}
\begin{tikzpicture}
\pgftransformxscale{1.000000}
\pgftransformyscale{-1.000000}
\definecolor{dialinecolor}{rgb}{0.000000, 0.000000, 0.000000}
\pgfsetstrokecolor{dialinecolor}
\definecolor{dialinecolor}{rgb}{1.000000, 1.000000, 1.000000}
\pgfsetfillcolor{dialinecolor}
\pgfsetlinewidth{0.050000\du}
\pgfsetdash{}{0pt}
\pgfsetdash{}{0pt}
\pgfsetbuttcap
\pgfsetmiterjoin
\pgfsetlinewidth{0.050000\du}
\pgfsetbuttcap
\pgfsetmiterjoin
\pgfsetdash{}{0pt}
\definecolor{dialinecolor}{rgb}{1.000000, 1.000000, 1.000000}
\pgfsetfillcolor{dialinecolor}
\pgfpathellipse{\pgfpoint{0.017617\du}{-190.709859\du}}{\pgfpoint{0.975015\du}{0\du}}{\pgfpoint{0\du}{0.975015\du}}
\pgfusepath{fill}
\definecolor{dialinecolor}{rgb}{0.000000, 0.000000, 0.000000}
\pgfsetstrokecolor{dialinecolor}
\pgfpathellipse{\pgfpoint{0.017617\du}{-190.709859\du}}{\pgfpoint{0.975015\du}{0\du}}{\pgfpoint{0\du}{0.975015\du}}
\pgfusepath{stroke}
\pgfsetbuttcap
\pgfsetmiterjoin
\pgfsetdash{}{0pt}
\definecolor{dialinecolor}{rgb}{0.000000, 0.000000, 0.000000}
\pgfsetstrokecolor{dialinecolor}
\pgfpathellipse{\pgfpoint{0.017617\du}{-190.709859\du}}{\pgfpoint{0.975015\du}{0\du}}{\pgfpoint{0\du}{0.975015\du}}
\pgfusepath{stroke}
\pgfsetlinewidth{0.050000\du}
\pgfsetdash{}{0pt}
\pgfsetdash{}{0pt}
\pgfsetbuttcap
\pgfsetmiterjoin
\pgfsetlinewidth{0.050000\du}
\pgfsetbuttcap
\pgfsetmiterjoin
\pgfsetdash{}{0pt}
\definecolor{dialinecolor}{rgb}{1.000000, 1.000000, 1.000000}
\pgfsetfillcolor{dialinecolor}
\pgfpathellipse{\pgfpoint{-0.005712\du}{-185.477681\du}}{\pgfpoint{0.975015\du}{0\du}}{\pgfpoint{0\du}{0.975015\du}}
\pgfusepath{fill}
\definecolor{dialinecolor}{rgb}{0.000000, 0.000000, 0.000000}
\pgfsetstrokecolor{dialinecolor}
\pgfpathellipse{\pgfpoint{-0.005712\du}{-185.477681\du}}{\pgfpoint{0.975015\du}{0\du}}{\pgfpoint{0\du}{0.975015\du}}
\pgfusepath{stroke}
\pgfsetbuttcap
\pgfsetmiterjoin
\pgfsetdash{}{0pt}
\definecolor{dialinecolor}{rgb}{0.000000, 0.000000, 0.000000}
\pgfsetstrokecolor{dialinecolor}
\pgfpathellipse{\pgfpoint{-0.005712\du}{-185.477681\du}}{\pgfpoint{0.975015\du}{0\du}}{\pgfpoint{0\du}{0.975015\du}}
\pgfusepath{stroke}
\pgfsetlinewidth{0.080000\du}
\pgfsetdash{}{0pt}
\pgfsetdash{}{0pt}
\pgfsetbuttcap
{
\definecolor{dialinecolor}{rgb}{0.000000, 0.000000, 0.000000}
\pgfsetfillcolor{dialinecolor}
% was here!!!
\pgfsetarrowsend{stealth}
\definecolor{dialinecolor}{rgb}{0.000000, 0.000000, 0.000000}
\pgfsetstrokecolor{dialinecolor}
\draw (-2.894676\du,-190.634796\du)--(-0.981519\du,-190.684106\du);
}
\pgfsetlinewidth{0.050000\du}
\pgfsetdash{}{0pt}
\pgfsetdash{}{0pt}
\pgfsetbuttcap
\pgfsetmiterjoin
\pgfsetlinewidth{0.050000\du}
\pgfsetbuttcap
\pgfsetmiterjoin
\pgfsetdash{}{0pt}
\definecolor{dialinecolor}{rgb}{1.000000, 1.000000, 1.000000}
\pgfsetfillcolor{dialinecolor}
\pgfpathellipse{\pgfpoint{0.021277\du}{-193.493278\du}}{\pgfpoint{0.975015\du}{0\du}}{\pgfpoint{0\du}{0.975015\du}}
\pgfusepath{fill}
\definecolor{dialinecolor}{rgb}{0.000000, 0.000000, 0.000000}
\pgfsetstrokecolor{dialinecolor}
\pgfpathellipse{\pgfpoint{0.021277\du}{-193.493278\du}}{\pgfpoint{0.975015\du}{0\du}}{\pgfpoint{0\du}{0.975015\du}}
\pgfusepath{stroke}
\pgfsetbuttcap
\pgfsetmiterjoin
\pgfsetdash{}{0pt}
\definecolor{dialinecolor}{rgb}{0.000000, 0.000000, 0.000000}
\pgfsetstrokecolor{dialinecolor}
\pgfpathellipse{\pgfpoint{0.021277\du}{-193.493278\du}}{\pgfpoint{0.975015\du}{0\du}}{\pgfpoint{0\du}{0.975015\du}}
\pgfusepath{stroke}
\pgfsetlinewidth{0.050000\du}
\pgfsetdash{}{0pt}
\pgfsetdash{}{0pt}
\pgfsetbuttcap
\pgfsetmiterjoin
\pgfsetlinewidth{0.050000\du}
\pgfsetbuttcap
\pgfsetmiterjoin
\pgfsetdash{}{0pt}
\definecolor{dialinecolor}{rgb}{1.000000, 1.000000, 1.000000}
\pgfsetfillcolor{dialinecolor}
\pgfpathellipse{\pgfpoint{8.189937\du}{-189.828452\du}}{\pgfpoint{0.975015\du}{0\du}}{\pgfpoint{0\du}{0.975015\du}}
\pgfusepath{fill}
\definecolor{dialinecolor}{rgb}{0.000000, 0.000000, 0.000000}
\pgfsetstrokecolor{dialinecolor}
\pgfpathellipse{\pgfpoint{8.189937\du}{-189.828452\du}}{\pgfpoint{0.975015\du}{0\du}}{\pgfpoint{0\du}{0.975015\du}}
\pgfusepath{stroke}
\pgfsetbuttcap
\pgfsetmiterjoin
\pgfsetdash{}{0pt}
\definecolor{dialinecolor}{rgb}{0.000000, 0.000000, 0.000000}
\pgfsetstrokecolor{dialinecolor}
\pgfpathellipse{\pgfpoint{8.189937\du}{-189.828452\du}}{\pgfpoint{0.975015\du}{0\du}}{\pgfpoint{0\du}{0.975015\du}}
\pgfusepath{stroke}
\pgfsetlinewidth{0.050000\du}
\pgfsetdash{}{0pt}
\pgfsetdash{}{0pt}
\pgfsetbuttcap
\pgfsetmiterjoin
\pgfsetlinewidth{0.050000\du}
\pgfsetbuttcap
\pgfsetmiterjoin
\pgfsetdash{}{0pt}
\definecolor{dialinecolor}{rgb}{1.000000, 1.000000, 1.000000}
\pgfsetfillcolor{dialinecolor}
\pgfpathellipse{\pgfpoint{8.134822\du}{-186.557197\du}}{\pgfpoint{0.975015\du}{0\du}}{\pgfpoint{0\du}{0.975015\du}}
\pgfusepath{fill}
\definecolor{dialinecolor}{rgb}{0.000000, 0.000000, 0.000000}
\pgfsetstrokecolor{dialinecolor}
\pgfpathellipse{\pgfpoint{8.134822\du}{-186.557197\du}}{\pgfpoint{0.975015\du}{0\du}}{\pgfpoint{0\du}{0.975015\du}}
\pgfusepath{stroke}
\pgfsetbuttcap
\pgfsetmiterjoin
\pgfsetdash{}{0pt}
\definecolor{dialinecolor}{rgb}{0.000000, 0.000000, 0.000000}
\pgfsetstrokecolor{dialinecolor}
\pgfpathellipse{\pgfpoint{8.134822\du}{-186.557197\du}}{\pgfpoint{0.975015\du}{0\du}}{\pgfpoint{0\du}{0.975015\du}}
\pgfusepath{stroke}
\pgfsetlinewidth{0.080000\du}
\pgfsetdash{}{0pt}
\pgfsetdash{}{0pt}
\pgfsetbuttcap
{
\definecolor{dialinecolor}{rgb}{0.000000, 0.000000, 0.000000}
\pgfsetfillcolor{dialinecolor}
% was here!!!
\pgfsetarrowsend{stealth}
\definecolor{dialinecolor}{rgb}{0.000000, 0.000000, 0.000000}
\pgfsetstrokecolor{dialinecolor}
\draw (-3.019651\du,-193.464706\du)--(-0.978057\du,-193.483889\du);
}
\pgfsetlinewidth{0.080000\du}
\pgfsetdash{}{0pt}
\pgfsetdash{}{0pt}
\pgfsetbuttcap
{
\definecolor{dialinecolor}{rgb}{0.000000, 0.000000, 0.000000}
\pgfsetfillcolor{dialinecolor}
% was here!!!
\pgfsetarrowsend{stealth}
\definecolor{dialinecolor}{rgb}{0.000000, 0.000000, 0.000000}
\pgfsetstrokecolor{dialinecolor}
\draw (-2.992058\du,-185.459325\du)--(-1.005266\du,-185.471538\du);
}
\pgfsetlinewidth{0.080000\du}
\pgfsetdash{}{0pt}
\pgfsetdash{}{0pt}
\pgfsetbuttcap
{
\definecolor{dialinecolor}{rgb}{0.000000, 0.000000, 0.000000}
\pgfsetfillcolor{dialinecolor}
% was here!!!
\pgfsetarrowsend{stealth}
\definecolor{dialinecolor}{rgb}{0.000000, 0.000000, 0.000000}
\pgfsetstrokecolor{dialinecolor}
\draw (1.010269\du,-193.351253\du)--(7.191219\du,-192.463633\du);
}
\pgfsetlinewidth{0.080000\du}
\pgfsetdash{}{0pt}
\pgfsetdash{}{0pt}
\pgfsetbuttcap
{
\definecolor{dialinecolor}{rgb}{0.000000, 0.000000, 0.000000}
\pgfsetfillcolor{dialinecolor}
% was here!!!
\pgfsetarrowsend{stealth}
\definecolor{dialinecolor}{rgb}{0.000000, 0.000000, 0.000000}
\pgfsetstrokecolor{dialinecolor}
\draw (0.933919\du,-193.083826\du)--(7.277295\du,-190.237904\du);
}
\pgfsetlinewidth{0.080000\du}
\pgfsetdash{}{0pt}
\pgfsetdash{}{0pt}
\pgfsetbuttcap
{
\definecolor{dialinecolor}{rgb}{0.000000, 0.000000, 0.000000}
\pgfsetfillcolor{dialinecolor}
% was here!!!
\pgfsetarrowsend{stealth}
\definecolor{dialinecolor}{rgb}{0.000000, 0.000000, 0.000000}
\pgfsetstrokecolor{dialinecolor}
\draw (0.933919\du,-193.083826\du)--(7.277295\du,-190.237904\du);
}
\pgfsetlinewidth{0.080000\du}
\pgfsetdash{}{0pt}
\pgfsetdash{}{0pt}
\pgfsetbuttcap
{
\definecolor{dialinecolor}{rgb}{0.000000, 0.000000, 0.000000}
\pgfsetfillcolor{dialinecolor}
% was here!!!
\pgfsetarrowsend{stealth}
\definecolor{dialinecolor}{rgb}{0.000000, 0.000000, 0.000000}
\pgfsetstrokecolor{dialinecolor}
\draw (0.998583\du,-190.903556\du)--(7.199245\du,-192.127911\du);
}
\pgfsetlinewidth{0.080000\du}
\pgfsetdash{}{0pt}
\pgfsetdash{}{0pt}
\pgfsetbuttcap
{
\definecolor{dialinecolor}{rgb}{0.000000, 0.000000, 0.000000}
\pgfsetfillcolor{dialinecolor}
% was here!!!
\pgfsetarrowsend{stealth}
\definecolor{dialinecolor}{rgb}{0.000000, 0.000000, 0.000000}
\pgfsetstrokecolor{dialinecolor}
\draw (1.011723\du,-190.602642\du)--(7.195831\du,-189.935669\du);
}
\pgfsetlinewidth{0.080000\du}
\pgfsetdash{}{0pt}
\pgfsetdash{}{0pt}
\pgfsetbuttcap
{
\definecolor{dialinecolor}{rgb}{0.000000, 0.000000, 0.000000}
\pgfsetfillcolor{dialinecolor}
% was here!!!
\pgfsetarrowsend{stealth}
\definecolor{dialinecolor}{rgb}{0.000000, 0.000000, 0.000000}
\pgfsetstrokecolor{dialinecolor}
\draw (0.907914\du,-190.254394\du)--(7.244525\du,-187.012661\du);
}
\pgfsetlinewidth{0.080000\du}
\pgfsetdash{}{0pt}
\pgfsetdash{}{0pt}
\pgfsetbuttcap
{
\definecolor{dialinecolor}{rgb}{0.000000, 0.000000, 0.000000}
\pgfsetfillcolor{dialinecolor}
% was here!!!
\pgfsetarrowsend{stealth}
\definecolor{dialinecolor}{rgb}{0.000000, 0.000000, 0.000000}
\pgfsetstrokecolor{dialinecolor}
\draw (0.985024\du,-185.609063\du)--(7.144086\du,-186.425815\du);
}
\pgfsetlinewidth{0.080000\du}
\pgfsetdash{}{0pt}
\pgfsetdash{}{0pt}
\pgfsetbuttcap
{
\definecolor{dialinecolor}{rgb}{0.000000, 0.000000, 0.000000}
\pgfsetfillcolor{dialinecolor}
% was here!!!
\pgfsetarrowsend{stealth}
\definecolor{dialinecolor}{rgb}{0.000000, 0.000000, 0.000000}
\pgfsetstrokecolor{dialinecolor}
\draw (0.877181\du,-185.946377\du)--(7.307044\du,-189.359756\du);
}
\pgfsetlinewidth{0.080000\du}
\pgfsetdash{}{0pt}
\pgfsetdash{}{0pt}
\pgfsetbuttcap
{
\definecolor{dialinecolor}{rgb}{0.000000, 0.000000, 0.000000}
\pgfsetfillcolor{dialinecolor}
% was here!!!
\pgfsetarrowsend{stealth}
\definecolor{dialinecolor}{rgb}{0.000000, 0.000000, 0.000000}
\pgfsetstrokecolor{dialinecolor}
\draw (0.633813\du,-186.012363\du)--(7.540686\du,-191.786926\du);
}
\pgfsetlinewidth{0.050000\du}
\pgfsetdash{}{0pt}
\pgfsetdash{}{0pt}
\pgfsetbuttcap
\pgfsetmiterjoin
\pgfsetlinewidth{0.050000\du}
\pgfsetbuttcap
\pgfsetmiterjoin
\pgfsetdash{}{0pt}
\definecolor{dialinecolor}{rgb}{1.000000, 1.000000, 1.000000}
\pgfsetfillcolor{dialinecolor}
\pgfpathellipse{\pgfpoint{8.180211\du}{-192.321608\du}}{\pgfpoint{0.975015\du}{0\du}}{\pgfpoint{0\du}{0.975015\du}}
\pgfusepath{fill}
\definecolor{dialinecolor}{rgb}{0.000000, 0.000000, 0.000000}
\pgfsetstrokecolor{dialinecolor}
\pgfpathellipse{\pgfpoint{8.180211\du}{-192.321608\du}}{\pgfpoint{0.975015\du}{0\du}}{\pgfpoint{0\du}{0.975015\du}}
\pgfusepath{stroke}
\pgfsetbuttcap
\pgfsetmiterjoin
\pgfsetdash{}{0pt}
\definecolor{dialinecolor}{rgb}{0.000000, 0.000000, 0.000000}
\pgfsetstrokecolor{dialinecolor}
\pgfpathellipse{\pgfpoint{8.180211\du}{-192.321608\du}}{\pgfpoint{0.975015\du}{0\du}}{\pgfpoint{0\du}{0.975015\du}}
\pgfusepath{stroke}
\pgfsetlinewidth{0.050000\du}
\pgfsetdash{}{0pt}
\pgfsetdash{}{0pt}
\pgfsetbuttcap
\pgfsetmiterjoin
\pgfsetlinewidth{0.050000\du}
\pgfsetbuttcap
\pgfsetmiterjoin
\pgfsetdash{}{0pt}
\definecolor{dialinecolor}{rgb}{1.000000, 1.000000, 1.000000}
\pgfsetfillcolor{dialinecolor}
\pgfpathellipse{\pgfpoint{15.361418\du}{-193.498046\du}}{\pgfpoint{0.975015\du}{0\du}}{\pgfpoint{0\du}{0.975015\du}}
\pgfusepath{fill}
\definecolor{dialinecolor}{rgb}{0.000000, 0.000000, 0.000000}
\pgfsetstrokecolor{dialinecolor}
\pgfpathellipse{\pgfpoint{15.361418\du}{-193.498046\du}}{\pgfpoint{0.975015\du}{0\du}}{\pgfpoint{0\du}{0.975015\du}}
\pgfusepath{stroke}
\pgfsetbuttcap
\pgfsetmiterjoin
\pgfsetdash{}{0pt}
\definecolor{dialinecolor}{rgb}{0.000000, 0.000000, 0.000000}
\pgfsetstrokecolor{dialinecolor}
\pgfpathellipse{\pgfpoint{15.361418\du}{-193.498046\du}}{\pgfpoint{0.975015\du}{0\du}}{\pgfpoint{0\du}{0.975015\du}}
\pgfusepath{stroke}
\pgfsetlinewidth{0.050000\du}
\pgfsetdash{}{0pt}
\pgfsetdash{}{0pt}
\pgfsetbuttcap
\pgfsetmiterjoin
\pgfsetlinewidth{0.050000\du}
\pgfsetbuttcap
\pgfsetmiterjoin
\pgfsetdash{}{0pt}
\definecolor{dialinecolor}{rgb}{1.000000, 1.000000, 1.000000}
\pgfsetfillcolor{dialinecolor}
\pgfpathellipse{\pgfpoint{15.403565\du}{-190.680688\du}}{\pgfpoint{0.975015\du}{0\du}}{\pgfpoint{0\du}{0.975015\du}}
\pgfusepath{fill}
\definecolor{dialinecolor}{rgb}{0.000000, 0.000000, 0.000000}
\pgfsetstrokecolor{dialinecolor}
\pgfpathellipse{\pgfpoint{15.403565\du}{-190.680688\du}}{\pgfpoint{0.975015\du}{0\du}}{\pgfpoint{0\du}{0.975015\du}}
\pgfusepath{stroke}
\pgfsetbuttcap
\pgfsetmiterjoin
\pgfsetdash{}{0pt}
\definecolor{dialinecolor}{rgb}{0.000000, 0.000000, 0.000000}
\pgfsetstrokecolor{dialinecolor}
\pgfpathellipse{\pgfpoint{15.403565\du}{-190.680688\du}}{\pgfpoint{0.975015\du}{0\du}}{\pgfpoint{0\du}{0.975015\du}}
\pgfusepath{stroke}
\pgfsetlinewidth{0.050000\du}
\pgfsetdash{}{0pt}
\pgfsetdash{}{0pt}
\pgfsetbuttcap
\pgfsetmiterjoin
\pgfsetlinewidth{0.050000\du}
\pgfsetbuttcap
\pgfsetmiterjoin
\pgfsetdash{}{0pt}
\definecolor{dialinecolor}{rgb}{1.000000, 1.000000, 1.000000}
\pgfsetfillcolor{dialinecolor}
\pgfpathellipse{\pgfpoint{15.348450\du}{-185.366936\du}}{\pgfpoint{0.975015\du}{0\du}}{\pgfpoint{0\du}{0.975015\du}}
\pgfusepath{fill}
\definecolor{dialinecolor}{rgb}{0.000000, 0.000000, 0.000000}
\pgfsetstrokecolor{dialinecolor}
\pgfpathellipse{\pgfpoint{15.348450\du}{-185.366936\du}}{\pgfpoint{0.975015\du}{0\du}}{\pgfpoint{0\du}{0.975015\du}}
\pgfusepath{stroke}
\pgfsetbuttcap
\pgfsetmiterjoin
\pgfsetdash{}{0pt}
\definecolor{dialinecolor}{rgb}{0.000000, 0.000000, 0.000000}
\pgfsetstrokecolor{dialinecolor}
\pgfpathellipse{\pgfpoint{15.348450\du}{-185.366936\du}}{\pgfpoint{0.975015\du}{0\du}}{\pgfpoint{0\du}{0.975015\du}}
\pgfusepath{stroke}
\pgfsetlinewidth{0.080000\du}
\pgfsetdash{}{0pt}
\pgfsetdash{}{0pt}
\pgfsetbuttcap
{
\definecolor{dialinecolor}{rgb}{0.000000, 0.000000, 0.000000}
\pgfsetfillcolor{dialinecolor}
% was here!!!
\pgfsetarrowsend{stealth}
\definecolor{dialinecolor}{rgb}{0.000000, 0.000000, 0.000000}
\pgfsetstrokecolor{dialinecolor}
\draw (9.166400\du,-192.483167\du)--(14.375230\du,-193.336487\du);
}
\pgfsetlinewidth{0.080000\du}
\pgfsetdash{}{0pt}
\pgfsetdash{}{0pt}
\pgfsetbuttcap
{
\definecolor{dialinecolor}{rgb}{0.000000, 0.000000, 0.000000}
\pgfsetfillcolor{dialinecolor}
% was here!!!
\pgfsetarrowsend{stealth}
\definecolor{dialinecolor}{rgb}{0.000000, 0.000000, 0.000000}
\pgfsetstrokecolor{dialinecolor}
\draw (9.079807\du,-190.283791\du)--(14.471549\du,-193.042706\du);
}
\pgfsetlinewidth{0.080000\du}
\pgfsetdash{}{0pt}
\pgfsetdash{}{0pt}
\pgfsetbuttcap
{
\definecolor{dialinecolor}{rgb}{0.000000, 0.000000, 0.000000}
\pgfsetfillcolor{dialinecolor}
% was here!!!
\pgfsetarrowsend{stealth}
\definecolor{dialinecolor}{rgb}{0.000000, 0.000000, 0.000000}
\pgfsetstrokecolor{dialinecolor}
\draw (8.855541\du,-187.249418\du)--(14.640699\du,-192.805825\du);
}
\pgfsetlinewidth{0.080000\du}
\pgfsetdash{}{0pt}
\pgfsetdash{}{0pt}
\pgfsetbuttcap
{
\definecolor{dialinecolor}{rgb}{0.000000, 0.000000, 0.000000}
\pgfsetfillcolor{dialinecolor}
% was here!!!
\pgfsetarrowsend{stealth}
\definecolor{dialinecolor}{rgb}{0.000000, 0.000000, 0.000000}
\pgfsetstrokecolor{dialinecolor}
\draw (9.154553\du,-192.100268\du)--(14.429224\du,-190.902028\du);
}
\pgfsetlinewidth{0.080000\du}
\pgfsetdash{}{0pt}
\pgfsetdash{}{0pt}
\pgfsetbuttcap
{
\definecolor{dialinecolor}{rgb}{0.000000, 0.000000, 0.000000}
\pgfsetfillcolor{dialinecolor}
% was here!!!
\pgfsetarrowsend{stealth}
\definecolor{dialinecolor}{rgb}{0.000000, 0.000000, 0.000000}
\pgfsetstrokecolor{dialinecolor}
\draw (9.179698\du,-189.945384\du)--(14.413805\du,-190.563755\du);
}
\pgfsetlinewidth{0.080000\du}
\pgfsetdash{}{0pt}
\pgfsetdash{}{0pt}
\pgfsetbuttcap
{
\definecolor{dialinecolor}{rgb}{0.000000, 0.000000, 0.000000}
\pgfsetfillcolor{dialinecolor}
% was here!!!
\pgfsetarrowsend{stealth}
\definecolor{dialinecolor}{rgb}{0.000000, 0.000000, 0.000000}
\pgfsetstrokecolor{dialinecolor}
\draw (9.004817\du,-187.050737\du)--(14.533570\du,-190.187147\du);
}
\pgfsetlinewidth{0.080000\du}
\pgfsetdash{}{0pt}
\pgfsetdash{}{0pt}
\pgfsetbuttcap
{
\definecolor{dialinecolor}{rgb}{0.000000, 0.000000, 0.000000}
\pgfsetfillcolor{dialinecolor}
% was here!!!
\pgfsetarrowsend{stealth}
\definecolor{dialinecolor}{rgb}{0.000000, 0.000000, 0.000000}
\pgfsetstrokecolor{dialinecolor}
\draw (8.898172\du,-191.625037\du)--(14.630489\du,-186.063507\du);
}
\pgfsetlinewidth{0.080000\du}
\pgfsetdash{}{0pt}
\pgfsetdash{}{0pt}
\pgfsetbuttcap
{
\definecolor{dialinecolor}{rgb}{0.000000, 0.000000, 0.000000}
\pgfsetfillcolor{dialinecolor}
% was here!!!
\pgfsetarrowsend{stealth}
\definecolor{dialinecolor}{rgb}{0.000000, 0.000000, 0.000000}
\pgfsetstrokecolor{dialinecolor}
\draw (9.037564\du,-189.300172\du)--(14.500823\du,-185.895216\du);
}
\pgfsetlinewidth{0.080000\du}
\pgfsetdash{}{0pt}
\pgfsetdash{}{0pt}
\pgfsetbuttcap
{
\definecolor{dialinecolor}{rgb}{0.000000, 0.000000, 0.000000}
\pgfsetfillcolor{dialinecolor}
% was here!!!
\pgfsetarrowsend{stealth}
\definecolor{dialinecolor}{rgb}{0.000000, 0.000000, 0.000000}
\pgfsetstrokecolor{dialinecolor}
\draw (9.121500\du,-186.394393\du)--(14.361771\du,-185.529740\du);
}
\definecolor{dialinecolor}{rgb}{1.000000, 1.000000, 1.000000}
\pgfsetfillcolor{dialinecolor}
\fill (-1.394217\du,-197.574197\du)--(-1.394217\du,-194.784908\du)--(1.393283\du,-194.784908\du)--(1.393283\du,-197.574197\du)--cycle;
\pgfsetlinewidth{0.000000\du}
\pgfsetdash{}{0pt}
\pgfsetdash{}{0pt}
\pgfsetmiterjoin
\definecolor{dialinecolor}{rgb}{1.000000, 1.000000, 1.000000}
\pgfsetstrokecolor{dialinecolor}
\draw (-1.394217\du,-197.574197\du)--(-1.394217\du,-194.784908\du)--(1.393283\du,-194.784908\du)--(1.393283\du,-197.574197\du)--cycle;
% setfont left to latex
\definecolor{dialinecolor}{rgb}{0.000000, 0.000000, 0.000000}
\pgfsetstrokecolor{dialinecolor}
\node at (-0.000467\du,-196.741697\du){Input};
% setfont left to latex
\definecolor{dialinecolor}{rgb}{0.000000, 0.000000, 0.000000}
\pgfsetstrokecolor{dialinecolor}
\node at (-0.000467\du,-196.294374\du){};
% setfont left to latex
\definecolor{dialinecolor}{rgb}{0.000000, 0.000000, 0.000000}
\pgfsetstrokecolor{dialinecolor}
\node at (-0.000467\du,-195.847052\du){};
% setfont left to latex
\definecolor{dialinecolor}{rgb}{0.000000, 0.000000, 0.000000}
\pgfsetstrokecolor{dialinecolor}
\node at (-0.000467\du,-195.399730\du){Layer};
\definecolor{dialinecolor}{rgb}{1.000000, 1.000000, 1.000000}
\pgfsetfillcolor{dialinecolor}
\fill (6.658195\du,-197.587703\du)--(6.658195\du,-194.798414\du)--(9.408882\du,-194.798414\du)--(9.408882\du,-197.587703\du)--cycle;
\pgfsetlinewidth{0.000000\du}
\pgfsetdash{}{0pt}
\pgfsetdash{}{0pt}
\pgfsetmiterjoin
\definecolor{dialinecolor}{rgb}{1.000000, 1.000000, 1.000000}
\pgfsetstrokecolor{dialinecolor}
\draw (6.658195\du,-197.587703\du)--(6.658195\du,-194.798414\du)--(9.408882\du,-194.798414\du)--(9.408882\du,-197.587703\du)--cycle;
% setfont left to latex
\definecolor{dialinecolor}{rgb}{0.000000, 0.000000, 0.000000}
\pgfsetstrokecolor{dialinecolor}
\node at (8.033539\du,-196.755203\du){Latent};
% setfont left to latex
\definecolor{dialinecolor}{rgb}{0.000000, 0.000000, 0.000000}
\pgfsetstrokecolor{dialinecolor}
\node at (8.033539\du,-196.307881\du){};
% setfont left to latex
\definecolor{dialinecolor}{rgb}{0.000000, 0.000000, 0.000000}
\pgfsetstrokecolor{dialinecolor}
\node at (8.033539\du,-195.860559\du){};
% setfont left to latex
\definecolor{dialinecolor}{rgb}{0.000000, 0.000000, 0.000000}
\pgfsetstrokecolor{dialinecolor}
\node at (8.033539\du,-195.413237\du){Space};
\definecolor{dialinecolor}{rgb}{1.000000, 1.000000, 1.000000}
\pgfsetfillcolor{dialinecolor}
\fill (13.668071\du,-197.541559\du)--(13.668071\du,-194.752271\du)--(16.455571\du,-194.752271\du)--(16.455571\du,-197.541559\du)--cycle;
\pgfsetlinewidth{0.000000\du}
\pgfsetdash{}{0pt}
\pgfsetdash{}{0pt}
\pgfsetmiterjoin
\definecolor{dialinecolor}{rgb}{1.000000, 1.000000, 1.000000}
\pgfsetstrokecolor{dialinecolor}
\draw (13.668071\du,-197.541559\du)--(13.668071\du,-194.752271\du)--(16.455571\du,-194.752271\du)--(16.455571\du,-197.541559\du)--cycle;
% setfont left to latex
\definecolor{dialinecolor}{rgb}{0.000000, 0.000000, 0.000000}
\pgfsetstrokecolor{dialinecolor}
\node at (15.061821\du,-196.709059\du){Output};
% setfont left to latex
\definecolor{dialinecolor}{rgb}{0.000000, 0.000000, 0.000000}
\pgfsetstrokecolor{dialinecolor}
\node at (15.061821\du,-196.261737\du){ };
% setfont left to latex
\definecolor{dialinecolor}{rgb}{0.000000, 0.000000, 0.000000}
\pgfsetstrokecolor{dialinecolor}
\node at (15.061821\du,-195.814415\du){};
% setfont left to latex
\definecolor{dialinecolor}{rgb}{0.000000, 0.000000, 0.000000}
\pgfsetstrokecolor{dialinecolor}
\node at (15.061821\du,-195.367093\du){Layer};
\pgfsetlinewidth{0.080000\du}
\pgfsetdash{}{0pt}
\pgfsetdash{}{0pt}
\pgfsetbuttcap
{
\definecolor{dialinecolor}{rgb}{0.000000, 0.000000, 0.000000}
\pgfsetfillcolor{dialinecolor}
% was here!!!
\pgfsetarrowsend{stealth}
\definecolor{dialinecolor}{rgb}{0.000000, 0.000000, 0.000000}
\pgfsetstrokecolor{dialinecolor}
\draw (16.336433\du,-193.498046\du)--(18.509790\du,-193.444114\du);
}
\pgfsetlinewidth{0.080000\du}
\pgfsetdash{}{0pt}
\pgfsetdash{}{0pt}
\pgfsetbuttcap
{
\definecolor{dialinecolor}{rgb}{0.000000, 0.000000, 0.000000}
\pgfsetfillcolor{dialinecolor}
% was here!!!
\pgfsetarrowsend{stealth}
\definecolor{dialinecolor}{rgb}{0.000000, 0.000000, 0.000000}
\pgfsetstrokecolor{dialinecolor}
\draw (16.401506\du,-190.691097\du)--(18.532845\du,-190.713328\du);
}
\pgfsetlinewidth{0.080000\du}
\pgfsetdash{}{0pt}
\pgfsetdash{}{0pt}
\pgfsetbuttcap
{
\definecolor{dialinecolor}{rgb}{0.000000, 0.000000, 0.000000}
\pgfsetfillcolor{dialinecolor}
% was here!!!
\pgfsetarrowsend{stealth}
\definecolor{dialinecolor}{rgb}{0.000000, 0.000000, 0.000000}
\pgfsetstrokecolor{dialinecolor}
\draw (16.346412\du,-185.367421\du)--(18.585028\du,-185.368508\du);
}
\pgfsetlinewidth{0.100000\du}
\pgfsetdash{}{0pt}
\pgfsetdash{}{0pt}
\pgfsetbuttcap
\pgfsetmiterjoin
\pgfsetlinewidth{0.100000\du}
\pgfsetbuttcap
\pgfsetmiterjoin
\pgfsetdash{}{0pt}
\definecolor{dialinecolor}{rgb}{0.000000, 0.000000, 0.000000}
\pgfsetfillcolor{dialinecolor}
\pgfpathellipse{\pgfpoint{8.059748\du}{-188.488165\du}}{\pgfpoint{0.054256\du}{0\du}}{\pgfpoint{0\du}{0.054256\du}}
\pgfusepath{fill}
\definecolor{dialinecolor}{rgb}{0.000000, 0.000000, 0.000000}
\pgfsetstrokecolor{dialinecolor}
\pgfpathellipse{\pgfpoint{8.059748\du}{-188.488165\du}}{\pgfpoint{0.054256\du}{0\du}}{\pgfpoint{0\du}{0.054256\du}}
\pgfusepath{stroke}
\pgfsetbuttcap
\pgfsetmiterjoin
\pgfsetdash{}{0pt}
\definecolor{dialinecolor}{rgb}{0.000000, 0.000000, 0.000000}
\pgfsetstrokecolor{dialinecolor}
\pgfpathellipse{\pgfpoint{8.059748\du}{-188.488165\du}}{\pgfpoint{0.054256\du}{0\du}}{\pgfpoint{0\du}{0.054256\du}}
\pgfusepath{stroke}
\pgfsetlinewidth{0.100000\du}
\pgfsetdash{}{0pt}
\pgfsetdash{}{0pt}
\pgfsetbuttcap
\pgfsetmiterjoin
\pgfsetlinewidth{0.100000\du}
\pgfsetbuttcap
\pgfsetmiterjoin
\pgfsetdash{}{0pt}
\definecolor{dialinecolor}{rgb}{0.000000, 0.000000, 0.000000}
\pgfsetfillcolor{dialinecolor}
\pgfpathellipse{\pgfpoint{8.067707\du}{-187.809990\du}}{\pgfpoint{0.054256\du}{0\du}}{\pgfpoint{0\du}{0.054256\du}}
\pgfusepath{fill}
\definecolor{dialinecolor}{rgb}{0.000000, 0.000000, 0.000000}
\pgfsetstrokecolor{dialinecolor}
\pgfpathellipse{\pgfpoint{8.067707\du}{-187.809990\du}}{\pgfpoint{0.054256\du}{0\du}}{\pgfpoint{0\du}{0.054256\du}}
\pgfusepath{stroke}
\pgfsetbuttcap
\pgfsetmiterjoin
\pgfsetdash{}{0pt}
\definecolor{dialinecolor}{rgb}{0.000000, 0.000000, 0.000000}
\pgfsetstrokecolor{dialinecolor}
\pgfpathellipse{\pgfpoint{8.067707\du}{-187.809990\du}}{\pgfpoint{0.054256\du}{0\du}}{\pgfpoint{0\du}{0.054256\du}}
\pgfusepath{stroke}
\pgfsetlinewidth{0.100000\du}
\pgfsetdash{}{0pt}
\pgfsetdash{}{0pt}
\pgfsetbuttcap
\pgfsetmiterjoin
\pgfsetlinewidth{0.100000\du}
\pgfsetbuttcap
\pgfsetmiterjoin
\pgfsetdash{}{0pt}
\definecolor{dialinecolor}{rgb}{0.000000, 0.000000, 0.000000}
\pgfsetfillcolor{dialinecolor}
\pgfpathellipse{\pgfpoint{15.354433\du}{-189.239925\du}}{\pgfpoint{0.054256\du}{0\du}}{\pgfpoint{0\du}{0.054256\du}}
\pgfusepath{fill}
\definecolor{dialinecolor}{rgb}{0.000000, 0.000000, 0.000000}
\pgfsetstrokecolor{dialinecolor}
\pgfpathellipse{\pgfpoint{15.354433\du}{-189.239925\du}}{\pgfpoint{0.054256\du}{0\du}}{\pgfpoint{0\du}{0.054256\du}}
\pgfusepath{stroke}
\pgfsetbuttcap
\pgfsetmiterjoin
\pgfsetdash{}{0pt}
\definecolor{dialinecolor}{rgb}{0.000000, 0.000000, 0.000000}
\pgfsetstrokecolor{dialinecolor}
\pgfpathellipse{\pgfpoint{15.354433\du}{-189.239925\du}}{\pgfpoint{0.054256\du}{0\du}}{\pgfpoint{0\du}{0.054256\du}}
\pgfusepath{stroke}
\pgfsetlinewidth{0.100000\du}
\pgfsetdash{}{0pt}
\pgfsetdash{}{0pt}
\pgfsetbuttcap
\pgfsetmiterjoin
\pgfsetlinewidth{0.100000\du}
\pgfsetbuttcap
\pgfsetmiterjoin
\pgfsetdash{}{0pt}
\definecolor{dialinecolor}{rgb}{0.000000, 0.000000, 0.000000}
\pgfsetfillcolor{dialinecolor}
\pgfpathellipse{\pgfpoint{15.356153\du}{-188.529835\du}}{\pgfpoint{0.054256\du}{0\du}}{\pgfpoint{0\du}{0.054256\du}}
\pgfusepath{fill}
\definecolor{dialinecolor}{rgb}{0.000000, 0.000000, 0.000000}
\pgfsetstrokecolor{dialinecolor}
\pgfpathellipse{\pgfpoint{15.356153\du}{-188.529835\du}}{\pgfpoint{0.054256\du}{0\du}}{\pgfpoint{0\du}{0.054256\du}}
\pgfusepath{stroke}
\pgfsetbuttcap
\pgfsetmiterjoin
\pgfsetdash{}{0pt}
\definecolor{dialinecolor}{rgb}{0.000000, 0.000000, 0.000000}
\pgfsetstrokecolor{dialinecolor}
\pgfpathellipse{\pgfpoint{15.356153\du}{-188.529835\du}}{\pgfpoint{0.054256\du}{0\du}}{\pgfpoint{0\du}{0.054256\du}}
\pgfusepath{stroke}
\pgfsetlinewidth{0.100000\du}
\pgfsetdash{}{0pt}
\pgfsetdash{}{0pt}
\pgfsetbuttcap
\pgfsetmiterjoin
\pgfsetlinewidth{0.100000\du}
\pgfsetbuttcap
\pgfsetmiterjoin
\pgfsetdash{}{0pt}
\definecolor{dialinecolor}{rgb}{0.000000, 0.000000, 0.000000}
\pgfsetfillcolor{dialinecolor}
\pgfpathellipse{\pgfpoint{15.357872\du}{-187.848401\du}}{\pgfpoint{0.054256\du}{0\du}}{\pgfpoint{0\du}{0.054256\du}}
\pgfusepath{fill}
\definecolor{dialinecolor}{rgb}{0.000000, 0.000000, 0.000000}
\pgfsetstrokecolor{dialinecolor}
\pgfpathellipse{\pgfpoint{15.357872\du}{-187.848401\du}}{\pgfpoint{0.054256\du}{0\du}}{\pgfpoint{0\du}{0.054256\du}}
\pgfusepath{stroke}
\pgfsetbuttcap
\pgfsetmiterjoin
\pgfsetdash{}{0pt}
\definecolor{dialinecolor}{rgb}{0.000000, 0.000000, 0.000000}
\pgfsetstrokecolor{dialinecolor}
\pgfpathellipse{\pgfpoint{15.357872\du}{-187.848401\du}}{\pgfpoint{0.054256\du}{0\du}}{\pgfpoint{0\du}{0.054256\du}}
\pgfusepath{stroke}
\pgfsetlinewidth{0.100000\du}
\pgfsetdash{}{0pt}
\pgfsetdash{}{0pt}
\pgfsetbuttcap
\pgfsetmiterjoin
\pgfsetlinewidth{0.100000\du}
\pgfsetbuttcap
\pgfsetmiterjoin
\pgfsetdash{}{0pt}
\definecolor{dialinecolor}{rgb}{0.000000, 0.000000, 0.000000}
\pgfsetfillcolor{dialinecolor}
\pgfpathellipse{\pgfpoint{15.353860\du}{-187.109655\du}}{\pgfpoint{0.054256\du}{0\du}}{\pgfpoint{0\du}{0.054256\du}}
\pgfusepath{fill}
\definecolor{dialinecolor}{rgb}{0.000000, 0.000000, 0.000000}
\pgfsetstrokecolor{dialinecolor}
\pgfpathellipse{\pgfpoint{15.353860\du}{-187.109655\du}}{\pgfpoint{0.054256\du}{0\du}}{\pgfpoint{0\du}{0.054256\du}}
\pgfusepath{stroke}
\pgfsetbuttcap
\pgfsetmiterjoin
\pgfsetdash{}{0pt}
\definecolor{dialinecolor}{rgb}{0.000000, 0.000000, 0.000000}
\pgfsetstrokecolor{dialinecolor}
\pgfpathellipse{\pgfpoint{15.353860\du}{-187.109655\du}}{\pgfpoint{0.054256\du}{0\du}}{\pgfpoint{0\du}{0.054256\du}}
\pgfusepath{stroke}
\pgfsetlinewidth{0.100000\du}
\pgfsetdash{}{0pt}
\pgfsetdash{}{0pt}
\pgfsetbuttcap
\pgfsetmiterjoin
\pgfsetlinewidth{0.100000\du}
\pgfsetbuttcap
\pgfsetmiterjoin
\pgfsetdash{}{0pt}
\definecolor{dialinecolor}{rgb}{0.000000, 0.000000, 0.000000}
\pgfsetfillcolor{dialinecolor}
\pgfpathellipse{\pgfpoint{-0.003305\du}{-187.033939\du}}{\pgfpoint{0.054256\du}{0\du}}{\pgfpoint{0\du}{0.054256\du}}
\pgfusepath{fill}
\definecolor{dialinecolor}{rgb}{0.000000, 0.000000, 0.000000}
\pgfsetstrokecolor{dialinecolor}
\pgfpathellipse{\pgfpoint{-0.003305\du}{-187.033939\du}}{\pgfpoint{0.054256\du}{0\du}}{\pgfpoint{0\du}{0.054256\du}}
\pgfusepath{stroke}
\pgfsetbuttcap
\pgfsetmiterjoin
\pgfsetdash{}{0pt}
\definecolor{dialinecolor}{rgb}{0.000000, 0.000000, 0.000000}
\pgfsetstrokecolor{dialinecolor}
\pgfpathellipse{\pgfpoint{-0.003305\du}{-187.033939\du}}{\pgfpoint{0.054256\du}{0\du}}{\pgfpoint{0\du}{0.054256\du}}
\pgfusepath{stroke}
\pgfsetlinewidth{0.100000\du}
\pgfsetdash{}{0pt}
\pgfsetdash{}{0pt}
\pgfsetbuttcap
\pgfsetmiterjoin
\pgfsetlinewidth{0.100000\du}
\pgfsetbuttcap
\pgfsetmiterjoin
\pgfsetdash{}{0pt}
\definecolor{dialinecolor}{rgb}{0.000000, 0.000000, 0.000000}
\pgfsetfillcolor{dialinecolor}
\pgfpathellipse{\pgfpoint{0.003557\du}{-187.765386\du}}{\pgfpoint{0.054256\du}{0\du}}{\pgfpoint{0\du}{0.054256\du}}
\pgfusepath{fill}
\definecolor{dialinecolor}{rgb}{0.000000, 0.000000, 0.000000}
\pgfsetstrokecolor{dialinecolor}
\pgfpathellipse{\pgfpoint{0.003557\du}{-187.765386\du}}{\pgfpoint{0.054256\du}{0\du}}{\pgfpoint{0\du}{0.054256\du}}
\pgfusepath{stroke}
\pgfsetbuttcap
\pgfsetmiterjoin
\pgfsetdash{}{0pt}
\definecolor{dialinecolor}{rgb}{0.000000, 0.000000, 0.000000}
\pgfsetstrokecolor{dialinecolor}
\pgfpathellipse{\pgfpoint{0.003557\du}{-187.765386\du}}{\pgfpoint{0.054256\du}{0\du}}{\pgfpoint{0\du}{0.054256\du}}
\pgfusepath{stroke}
\pgfsetlinewidth{0.100000\du}
\pgfsetdash{}{0pt}
\pgfsetdash{}{0pt}
\pgfsetbuttcap
\pgfsetmiterjoin
\pgfsetlinewidth{0.100000\du}
\pgfsetbuttcap
\pgfsetmiterjoin
\pgfsetdash{}{0pt}
\definecolor{dialinecolor}{rgb}{0.000000, 0.000000, 0.000000}
\pgfsetfillcolor{dialinecolor}
\pgfpathellipse{\pgfpoint{0.005602\du}{-189.260357\du}}{\pgfpoint{0.054256\du}{0\du}}{\pgfpoint{0\du}{0.054256\du}}
\pgfusepath{fill}
\definecolor{dialinecolor}{rgb}{0.000000, 0.000000, 0.000000}
\pgfsetstrokecolor{dialinecolor}
\pgfpathellipse{\pgfpoint{0.005602\du}{-189.260357\du}}{\pgfpoint{0.054256\du}{0\du}}{\pgfpoint{0\du}{0.054256\du}}
\pgfusepath{stroke}
\pgfsetbuttcap
\pgfsetmiterjoin
\pgfsetdash{}{0pt}
\definecolor{dialinecolor}{rgb}{0.000000, 0.000000, 0.000000}
\pgfsetstrokecolor{dialinecolor}
\pgfpathellipse{\pgfpoint{0.005602\du}{-189.260357\du}}{\pgfpoint{0.054256\du}{0\du}}{\pgfpoint{0\du}{0.054256\du}}
\pgfusepath{stroke}
\pgfsetlinewidth{0.100000\du}
\pgfsetdash{}{0pt}
\pgfsetdash{}{0pt}
\pgfsetbuttcap
\pgfsetmiterjoin
\pgfsetlinewidth{0.100000\du}
\pgfsetbuttcap
\pgfsetmiterjoin
\pgfsetdash{}{0pt}
\definecolor{dialinecolor}{rgb}{0.000000, 0.000000, 0.000000}
\pgfsetfillcolor{dialinecolor}
\pgfpathellipse{\pgfpoint{0.000831\du}{-188.593123\du}}{\pgfpoint{0.054256\du}{0\du}}{\pgfpoint{0\du}{0.054256\du}}
\pgfusepath{fill}
\definecolor{dialinecolor}{rgb}{0.000000, 0.000000, 0.000000}
\pgfsetstrokecolor{dialinecolor}
\pgfpathellipse{\pgfpoint{0.000831\du}{-188.593123\du}}{\pgfpoint{0.054256\du}{0\du}}{\pgfpoint{0\du}{0.054256\du}}
\pgfusepath{stroke}
\pgfsetbuttcap
\pgfsetmiterjoin
\pgfsetdash{}{0pt}
\definecolor{dialinecolor}{rgb}{0.000000, 0.000000, 0.000000}
\pgfsetstrokecolor{dialinecolor}
\pgfpathellipse{\pgfpoint{0.000831\du}{-188.593123\du}}{\pgfpoint{0.054256\du}{0\du}}{\pgfpoint{0\du}{0.054256\du}}
\pgfusepath{stroke}
\definecolor{dialinecolor}{rgb}{1.000000, 1.000000, 1.000000}
\pgfsetfillcolor{dialinecolor}
\fill (3.958726\du,-184.167113\du)--(3.958726\du,-182.267113\du)--(11.988493\du,-182.267113\du)--(11.988493\du,-184.167113\du)--cycle;
\pgfsetlinewidth{0.050000\du}
\pgfsetdash{}{0pt}
\pgfsetdash{}{0pt}
\pgfsetmiterjoin
\definecolor{dialinecolor}{rgb}{0.000000, 0.000000, 0.000000}
\pgfsetstrokecolor{dialinecolor}
\draw (3.958726\du,-184.167113\du)--(3.958726\du,-182.267113\du)--(11.988493\du,-182.267113\du)--(11.988493\du,-184.167113\du)--cycle;
% setfont left to latex
\definecolor{dialinecolor}{rgb}{0.000000, 0.000000, 0.000000}
\pgfsetstrokecolor{dialinecolor}
\node at (7.973609\du,-183.022113\du){Softmax};
\definecolor{dialinecolor}{rgb}{1.000000, 1.000000, 1.000000}
\pgfsetfillcolor{dialinecolor}
\fill (3.996255\du,-180.794707\du)--(3.996255\du,-178.894707\du)--(12.026022\du,-178.894707\du)--(12.026022\du,-180.794707\du)--cycle;
\pgfsetlinewidth{0.050000\du}
\pgfsetdash{}{0pt}
\pgfsetdash{}{0pt}
\pgfsetmiterjoin
\definecolor{dialinecolor}{rgb}{0.000000, 0.000000, 0.000000}
\pgfsetstrokecolor{dialinecolor}
\draw (3.996255\du,-180.794707\du)--(3.996255\du,-178.894707\du)--(12.026022\du,-178.894707\du)--(12.026022\du,-180.794707\du)--cycle;
% setfont left to latex
\definecolor{dialinecolor}{rgb}{0.000000, 0.000000, 0.000000}
\pgfsetstrokecolor{dialinecolor}
\node at (8.011138\du,-179.649707\du){KL Divergence};
\pgfsetlinewidth{0.050000\du}
\pgfsetdash{}{0pt}
\pgfsetdash{}{0pt}
\pgfsetbuttcap
\pgfsetmiterjoin
\pgfsetlinewidth{0.050000\du}
\pgfsetbuttcap
\pgfsetmiterjoin
\pgfsetdash{}{0pt}
\definecolor{dialinecolor}{rgb}{1.000000, 1.000000, 1.000000}
\pgfsetfillcolor{dialinecolor}
\fill (7.876151\du,-185.442041\du)--(7.876151\du,-184.867076\du)--(7.588669\du,-184.867076\du)--(8.163634\du,-184.292111\du)--(8.738599\du,-184.867076\du)--(8.451116\du,-184.867076\du)--(8.451116\du,-185.442041\du)--cycle;
\definecolor{dialinecolor}{rgb}{0.000000, 0.000000, 0.000000}
\pgfsetstrokecolor{dialinecolor}
\draw (7.876151\du,-185.442041\du)--(7.876151\du,-184.867076\du)--(7.588669\du,-184.867076\du)--(8.163634\du,-184.292111\du)--(8.738599\du,-184.867076\du)--(8.451116\du,-184.867076\du)--(8.451116\du,-185.442041\du)--cycle;
\pgfsetbuttcap
\pgfsetmiterjoin
\pgfsetdash{}{0pt}
\definecolor{dialinecolor}{rgb}{0.000000, 0.000000, 0.000000}
\pgfsetstrokecolor{dialinecolor}
\draw (7.876151\du,-185.442041\du)--(7.876151\du,-184.867076\du)--(7.588669\du,-184.867076\du)--(8.163634\du,-184.292111\du)--(8.738599\du,-184.867076\du)--(8.451116\du,-184.867076\du)--(8.451116\du,-185.442041\du)--cycle;
\pgfsetlinewidth{0.050000\du}
\pgfsetdash{}{0pt}
\pgfsetdash{}{0pt}
\pgfsetbuttcap
\pgfsetmiterjoin
\pgfsetlinewidth{0.050000\du}
\pgfsetbuttcap
\pgfsetmiterjoin
\pgfsetdash{}{0pt}
\definecolor{dialinecolor}{rgb}{1.000000, 1.000000, 1.000000}
\pgfsetfillcolor{dialinecolor}
\fill (7.805193\du,-182.104665\du)--(7.805193\du,-181.529700\du)--(7.517710\du,-181.529700\du)--(8.092675\du,-180.954735\du)--(8.667640\du,-181.529700\du)--(8.380158\du,-181.529700\du)--(8.380158\du,-182.104665\du)--cycle;
\definecolor{dialinecolor}{rgb}{0.000000, 0.000000, 0.000000}
\pgfsetstrokecolor{dialinecolor}
\draw (7.805193\du,-182.104665\du)--(7.805193\du,-181.529700\du)--(7.517710\du,-181.529700\du)--(8.092675\du,-180.954735\du)--(8.667640\du,-181.529700\du)--(8.380158\du,-181.529700\du)--(8.380158\du,-182.104665\du)--cycle;
\pgfsetbuttcap
\pgfsetmiterjoin
\pgfsetdash{}{0pt}
\definecolor{dialinecolor}{rgb}{0.000000, 0.000000, 0.000000}
\pgfsetstrokecolor{dialinecolor}
\draw (7.805193\du,-182.104665\du)--(7.805193\du,-181.529700\du)--(7.517710\du,-181.529700\du)--(8.092675\du,-180.954735\du)--(8.667640\du,-181.529700\du)--(8.380158\du,-181.529700\du)--(8.380158\du,-182.104665\du)--cycle;
\pgfsetlinewidth{0.100000\du}
\pgfsetdash{}{0pt}
\pgfsetdash{}{0pt}
\pgfsetmiterjoin
\definecolor{dialinecolor}{rgb}{0.000000, 0.000000, 0.000000}
\pgfsetstrokecolor{dialinecolor}
\draw (-4.106773\du,-198.006085\du)--(-4.106773\du,-178.384799\du)--(19.141478\du,-178.384799\du)--(19.141478\du,-198.006085\du)--cycle;
\end{tikzpicture}

  \caption{\label{fig:archi}Architecture}
\end{figure}
\subsubsection{Training Details}
\subsubsection{Noising Text}
The purpose of the experiment is to rediscover the different classes of the
corpus $C1$ with keywords and background knowledge.
