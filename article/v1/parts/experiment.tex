\section{Experiment}

\subsection{Data}
To experiment our algoritm we use the dataset 20NewsGroup \cite{Newsgroups20}.
The 20 Newsgroups data set is a collection of approximately 20,000 newsgroup 
documents, partitioned evenly across 20 different newsgroups.\\
Each document are represented by a vector using frequency-inverse document 
frequency (TFIDF) representation.
The term frequency-inverse document frequency is a method of weghting depicting 
the significiance of each word in a document rather a corpus.
\begin{equation}
TF(t, X) = \frac{f_{t, X}}{max_{t' \in C}f_{t', X}} 
\end{equation}
\begin{equation}
IDF(t, C) = log(\frac{N}{|X \in C : t \in X|})
\end{equation}
\begin{equation}
TFIDF(t,c,C) = TF(t, X) . IDF(t, C)   
\end{equation}
\subsection{Generate Constraint}
\subsubsection{Lexical Constraints}
To generate the set of keywords $KW$ we rank each word of each 
document of each class using TFIDF~\ref{algo:gen_kw}
\begin{algorithm}
  \SetKwInOut{Input}{input}
  \SetKwInOut{Output}{output}
  \Input{Corpus C, The number of keywords per classes $P$}
  \Output{KW}
  $KW \gets \{\}$\\
  \ForEach{Class $c_i \in C$}{
    $rank \gets [0 ... 0]$\\
    \ForEach{Document $X \in c_i$}{
      \ForEach{Word $w \in X$}{
        $rank_w \gets TFIDF(w)$\\
      }
    }
    $KW \gets KW \cup \{\{w_1, w_2 ... w_P\} : \not\exists (v_1, v_2) | v_1 \not\in 
    \{w_1, w_2 ... w_P\}, v_2 \in \{w_1, w_2 ... w_P\}, rank_{v_1} \ge rank_{v_2}\}$\\
  }
  \Return{KW}
  \caption{\label{algo:gen_kw}Extract Keywords}
\end{algorithm}
\subsubsection{Background Knowledge}
We generate pairwise constraints randomly~\ref{algo:gen_pair}
\begin{algorithm}[!h]
  \SetKwInOut{Input}{input}
  \SetKwInOut{Output}{output}
  \Input{Corpus C, The set of labels L, The number of pair $N_p$}
  \Output{Must-Link Pair ML, Cannot-Link Pair CL}
  \For{i = 1 : $N_p$}{
    Choose randomly ($X_i, X_j$)\\
    \If{$L_i == L_j$}{
      Insert ($X_i, X_j$) in ML 
    }
  }
  \For{i = 1 : $N_p$}{
    Choose randomly ($X_i, X_j$)\\
    \If{$L_i != L_j$}{
      Insert ($X_i, X_j$) in CL 
    }
  }
  \Return{ML, CL}
  \caption{\label{algo:gen_pair}Extract Pair}
\end{algorithm}
\subsection{Evaluation}
\subsubsection{Reference Algorithm}
We evaluate our algorithm with four algorithms :
\begin{itemize}
\item \textbf{COPKmeans} proposed by Wagstaff \cite{Wagstaff:2001:CKC:645530.655669}.
\item \textbf{COPKmeans with lexical constraints}, we learn a $k$-means
  friendly space taking into account lexical constraints. The loss
  function for this model is :
  \begin{equation}
  Min~L(KW, C, K; \theta) = (L_{rec}(C, \theta) + \omega_{KW} )+
  \lambda L_{clust}(C,K)
  \end{equation}
\item \textbf{Deep $K$-Means} see in section 2.3
\end{itemize}
\subsubsection{Metric}
To evaluate our algorithm and compare results with reference algorithms we can
use the NMI Metric, Accuracy Metric \cite{NMI_ACC}, and Adjusted
Rand index\cite{ARI}. 
\begin{itemize}
\item The NMI Metric is defined as follows
$$NMI(S,C) = \frac{I(S,C)}{[H(S)+H(C)]/2}$$ 
with
$I(S,C) =\sum_k \sum_f\frac{|s_k \cap c_f|}{N}log\frac{N|s_k \cap c_f|}{|s_k| |c_f|}$
and
$H(S) = -\sum_k\frac{|s_k|}{N}log\frac{N|s_k|}{|s_k|}$
\item The Accuracy is the proportion of true results among the total
  number of cases examined. The Accuracy metric is defined as follows :
$$
ACC(S,C) = \frac{1}{N}\sum_k {max}_j|s_k \cap c_j|
$$
\item Let a : the number of pairs of document in C
  that are in the same cluster in the predicted partition and in the
  same cluster in the real partition, and b : the number of pairs of
  document in C that are in different clusters in predicted partition
  and in different cluster in real partition.
  The Adjusted Rand index is defined as follows :
  $$ARI = \frac{a+b}{\binom{N}{2}}$$
\end{itemize}
\subsection{Experimental Setup}
\subsubsection{Autoencoder Architecture}
We use the same architecture used in~\cite{Deap-K-Means}. The encoder is a fully-connected 
multilayer perceptron formed by 3 hidden layers (with dimensions 500, 500, 2000) 
and an embedding layer (with dimension K, the number of cluster). 
The decoder is a mirrored version of the encoder~\ref{fig:archi}.
The ReLu activation function is used on layers, except for the third
hidden layer of encoder and decoder part.  
\begin{figure}[!h]
  \centering
  \tikzset{every picture/.style={scale=1.6}}
  % Graphic for TeX using PGF
% Title: /home/maxence/Documents/stage/article/v1/ressources/dia_file/archi.dia
% Creator: Dia v0.97.3
% CreationDate: Wed May 30 23:21:17 2018
% For: maxence
% \usepackage{tikz}
% The following commands are not supported in PSTricks at present
% We define them conditionally, so when they are implemented,
% this pgf file will use them.
\ifx\du\undefined
  \newlength{\du}
\fi
\setlength{\du}{15\unitlength}
\begin{tikzpicture}
\pgftransformxscale{1.000000}
\pgftransformyscale{-1.000000}
\definecolor{dialinecolor}{rgb}{0.000000, 0.000000, 0.000000}
\pgfsetstrokecolor{dialinecolor}
\definecolor{dialinecolor}{rgb}{1.000000, 1.000000, 1.000000}
\pgfsetfillcolor{dialinecolor}
\pgfsetlinewidth{0.050000\du}
\pgfsetdash{}{0pt}
\pgfsetdash{}{0pt}
\pgfsetbuttcap
\pgfsetmiterjoin
\pgfsetlinewidth{0.050000\du}
\pgfsetbuttcap
\pgfsetmiterjoin
\pgfsetdash{}{0pt}
\definecolor{dialinecolor}{rgb}{0.000000, 0.000000, 0.000000}
\pgfsetstrokecolor{dialinecolor}
\pgfpathellipse{\pgfpoint{18.426580\du}{13.409018\du}}{\pgfpoint{0.380944\du}{0\du}}{\pgfpoint{0\du}{0.380944\du}}
\pgfusepath{stroke}
\pgfsetbuttcap
\pgfsetmiterjoin
\pgfsetdash{}{0pt}
\definecolor{dialinecolor}{rgb}{0.000000, 0.000000, 0.000000}
\pgfsetstrokecolor{dialinecolor}
\pgfpathellipse{\pgfpoint{18.426580\du}{13.409018\du}}{\pgfpoint{0.380944\du}{0\du}}{\pgfpoint{0\du}{0.380944\du}}
\pgfusepath{stroke}
\pgfsetlinewidth{0.050000\du}
\pgfsetdash{}{0pt}
\pgfsetdash{}{0pt}
\pgfsetbuttcap
\pgfsetmiterjoin
\pgfsetlinewidth{0.050000\du}
\pgfsetbuttcap
\pgfsetmiterjoin
\pgfsetdash{}{0pt}
\definecolor{dialinecolor}{rgb}{0.000000, 0.000000, 0.000000}
\pgfsetstrokecolor{dialinecolor}
\pgfpathellipse{\pgfpoint{18.426300\du}{14.460483\du}}{\pgfpoint{0.380944\du}{0\du}}{\pgfpoint{0\du}{0.380944\du}}
\pgfusepath{stroke}
\pgfsetbuttcap
\pgfsetmiterjoin
\pgfsetdash{}{0pt}
\definecolor{dialinecolor}{rgb}{0.000000, 0.000000, 0.000000}
\pgfsetstrokecolor{dialinecolor}
\pgfpathellipse{\pgfpoint{18.426300\du}{14.460483\du}}{\pgfpoint{0.380944\du}{0\du}}{\pgfpoint{0\du}{0.380944\du}}
\pgfusepath{stroke}
\pgfsetlinewidth{0.050000\du}
\pgfsetdash{}{0pt}
\pgfsetdash{}{0pt}
\pgfsetbuttcap
\pgfsetmiterjoin
\pgfsetlinewidth{0.050000\du}
\pgfsetbuttcap
\pgfsetmiterjoin
\pgfsetdash{}{0pt}
\definecolor{dialinecolor}{rgb}{0.000000, 0.000000, 0.000000}
\pgfsetstrokecolor{dialinecolor}
\pgfpathellipse{\pgfpoint{18.423869\du}{18.551263\du}}{\pgfpoint{0.380944\du}{0\du}}{\pgfpoint{0\du}{0.380944\du}}
\pgfusepath{stroke}
\pgfsetbuttcap
\pgfsetmiterjoin
\pgfsetdash{}{0pt}
\definecolor{dialinecolor}{rgb}{0.000000, 0.000000, 0.000000}
\pgfsetstrokecolor{dialinecolor}
\pgfpathellipse{\pgfpoint{18.423869\du}{18.551263\du}}{\pgfpoint{0.380944\du}{0\du}}{\pgfpoint{0\du}{0.380944\du}}
\pgfusepath{stroke}
\pgfsetlinewidth{0.050000\du}
\pgfsetdash{}{0pt}
\pgfsetdash{}{0pt}
\pgfsetbuttcap
\pgfsetmiterjoin
\pgfsetlinewidth{0.050000\du}
\pgfsetbuttcap
\pgfsetmiterjoin
\pgfsetdash{}{0pt}
\definecolor{dialinecolor}{rgb}{0.000000, 0.000000, 0.000000}
\pgfsetstrokecolor{dialinecolor}
\pgfpathellipse{\pgfpoint{20.392625\du}{14.958377\du}}{\pgfpoint{0.380944\du}{0\du}}{\pgfpoint{0\du}{0.380944\du}}
\pgfusepath{stroke}
\pgfsetbuttcap
\pgfsetmiterjoin
\pgfsetdash{}{0pt}
\definecolor{dialinecolor}{rgb}{0.000000, 0.000000, 0.000000}
\pgfsetstrokecolor{dialinecolor}
\pgfpathellipse{\pgfpoint{20.392625\du}{14.958377\du}}{\pgfpoint{0.380944\du}{0\du}}{\pgfpoint{0\du}{0.380944\du}}
\pgfusepath{stroke}
\pgfsetlinewidth{0.050000\du}
\pgfsetdash{}{0pt}
\pgfsetdash{}{0pt}
\pgfsetbuttcap
\pgfsetmiterjoin
\pgfsetlinewidth{0.050000\du}
\pgfsetbuttcap
\pgfsetmiterjoin
\pgfsetdash{}{0pt}
\definecolor{dialinecolor}{rgb}{0.000000, 0.000000, 0.000000}
\pgfsetstrokecolor{dialinecolor}
\pgfpathellipse{\pgfpoint{20.422677\du}{17.028417\du}}{\pgfpoint{0.380944\du}{0\du}}{\pgfpoint{0\du}{0.380944\du}}
\pgfusepath{stroke}
\pgfsetbuttcap
\pgfsetmiterjoin
\pgfsetdash{}{0pt}
\definecolor{dialinecolor}{rgb}{0.000000, 0.000000, 0.000000}
\pgfsetstrokecolor{dialinecolor}
\pgfpathellipse{\pgfpoint{20.422677\du}{17.028417\du}}{\pgfpoint{0.380944\du}{0\du}}{\pgfpoint{0\du}{0.380944\du}}
\pgfusepath{stroke}
\pgfsetlinewidth{0.050000\du}
\pgfsetdash{}{0pt}
\pgfsetdash{}{0pt}
\pgfsetbuttcap
\pgfsetmiterjoin
\pgfsetlinewidth{0.050000\du}
\pgfsetbuttcap
\pgfsetmiterjoin
\pgfsetdash{}{0pt}
\definecolor{dialinecolor}{rgb}{0.000000, 0.000000, 0.000000}
\pgfsetstrokecolor{dialinecolor}
\pgfpathellipse{\pgfpoint{22.008352\du}{14.997267\du}}{\pgfpoint{0.380944\du}{0\du}}{\pgfpoint{0\du}{0.380944\du}}
\pgfusepath{stroke}
\pgfsetbuttcap
\pgfsetmiterjoin
\pgfsetdash{}{0pt}
\definecolor{dialinecolor}{rgb}{0.000000, 0.000000, 0.000000}
\pgfsetstrokecolor{dialinecolor}
\pgfpathellipse{\pgfpoint{22.008352\du}{14.997267\du}}{\pgfpoint{0.380944\du}{0\du}}{\pgfpoint{0\du}{0.380944\du}}
\pgfusepath{stroke}
\pgfsetlinewidth{0.050000\du}
\pgfsetdash{}{0pt}
\pgfsetdash{}{0pt}
\pgfsetbuttcap
\pgfsetmiterjoin
\pgfsetlinewidth{0.050000\du}
\pgfsetbuttcap
\pgfsetmiterjoin
\pgfsetdash{}{0pt}
\definecolor{dialinecolor}{rgb}{0.000000, 0.000000, 0.000000}
\pgfsetstrokecolor{dialinecolor}
\pgfpathellipse{\pgfpoint{22.003049\du}{17.049630\du}}{\pgfpoint{0.380944\du}{0\du}}{\pgfpoint{0\du}{0.380944\du}}
\pgfusepath{stroke}
\pgfsetbuttcap
\pgfsetmiterjoin
\pgfsetdash{}{0pt}
\definecolor{dialinecolor}{rgb}{0.000000, 0.000000, 0.000000}
\pgfsetstrokecolor{dialinecolor}
\pgfpathellipse{\pgfpoint{22.003049\du}{17.049630\du}}{\pgfpoint{0.380944\du}{0\du}}{\pgfpoint{0\du}{0.380944\du}}
\pgfusepath{stroke}
\pgfsetlinewidth{0.050000\du}
\pgfsetdash{}{0pt}
\pgfsetdash{}{0pt}
\pgfsetbuttcap
\pgfsetmiterjoin
\pgfsetlinewidth{0.050000\du}
\pgfsetbuttcap
\pgfsetmiterjoin
\pgfsetdash{}{0pt}
\definecolor{dialinecolor}{rgb}{0.000000, 0.000000, 0.000000}
\pgfsetstrokecolor{dialinecolor}
\pgfpathellipse{\pgfpoint{23.995307\du}{13.657310\du}}{\pgfpoint{0.380944\du}{0\du}}{\pgfpoint{0\du}{0.380944\du}}
\pgfusepath{stroke}
\pgfsetbuttcap
\pgfsetmiterjoin
\pgfsetdash{}{0pt}
\definecolor{dialinecolor}{rgb}{0.000000, 0.000000, 0.000000}
\pgfsetstrokecolor{dialinecolor}
\pgfpathellipse{\pgfpoint{23.995307\du}{13.657310\du}}{\pgfpoint{0.380944\du}{0\du}}{\pgfpoint{0\du}{0.380944\du}}
\pgfusepath{stroke}
\pgfsetlinewidth{0.050000\du}
\pgfsetdash{}{0pt}
\pgfsetdash{}{0pt}
\pgfsetbuttcap
\pgfsetmiterjoin
\pgfsetlinewidth{0.050000\du}
\pgfsetbuttcap
\pgfsetmiterjoin
\pgfsetdash{}{0pt}
\definecolor{dialinecolor}{rgb}{0.000000, 0.000000, 0.000000}
\pgfsetstrokecolor{dialinecolor}
\pgfpathellipse{\pgfpoint{24.007682\du}{18.414336\du}}{\pgfpoint{0.380944\du}{0\du}}{\pgfpoint{0\du}{0.380944\du}}
\pgfusepath{stroke}
\pgfsetbuttcap
\pgfsetmiterjoin
\pgfsetdash{}{0pt}
\definecolor{dialinecolor}{rgb}{0.000000, 0.000000, 0.000000}
\pgfsetstrokecolor{dialinecolor}
\pgfpathellipse{\pgfpoint{24.007682\du}{18.414336\du}}{\pgfpoint{0.380944\du}{0\du}}{\pgfpoint{0\du}{0.380944\du}}
\pgfusepath{stroke}
\pgfsetlinewidth{0.050000\du}
\pgfsetdash{}{0pt}
\pgfsetdash{}{0pt}
\pgfsetbuttcap
\pgfsetmiterjoin
\pgfsetlinewidth{0.050000\du}
\pgfsetbuttcap
\pgfsetmiterjoin
\pgfsetdash{}{0pt}
\definecolor{dialinecolor}{rgb}{0.000000, 0.000000, 0.000000}
\pgfsetstrokecolor{dialinecolor}
\pgfpathellipse{\pgfpoint{25.558002\du}{15.428599\du}}{\pgfpoint{0.380944\du}{0\du}}{\pgfpoint{0\du}{0.380944\du}}
\pgfusepath{stroke}
\pgfsetbuttcap
\pgfsetmiterjoin
\pgfsetdash{}{0pt}
\definecolor{dialinecolor}{rgb}{0.000000, 0.000000, 0.000000}
\pgfsetstrokecolor{dialinecolor}
\pgfpathellipse{\pgfpoint{25.558002\du}{15.428599\du}}{\pgfpoint{0.380944\du}{0\du}}{\pgfpoint{0\du}{0.380944\du}}
\pgfusepath{stroke}
\pgfsetlinewidth{0.050000\du}
\pgfsetdash{}{0pt}
\pgfsetdash{}{0pt}
\pgfsetbuttcap
\pgfsetmiterjoin
\pgfsetlinewidth{0.050000\du}
\pgfsetbuttcap
\pgfsetmiterjoin
\pgfsetdash{}{0pt}
\definecolor{dialinecolor}{rgb}{0.000000, 0.000000, 0.000000}
\pgfsetstrokecolor{dialinecolor}
\pgfpathellipse{\pgfpoint{25.588054\du}{16.579407\du}}{\pgfpoint{0.380944\du}{0\du}}{\pgfpoint{0\du}{0.380944\du}}
\pgfusepath{stroke}
\pgfsetbuttcap
\pgfsetmiterjoin
\pgfsetdash{}{0pt}
\definecolor{dialinecolor}{rgb}{0.000000, 0.000000, 0.000000}
\pgfsetstrokecolor{dialinecolor}
\pgfpathellipse{\pgfpoint{25.588054\du}{16.579407\du}}{\pgfpoint{0.380944\du}{0\du}}{\pgfpoint{0\du}{0.380944\du}}
\pgfusepath{stroke}
\pgfsetlinewidth{0.050000\du}
\pgfsetdash{}{0pt}
\pgfsetdash{}{0pt}
\pgfsetbuttcap
\pgfsetmiterjoin
\pgfsetlinewidth{0.050000\du}
\pgfsetbuttcap
\pgfsetmiterjoin
\pgfsetdash{}{0pt}
\definecolor{dialinecolor}{rgb}{0.000000, 0.000000, 0.000000}
\pgfsetstrokecolor{dialinecolor}
\pgfpathellipse{\pgfpoint{26.996954\du}{13.593670\du}}{\pgfpoint{0.380944\du}{0\du}}{\pgfpoint{0\du}{0.380944\du}}
\pgfusepath{stroke}
\pgfsetbuttcap
\pgfsetmiterjoin
\pgfsetdash{}{0pt}
\definecolor{dialinecolor}{rgb}{0.000000, 0.000000, 0.000000}
\pgfsetstrokecolor{dialinecolor}
\pgfpathellipse{\pgfpoint{26.996954\du}{13.593670\du}}{\pgfpoint{0.380944\du}{0\du}}{\pgfpoint{0\du}{0.380944\du}}
\pgfusepath{stroke}
\pgfsetlinewidth{0.050000\du}
\pgfsetdash{}{0pt}
\pgfsetdash{}{0pt}
\pgfsetbuttcap
\pgfsetmiterjoin
\pgfsetlinewidth{0.050000\du}
\pgfsetbuttcap
\pgfsetmiterjoin
\pgfsetdash{}{0pt}
\definecolor{dialinecolor}{rgb}{0.000000, 0.000000, 0.000000}
\pgfsetstrokecolor{dialinecolor}
\pgfpathellipse{\pgfpoint{26.986347\du}{18.405497\du}}{\pgfpoint{0.380944\du}{0\du}}{\pgfpoint{0\du}{0.380944\du}}
\pgfusepath{stroke}
\pgfsetbuttcap
\pgfsetmiterjoin
\pgfsetdash{}{0pt}
\definecolor{dialinecolor}{rgb}{0.000000, 0.000000, 0.000000}
\pgfsetstrokecolor{dialinecolor}
\pgfpathellipse{\pgfpoint{26.986347\du}{18.405497\du}}{\pgfpoint{0.380944\du}{0\du}}{\pgfpoint{0\du}{0.380944\du}}
\pgfusepath{stroke}
\pgfsetlinewidth{0.050000\du}
\pgfsetdash{}{0pt}
\pgfsetdash{}{0pt}
\pgfsetbuttcap
\pgfsetmiterjoin
\pgfsetlinewidth{0.050000\du}
\pgfsetbuttcap
\pgfsetmiterjoin
\pgfsetdash{}{0pt}
\definecolor{dialinecolor}{rgb}{0.000000, 0.000000, 0.000000}
\pgfsetstrokecolor{dialinecolor}
\pgfpathellipse{\pgfpoint{28.996284\du}{14.977822\du}}{\pgfpoint{0.380944\du}{0\du}}{\pgfpoint{0\du}{0.380944\du}}
\pgfusepath{stroke}
\pgfsetbuttcap
\pgfsetmiterjoin
\pgfsetdash{}{0pt}
\definecolor{dialinecolor}{rgb}{0.000000, 0.000000, 0.000000}
\pgfsetstrokecolor{dialinecolor}
\pgfpathellipse{\pgfpoint{28.996284\du}{14.977822\du}}{\pgfpoint{0.380944\du}{0\du}}{\pgfpoint{0\du}{0.380944\du}}
\pgfusepath{stroke}
\pgfsetlinewidth{0.050000\du}
\pgfsetdash{}{0pt}
\pgfsetdash{}{0pt}
\pgfsetbuttcap
\pgfsetmiterjoin
\pgfsetlinewidth{0.050000\du}
\pgfsetbuttcap
\pgfsetmiterjoin
\pgfsetdash{}{0pt}
\definecolor{dialinecolor}{rgb}{0.000000, 0.000000, 0.000000}
\pgfsetstrokecolor{dialinecolor}
\pgfpathellipse{\pgfpoint{29.008658\du}{16.977152\du}}{\pgfpoint{0.380944\du}{0\du}}{\pgfpoint{0\du}{0.380944\du}}
\pgfusepath{stroke}
\pgfsetbuttcap
\pgfsetmiterjoin
\pgfsetdash{}{0pt}
\definecolor{dialinecolor}{rgb}{0.000000, 0.000000, 0.000000}
\pgfsetstrokecolor{dialinecolor}
\pgfpathellipse{\pgfpoint{29.008658\du}{16.977152\du}}{\pgfpoint{0.380944\du}{0\du}}{\pgfpoint{0\du}{0.380944\du}}
\pgfusepath{stroke}
\pgfsetlinewidth{0.050000\du}
\pgfsetdash{}{0pt}
\pgfsetdash{}{0pt}
\pgfsetbuttcap
\pgfsetmiterjoin
\pgfsetlinewidth{0.050000\du}
\pgfsetbuttcap
\pgfsetmiterjoin
\pgfsetdash{}{0pt}
\definecolor{dialinecolor}{rgb}{0.000000, 0.000000, 0.000000}
\pgfsetstrokecolor{dialinecolor}
\pgfpathellipse{\pgfpoint{31.036272\du}{14.999035\du}}{\pgfpoint{0.380944\du}{0\du}}{\pgfpoint{0\du}{0.380944\du}}
\pgfusepath{stroke}
\pgfsetbuttcap
\pgfsetmiterjoin
\pgfsetdash{}{0pt}
\definecolor{dialinecolor}{rgb}{0.000000, 0.000000, 0.000000}
\pgfsetstrokecolor{dialinecolor}
\pgfpathellipse{\pgfpoint{31.036272\du}{14.999035\du}}{\pgfpoint{0.380944\du}{0\du}}{\pgfpoint{0\du}{0.380944\du}}
\pgfusepath{stroke}
\pgfsetlinewidth{0.050000\du}
\pgfsetdash{}{0pt}
\pgfsetdash{}{0pt}
\pgfsetbuttcap
\pgfsetmiterjoin
\pgfsetlinewidth{0.050000\du}
\pgfsetbuttcap
\pgfsetmiterjoin
\pgfsetdash{}{0pt}
\definecolor{dialinecolor}{rgb}{0.000000, 0.000000, 0.000000}
\pgfsetstrokecolor{dialinecolor}
\pgfpathellipse{\pgfpoint{31.048646\du}{16.998365\du}}{\pgfpoint{0.380944\du}{0\du}}{\pgfpoint{0\du}{0.380944\du}}
\pgfusepath{stroke}
\pgfsetbuttcap
\pgfsetmiterjoin
\pgfsetdash{}{0pt}
\definecolor{dialinecolor}{rgb}{0.000000, 0.000000, 0.000000}
\pgfsetstrokecolor{dialinecolor}
\pgfpathellipse{\pgfpoint{31.048646\du}{16.998365\du}}{\pgfpoint{0.380944\du}{0\du}}{\pgfpoint{0\du}{0.380944\du}}
\pgfusepath{stroke}
\pgfsetlinewidth{0.050000\du}
\pgfsetdash{}{0pt}
\pgfsetdash{}{0pt}
\pgfsetbuttcap
\pgfsetmiterjoin
\pgfsetlinewidth{0.050000\du}
\pgfsetbuttcap
\pgfsetmiterjoin
\pgfsetdash{}{0pt}
\definecolor{dialinecolor}{rgb}{0.000000, 0.000000, 0.000000}
\pgfsetstrokecolor{dialinecolor}
\pgfpathellipse{\pgfpoint{32.581289\du}{13.393914\du}}{\pgfpoint{0.380944\du}{0\du}}{\pgfpoint{0\du}{0.380944\du}}
\pgfusepath{stroke}
\pgfsetbuttcap
\pgfsetmiterjoin
\pgfsetdash{}{0pt}
\definecolor{dialinecolor}{rgb}{0.000000, 0.000000, 0.000000}
\pgfsetstrokecolor{dialinecolor}
\pgfpathellipse{\pgfpoint{32.581289\du}{13.393914\du}}{\pgfpoint{0.380944\du}{0\du}}{\pgfpoint{0\du}{0.380944\du}}
\pgfusepath{stroke}
\pgfsetlinewidth{0.050000\du}
\pgfsetdash{}{0pt}
\pgfsetdash{}{0pt}
\pgfsetbuttcap
\pgfsetmiterjoin
\pgfsetlinewidth{0.050000\du}
\pgfsetbuttcap
\pgfsetmiterjoin
\pgfsetdash{}{0pt}
\definecolor{dialinecolor}{rgb}{0.000000, 0.000000, 0.000000}
\pgfsetstrokecolor{dialinecolor}
\pgfpathellipse{\pgfpoint{32.575986\du}{18.557524\du}}{\pgfpoint{0.380944\du}{0\du}}{\pgfpoint{0\du}{0.380944\du}}
\pgfusepath{stroke}
\pgfsetbuttcap
\pgfsetmiterjoin
\pgfsetdash{}{0pt}
\definecolor{dialinecolor}{rgb}{0.000000, 0.000000, 0.000000}
\pgfsetstrokecolor{dialinecolor}
\pgfpathellipse{\pgfpoint{32.575986\du}{18.557524\du}}{\pgfpoint{0.380944\du}{0\du}}{\pgfpoint{0\du}{0.380944\du}}
\pgfusepath{stroke}
\pgfsetlinewidth{0.050000\du}
\pgfsetdash{}{0pt}
\pgfsetdash{}{0pt}
\pgfsetbuttcap
{
\definecolor{dialinecolor}{rgb}{0.000000, 0.000000, 0.000000}
\pgfsetfillcolor{dialinecolor}
% was here!!!
\pgfsetarrowsend{stealth}
\definecolor{dialinecolor}{rgb}{0.000000, 0.000000, 0.000000}
\pgfsetstrokecolor{dialinecolor}
\draw (18.745534\du,13.660372\du)--(20.073670\du,14.707022\du);
}
\pgfsetlinewidth{0.050000\du}
\pgfsetdash{}{0pt}
\pgfsetdash{}{0pt}
\pgfsetbuttcap
{
\definecolor{dialinecolor}{rgb}{0.000000, 0.000000, 0.000000}
\pgfsetfillcolor{dialinecolor}
% was here!!!
\pgfsetarrowsend{stealth}
\definecolor{dialinecolor}{rgb}{0.000000, 0.000000, 0.000000}
\pgfsetstrokecolor{dialinecolor}
\draw (18.818989\du,14.559916\du)--(19.999936\du,14.858944\du);
}
\pgfsetlinewidth{0.050000\du}
\pgfsetdash{}{0pt}
\pgfsetdash{}{0pt}
\pgfsetbuttcap
{
\definecolor{dialinecolor}{rgb}{0.000000, 0.000000, 0.000000}
\pgfsetfillcolor{dialinecolor}
% was here!!!
\pgfsetarrowsend{stealth}
\definecolor{dialinecolor}{rgb}{0.000000, 0.000000, 0.000000}
\pgfsetstrokecolor{dialinecolor}
\draw (18.618533\du,18.196009\du)--(20.197960\du,15.313630\du);
}
\pgfsetlinewidth{0.050000\du}
\pgfsetdash{}{0pt}
\pgfsetdash{}{0pt}
\pgfsetbuttcap
{
\definecolor{dialinecolor}{rgb}{0.000000, 0.000000, 0.000000}
\pgfsetfillcolor{dialinecolor}
% was here!!!
\pgfsetarrowsend{stealth}
\definecolor{dialinecolor}{rgb}{0.000000, 0.000000, 0.000000}
\pgfsetstrokecolor{dialinecolor}
\draw (18.622730\du,13.764684\du)--(20.226527\du,16.672750\du);
}
\pgfsetlinewidth{0.050000\du}
\pgfsetdash{}{0pt}
\pgfsetdash{}{0pt}
\pgfsetbuttcap
{
\definecolor{dialinecolor}{rgb}{0.000000, 0.000000, 0.000000}
\pgfsetfillcolor{dialinecolor}
% was here!!!
\pgfsetarrowsend{stealth}
\definecolor{dialinecolor}{rgb}{0.000000, 0.000000, 0.000000}
\pgfsetstrokecolor{dialinecolor}
\draw (18.551074\du,14.620979\du)--(20.297903\du,16.867921\du);
}
\pgfsetlinewidth{0.050000\du}
\pgfsetdash{}{0pt}
\pgfsetdash{}{0pt}
\pgfsetbuttcap
{
\definecolor{dialinecolor}{rgb}{0.000000, 0.000000, 0.000000}
\pgfsetfillcolor{dialinecolor}
% was here!!!
\pgfsetarrowsend{stealth}
\definecolor{dialinecolor}{rgb}{0.000000, 0.000000, 0.000000}
\pgfsetstrokecolor{dialinecolor}
\draw (18.746430\du,18.305511\du)--(20.100115\du,17.274169\du);
}
\pgfsetlinewidth{0.050000\du}
\pgfsetdash{}{0pt}
\pgfsetdash{}{0pt}
\pgfsetbuttcap
{
\definecolor{dialinecolor}{rgb}{0.000000, 0.000000, 0.000000}
\pgfsetfillcolor{dialinecolor}
% was here!!!
\pgfsetarrowsend{stealth}
\definecolor{dialinecolor}{rgb}{0.000000, 0.000000, 0.000000}
\pgfsetstrokecolor{dialinecolor}
\draw (20.798529\du,14.968147\du)--(21.602448\du,14.987497\du);
}
\pgfsetlinewidth{0.050000\du}
\pgfsetdash{}{0pt}
\pgfsetdash{}{0pt}
\pgfsetbuttcap
{
\definecolor{dialinecolor}{rgb}{0.000000, 0.000000, 0.000000}
\pgfsetfillcolor{dialinecolor}
% was here!!!
\pgfsetarrowsend{stealth}
\definecolor{dialinecolor}{rgb}{0.000000, 0.000000, 0.000000}
\pgfsetstrokecolor{dialinecolor}
\draw (20.639929\du,15.279519\du)--(21.755745\du,16.728487\du);
}
\pgfsetlinewidth{0.050000\du}
\pgfsetdash{}{0pt}
\pgfsetdash{}{0pt}
\pgfsetbuttcap
{
\definecolor{dialinecolor}{rgb}{0.000000, 0.000000, 0.000000}
\pgfsetfillcolor{dialinecolor}
% was here!!!
\pgfsetarrowsend{stealth}
\definecolor{dialinecolor}{rgb}{0.000000, 0.000000, 0.000000}
\pgfsetstrokecolor{dialinecolor}
\draw (20.672374\du,16.708570\du)--(21.758654\du,15.317114\du);
}
\pgfsetlinewidth{0.050000\du}
\pgfsetdash{}{0pt}
\pgfsetdash{}{0pt}
\pgfsetbuttcap
{
\definecolor{dialinecolor}{rgb}{0.000000, 0.000000, 0.000000}
\pgfsetfillcolor{dialinecolor}
% was here!!!
\pgfsetarrowsend{stealth}
\definecolor{dialinecolor}{rgb}{0.000000, 0.000000, 0.000000}
\pgfsetstrokecolor{dialinecolor}
\draw (20.828959\du,17.033870\du)--(21.596766\du,17.044176\du);
}
\pgfsetlinewidth{0.050000\du}
\pgfsetdash{}{0pt}
\pgfsetdash{}{0pt}
\pgfsetbuttcap
{
\definecolor{dialinecolor}{rgb}{0.000000, 0.000000, 0.000000}
\pgfsetfillcolor{dialinecolor}
% was here!!!
\pgfsetarrowsend{stealth}
\definecolor{dialinecolor}{rgb}{0.000000, 0.000000, 0.000000}
\pgfsetstrokecolor{dialinecolor}
\draw (22.344524\du,14.770560\du)--(23.659136\du,13.884016\du);
}
\pgfsetlinewidth{0.050000\du}
\pgfsetdash{}{0pt}
\pgfsetdash{}{0pt}
\pgfsetbuttcap
{
\definecolor{dialinecolor}{rgb}{0.000000, 0.000000, 0.000000}
\pgfsetfillcolor{dialinecolor}
% was here!!!
\pgfsetarrowsend{stealth}
\definecolor{dialinecolor}{rgb}{0.000000, 0.000000, 0.000000}
\pgfsetstrokecolor{dialinecolor}
\draw (22.208549\du,16.699714\du)--(23.789807\du,14.007225\du);
}
\pgfsetlinewidth{0.050000\du}
\pgfsetdash{}{0pt}
\pgfsetdash{}{0pt}
\pgfsetbuttcap
{
\definecolor{dialinecolor}{rgb}{0.000000, 0.000000, 0.000000}
\pgfsetfillcolor{dialinecolor}
% was here!!!
\pgfsetarrowsend{stealth}
\definecolor{dialinecolor}{rgb}{0.000000, 0.000000, 0.000000}
\pgfsetstrokecolor{dialinecolor}
\draw (22.212385\du,15.345982\du)--(23.803649\du,18.065621\du);
}
\pgfsetlinewidth{0.050000\du}
\pgfsetdash{}{0pt}
\pgfsetdash{}{0pt}
\pgfsetbuttcap
{
\definecolor{dialinecolor}{rgb}{0.000000, 0.000000, 0.000000}
\pgfsetfillcolor{dialinecolor}
% was here!!!
\pgfsetarrowsend{stealth}
\definecolor{dialinecolor}{rgb}{0.000000, 0.000000, 0.000000}
\pgfsetstrokecolor{dialinecolor}
\draw (22.338296\du,17.277858\du)--(23.672434\du,18.186107\du);
}
\pgfsetlinewidth{0.050000\du}
\pgfsetdash{}{0pt}
\pgfsetdash{}{0pt}
\pgfsetbuttcap
{
\definecolor{dialinecolor}{rgb}{0.000000, 0.000000, 0.000000}
\pgfsetfillcolor{dialinecolor}
% was here!!!
\pgfsetarrowsend{stealth}
\definecolor{dialinecolor}{rgb}{0.000000, 0.000000, 0.000000}
\pgfsetstrokecolor{dialinecolor}
\draw (24.251687\du,13.947912\du)--(25.301622\du,15.137997\du);
}
\pgfsetlinewidth{0.050000\du}
\pgfsetdash{}{0pt}
\pgfsetdash{}{0pt}
\pgfsetbuttcap
{
\definecolor{dialinecolor}{rgb}{0.000000, 0.000000, 0.000000}
\pgfsetfillcolor{dialinecolor}
% was here!!!
\pgfsetarrowsend{stealth}
\definecolor{dialinecolor}{rgb}{0.000000, 0.000000, 0.000000}
\pgfsetstrokecolor{dialinecolor}
\draw (24.188957\du,14.012584\du)--(25.394405\du,16.224132\du);
}
\pgfsetlinewidth{0.050000\du}
\pgfsetdash{}{0pt}
\pgfsetdash{}{0pt}
\pgfsetbuttcap
{
\definecolor{dialinecolor}{rgb}{0.000000, 0.000000, 0.000000}
\pgfsetfillcolor{dialinecolor}
% was here!!!
\pgfsetarrowsend{stealth}
\definecolor{dialinecolor}{rgb}{0.000000, 0.000000, 0.000000}
\pgfsetstrokecolor{dialinecolor}
\draw (24.194848\du,18.053875\du)--(25.370836\du,15.789060\du);
}
\pgfsetlinewidth{0.050000\du}
\pgfsetdash{}{0pt}
\pgfsetdash{}{0pt}
\pgfsetbuttcap
{
\definecolor{dialinecolor}{rgb}{0.000000, 0.000000, 0.000000}
\pgfsetfillcolor{dialinecolor}
% was here!!!
\pgfsetarrowsend{stealth}
\definecolor{dialinecolor}{rgb}{0.000000, 0.000000, 0.000000}
\pgfsetstrokecolor{dialinecolor}
\draw (24.229061\du,18.075662\du)--(25.207110\du,16.579407\du);
}
\pgfsetlinewidth{0.050000\du}
\pgfsetdash{}{0pt}
\pgfsetdash{}{0pt}
\pgfsetbuttcap
{
\definecolor{dialinecolor}{rgb}{0.000000, 0.000000, 0.000000}
\pgfsetfillcolor{dialinecolor}
% was here!!!
\pgfsetarrowsend{stealth}
\definecolor{dialinecolor}{rgb}{0.000000, 0.000000, 0.000000}
\pgfsetstrokecolor{dialinecolor}
\draw (25.808484\du,15.109189\du)--(26.746472\du,13.913081\du);
}
\pgfsetlinewidth{0.050000\du}
\pgfsetdash{}{0pt}
\pgfsetdash{}{0pt}
\pgfsetbuttcap
{
\definecolor{dialinecolor}{rgb}{0.000000, 0.000000, 0.000000}
\pgfsetfillcolor{dialinecolor}
% was here!!!
\pgfsetarrowsend{stealth}
\definecolor{dialinecolor}{rgb}{0.000000, 0.000000, 0.000000}
\pgfsetstrokecolor{dialinecolor}
\draw (25.732361\du,15.791990\du)--(26.811989\du,18.042106\du);
}
\pgfsetlinewidth{0.050000\du}
\pgfsetdash{}{0pt}
\pgfsetdash{}{0pt}
\pgfsetbuttcap
{
\definecolor{dialinecolor}{rgb}{0.000000, 0.000000, 0.000000}
\pgfsetfillcolor{dialinecolor}
% was here!!!
\pgfsetarrowsend{stealth}
\definecolor{dialinecolor}{rgb}{0.000000, 0.000000, 0.000000}
\pgfsetstrokecolor{dialinecolor}
\draw (25.760039\du,16.214937\du)--(26.824969\du,13.958140\du);
}
\pgfsetlinewidth{0.050000\du}
\pgfsetdash{}{0pt}
\pgfsetdash{}{0pt}
\pgfsetbuttcap
{
\definecolor{dialinecolor}{rgb}{0.000000, 0.000000, 0.000000}
\pgfsetfillcolor{dialinecolor}
% was here!!!
\pgfsetarrowsend{stealth}
\definecolor{dialinecolor}{rgb}{0.000000, 0.000000, 0.000000}
\pgfsetstrokecolor{dialinecolor}
\draw (25.834530\du,16.901291\du)--(26.739871\du,18.083613\du);
}
\pgfsetlinewidth{0.050000\du}
\pgfsetdash{}{0pt}
\pgfsetdash{}{0pt}
\pgfsetbuttcap
{
\definecolor{dialinecolor}{rgb}{0.000000, 0.000000, 0.000000}
\pgfsetfillcolor{dialinecolor}
% was here!!!
\pgfsetarrowsend{stealth}
\definecolor{dialinecolor}{rgb}{0.000000, 0.000000, 0.000000}
\pgfsetstrokecolor{dialinecolor}
\draw (27.329850\du,13.824137\du)--(28.663387\du,14.747355\du);
}
\pgfsetlinewidth{0.050000\du}
\pgfsetdash{}{0pt}
\pgfsetdash{}{0pt}
\pgfsetbuttcap
{
\definecolor{dialinecolor}{rgb}{0.000000, 0.000000, 0.000000}
\pgfsetfillcolor{dialinecolor}
% was here!!!
\pgfsetarrowsend{stealth}
\definecolor{dialinecolor}{rgb}{0.000000, 0.000000, 0.000000}
\pgfsetstrokecolor{dialinecolor}
\draw (27.204460\du,13.942674\du)--(28.801152\du,16.628148\du);
}
\pgfsetlinewidth{0.050000\du}
\pgfsetdash{}{0pt}
\pgfsetdash{}{0pt}
\pgfsetbuttcap
{
\definecolor{dialinecolor}{rgb}{0.000000, 0.000000, 0.000000}
\pgfsetfillcolor{dialinecolor}
% was here!!!
\pgfsetarrowsend{stealth}
\definecolor{dialinecolor}{rgb}{0.000000, 0.000000, 0.000000}
\pgfsetstrokecolor{dialinecolor}
\draw (27.191708\du,18.055282\du)--(28.790923\du,15.328037\du);
}
\pgfsetlinewidth{0.050000\du}
\pgfsetdash{}{0pt}
\pgfsetdash{}{0pt}
\pgfsetbuttcap
{
\definecolor{dialinecolor}{rgb}{0.000000, 0.000000, 0.000000}
\pgfsetfillcolor{dialinecolor}
% was here!!!
\pgfsetarrowsend{stealth}
\definecolor{dialinecolor}{rgb}{0.000000, 0.000000, 0.000000}
\pgfsetstrokecolor{dialinecolor}
\draw (27.310233\du,18.176738\du)--(28.684772\du,17.205910\du);
}
\pgfsetlinewidth{0.050000\du}
\pgfsetdash{}{0pt}
\pgfsetdash{}{0pt}
\pgfsetbuttcap
{
\definecolor{dialinecolor}{rgb}{0.000000, 0.000000, 0.000000}
\pgfsetfillcolor{dialinecolor}
% was here!!!
\pgfsetarrowsend{stealth}
\definecolor{dialinecolor}{rgb}{0.000000, 0.000000, 0.000000}
\pgfsetstrokecolor{dialinecolor}
\draw (29.285398\du,15.262454\du)--(30.759532\du,16.713733\du);
}
\pgfsetlinewidth{0.050000\du}
\pgfsetdash{}{0pt}
\pgfsetdash{}{0pt}
\pgfsetbuttcap
{
\definecolor{dialinecolor}{rgb}{0.000000, 0.000000, 0.000000}
\pgfsetfillcolor{dialinecolor}
% was here!!!
\pgfsetarrowsend{stealth}
\definecolor{dialinecolor}{rgb}{0.000000, 0.000000, 0.000000}
\pgfsetstrokecolor{dialinecolor}
\draw (29.298741\du,16.694150\du)--(30.746188\du,15.282037\du);
}
\pgfsetlinewidth{0.050000\du}
\pgfsetdash{}{0pt}
\pgfsetdash{}{0pt}
\pgfsetbuttcap
{
\definecolor{dialinecolor}{rgb}{0.000000, 0.000000, 0.000000}
\pgfsetfillcolor{dialinecolor}
% was here!!!
\pgfsetarrowsend{stealth}
\definecolor{dialinecolor}{rgb}{0.000000, 0.000000, 0.000000}
\pgfsetstrokecolor{dialinecolor}
\draw (29.398703\du,14.982006\du)--(30.633852\du,14.994850\du);
}
\pgfsetlinewidth{0.050000\du}
\pgfsetdash{}{0pt}
\pgfsetdash{}{0pt}
\pgfsetbuttcap
{
\definecolor{dialinecolor}{rgb}{0.000000, 0.000000, 0.000000}
\pgfsetfillcolor{dialinecolor}
% was here!!!
\pgfsetarrowsend{stealth}
\definecolor{dialinecolor}{rgb}{0.000000, 0.000000, 0.000000}
\pgfsetstrokecolor{dialinecolor}
\draw (29.411077\du,16.981336\du)--(30.646227\du,16.994180\du);
}
\pgfsetlinewidth{0.050000\du}
\pgfsetdash{}{0pt}
\pgfsetdash{}{0pt}
\pgfsetbuttcap
{
\definecolor{dialinecolor}{rgb}{0.000000, 0.000000, 0.000000}
\pgfsetfillcolor{dialinecolor}
% was here!!!
\pgfsetarrowsend{stealth}
\definecolor{dialinecolor}{rgb}{0.000000, 0.000000, 0.000000}
\pgfsetstrokecolor{dialinecolor}
\draw (31.317664\du,14.706696\du)--(32.299897\du,13.686253\du);
}
\pgfsetlinewidth{0.050000\du}
\pgfsetdash{}{0pt}
\pgfsetdash{}{0pt}
\pgfsetbuttcap
{
\definecolor{dialinecolor}{rgb}{0.000000, 0.000000, 0.000000}
\pgfsetfillcolor{dialinecolor}
% was here!!!
\pgfsetarrowsend{stealth}
\definecolor{dialinecolor}{rgb}{0.000000, 0.000000, 0.000000}
\pgfsetstrokecolor{dialinecolor}
\draw (31.207486\du,16.624808\du)--(32.422449\du,13.767471\du);
}
\pgfsetlinewidth{0.050000\du}
\pgfsetdash{}{0pt}
\pgfsetdash{}{0pt}
\pgfsetbuttcap
{
\definecolor{dialinecolor}{rgb}{0.000000, 0.000000, 0.000000}
\pgfsetfillcolor{dialinecolor}
% was here!!!
\pgfsetarrowsend{stealth}
\definecolor{dialinecolor}{rgb}{0.000000, 0.000000, 0.000000}
\pgfsetstrokecolor{dialinecolor}
\draw (31.332412\du,17.288042\du)--(32.292220\du,18.267846\du);
}
\pgfsetlinewidth{0.050000\du}
\pgfsetdash{}{0pt}
\pgfsetdash{}{0pt}
\pgfsetbuttcap
{
\definecolor{dialinecolor}{rgb}{0.000000, 0.000000, 0.000000}
\pgfsetfillcolor{dialinecolor}
% was here!!!
\pgfsetarrowsend{stealth}
\definecolor{dialinecolor}{rgb}{0.000000, 0.000000, 0.000000}
\pgfsetstrokecolor{dialinecolor}
\draw (31.197348\du,15.371304\du)--(32.414910\du,18.185255\du);
}
\pgfsetlinewidth{0.040000\du}
\pgfsetdash{}{0pt}
\pgfsetdash{}{0pt}
\pgfsetmiterjoin
\pgfsetbuttcap
{
\definecolor{dialinecolor}{rgb}{0.000000, 0.000000, 0.000000}
\pgfsetfillcolor{dialinecolor}
% was here!!!
{\pgfsetcornersarced{\pgfpoint{0.000000\du}{0.000000\du}}\definecolor{dialinecolor}{rgb}{0.000000, 0.000000, 0.000000}
\pgfsetstrokecolor{dialinecolor}
\draw (19.973756\du,12.976645\du)--(19.957614\du,12.976645\du)--(19.957614\du,11.959664\du)--(24.396604\du,11.959664\du)--(24.396604\du,13.001147\du);
}}
\pgfsetlinewidth{0.040000\du}
\pgfsetdash{}{0pt}
\pgfsetdash{}{0pt}
\pgfsetmiterjoin
\pgfsetbuttcap
{
\definecolor{dialinecolor}{rgb}{0.000000, 0.000000, 0.000000}
\pgfsetfillcolor{dialinecolor}
% was here!!!
{\pgfsetcornersarced{\pgfpoint{0.000000\du}{0.000000\du}}\definecolor{dialinecolor}{rgb}{0.000000, 0.000000, 0.000000}
\pgfsetstrokecolor{dialinecolor}
\draw (26.811438\du,12.988699\du)--(26.798990\du,12.988699\du)--(26.798990\du,11.965136\du)--(31.288697\du,11.965136\du)--(31.288697\du,13.046687\du);
}}
\pgfsetlinewidth{0.100000\du}
\pgfsetdash{}{0pt}
\pgfsetdash{}{0pt}
\pgfsetbuttcap
\pgfsetmiterjoin
\pgfsetlinewidth{0.100000\du}
\pgfsetbuttcap
\pgfsetmiterjoin
\pgfsetdash{}{0pt}
\definecolor{dialinecolor}{rgb}{1.000000, 1.000000, 1.000000}
\pgfsetstrokecolor{dialinecolor}
\pgfpathmoveto{\pgfpoint{20.863680\du}{12.072662\du}}
\pgfpathlineto{\pgfpoint{23.296180\du}{12.072662\du}}
\pgfpathcurveto{\pgfpoint{23.632039\du}{12.072662\du}}{\pgfpoint{23.904305\du}{12.223372\du}}{\pgfpoint{23.904305\du}{12.409282\du}}
\pgfpathcurveto{\pgfpoint{23.904305\du}{12.595191\du}}{\pgfpoint{23.632039\du}{12.745901\du}}{\pgfpoint{23.296180\du}{12.745901\du}}
\pgfpathlineto{\pgfpoint{20.863680\du}{12.745901\du}}
\pgfpathcurveto{\pgfpoint{20.527822\du}{12.745901\du}}{\pgfpoint{20.255555\du}{12.595191\du}}{\pgfpoint{20.255555\du}{12.409282\du}}
\pgfpathcurveto{\pgfpoint{20.255555\du}{12.223372\du}}{\pgfpoint{20.527822\du}{12.072662\du}}{\pgfpoint{20.863680\du}{12.072662\du}}
\pgfusepath{stroke}
% setfont left to latex
\definecolor{dialinecolor}{rgb}{0.000000, 0.000000, 0.000000}
\pgfsetstrokecolor{dialinecolor}
\node at (22.079930\du,12.502591\du){Hidden Layers};
\pgfsetlinewidth{0.100000\du}
\pgfsetdash{}{0pt}
\pgfsetdash{}{0pt}
\pgfsetbuttcap
\pgfsetmiterjoin
\pgfsetlinewidth{0.100000\du}
\pgfsetbuttcap
\pgfsetmiterjoin
\pgfsetdash{}{0pt}
\definecolor{dialinecolor}{rgb}{1.000000, 1.000000, 1.000000}
\pgfsetstrokecolor{dialinecolor}
\pgfpathmoveto{\pgfpoint{27.794330\du}{12.091991\du}}
\pgfpathlineto{\pgfpoint{30.226830\du}{12.091991\du}}
\pgfpathcurveto{\pgfpoint{30.562688\du}{12.091991\du}}{\pgfpoint{30.834955\du}{12.242701\du}}{\pgfpoint{30.834955\du}{12.428611\du}}
\pgfpathcurveto{\pgfpoint{30.834955\du}{12.614521\du}}{\pgfpoint{30.562688\du}{12.765230\du}}{\pgfpoint{30.226830\du}{12.765230\du}}
\pgfpathlineto{\pgfpoint{27.794330\du}{12.765230\du}}
\pgfpathcurveto{\pgfpoint{27.458471\du}{12.765230\du}}{\pgfpoint{27.186205\du}{12.614521\du}}{\pgfpoint{27.186205\du}{12.428611\du}}
\pgfpathcurveto{\pgfpoint{27.186205\du}{12.242701\du}}{\pgfpoint{27.458471\du}{12.091991\du}}{\pgfpoint{27.794330\du}{12.091991\du}}
\pgfusepath{stroke}
% setfont left to latex
\definecolor{dialinecolor}{rgb}{0.000000, 0.000000, 0.000000}
\pgfsetstrokecolor{dialinecolor}
\node at (29.010580\du,12.521920\du){Hidden Layers};
\pgfsetlinewidth{0.040000\du}
\pgfsetdash{}{0pt}
\pgfsetdash{}{0pt}
\pgfsetmiterjoin
\pgfsetbuttcap
{
\definecolor{dialinecolor}{rgb}{0.000000, 0.000000, 0.000000}
\pgfsetfillcolor{dialinecolor}
% was here!!!
{\pgfsetcornersarced{\pgfpoint{0.000000\du}{0.000000\du}}\definecolor{dialinecolor}{rgb}{0.000000, 0.000000, 0.000000}
\pgfsetstrokecolor{dialinecolor}
\draw (24.595765\du,13.001147\du)--(24.608213\du,13.001147\du)--(24.608213\du,11.932851\du)--(26.587381\du,11.932851\du)--(26.587381\du,13.001147\du);
}}
\pgfsetlinewidth{0.000000\du}
\pgfsetdash{{1.000000\du}{1.000000\du}}{0\du}
\pgfsetdash{{1.000000\du}{1.000000\du}}{0\du}
\pgfsetbuttcap
\pgfsetmiterjoin
\pgfsetlinewidth{0.000000\du}
\pgfsetbuttcap
\pgfsetmiterjoin
\pgfsetdash{{1.000000\du}{1.000000\du}}{0\du}
\definecolor{dialinecolor}{rgb}{1.000000, 1.000000, 1.000000}
\pgfsetstrokecolor{dialinecolor}
\pgfpathmoveto{\pgfpoint{24.912450\du}{12.006888\du}}
\pgfpathlineto{\pgfpoint{26.392450\du}{12.006888\du}}
\pgfpathcurveto{\pgfpoint{26.596796\du}{12.006888\du}}{\pgfpoint{26.762450\du}{12.241150\du}}{\pgfpoint{26.762450\du}{12.530127\du}}
\pgfpathcurveto{\pgfpoint{26.762450\du}{12.819104\du}}{\pgfpoint{26.596796\du}{13.053366\du}}{\pgfpoint{26.392450\du}{13.053366\du}}
\pgfpathlineto{\pgfpoint{24.912450\du}{13.053366\du}}
\pgfpathcurveto{\pgfpoint{24.708105\du}{13.053366\du}}{\pgfpoint{24.542450\du}{12.819104\du}}{\pgfpoint{24.542450\du}{12.530127\du}}
\pgfpathcurveto{\pgfpoint{24.542450\du}{12.241150\du}}{\pgfpoint{24.708105\du}{12.006888\du}}{\pgfpoint{24.912450\du}{12.006888\du}}
\pgfusepath{stroke}
% setfont left to latex
\definecolor{dialinecolor}{rgb}{0.000000, 0.000000, 0.000000}
\pgfsetstrokecolor{dialinecolor}
\node at (25.652450\du,12.465039\du){Embedding};
% setfont left to latex
\definecolor{dialinecolor}{rgb}{0.000000, 0.000000, 0.000000}
\pgfsetstrokecolor{dialinecolor}
\node at (25.652450\du,12.725389\du){Layer};
\pgfsetlinewidth{0.100000\du}
\pgfsetdash{}{0pt}
\pgfsetdash{}{0pt}
\pgfsetbuttcap
\pgfsetmiterjoin
\pgfsetlinewidth{0.100000\du}
\pgfsetbuttcap
\pgfsetmiterjoin
\pgfsetdash{}{0pt}
\definecolor{dialinecolor}{rgb}{0.000000, 0.000000, 0.000000}
\pgfsetfillcolor{dialinecolor}
\fill (17.102312\du,13.262546\du)--(17.544202\du,13.262546\du)--(17.544202\du,13.122511\du)--(17.986092\du,13.402582\du)--(17.544202\du,13.682653\du)--(17.544202\du,13.542617\du)--(17.102312\du,13.542617\du)--cycle;
\definecolor{dialinecolor}{rgb}{0.000000, 0.000000, 0.000000}
\pgfsetstrokecolor{dialinecolor}
\draw (17.102312\du,13.262546\du)--(17.544202\du,13.262546\du)--(17.544202\du,13.122511\du)--(17.986092\du,13.402582\du)--(17.544202\du,13.682653\du)--(17.544202\du,13.542617\du)--(17.102312\du,13.542617\du)--cycle;
\pgfsetbuttcap
\pgfsetmiterjoin
\pgfsetdash{}{0pt}
\definecolor{dialinecolor}{rgb}{0.000000, 0.000000, 0.000000}
\pgfsetstrokecolor{dialinecolor}
\draw (17.102312\du,13.262546\du)--(17.544202\du,13.262546\du)--(17.544202\du,13.122511\du)--(17.986092\du,13.402582\du)--(17.544202\du,13.682653\du)--(17.544202\du,13.542617\du)--(17.102312\du,13.542617\du)--cycle;
\pgfsetlinewidth{0.100000\du}
\pgfsetdash{}{0pt}
\pgfsetdash{}{0pt}
\pgfsetbuttcap
\pgfsetmiterjoin
\pgfsetlinewidth{0.100000\du}
\pgfsetbuttcap
\pgfsetmiterjoin
\pgfsetdash{}{0pt}
\definecolor{dialinecolor}{rgb}{0.000000, 0.000000, 0.000000}
\pgfsetfillcolor{dialinecolor}
\fill (17.098788\du,14.338038\du)--(17.540677\du,14.338038\du)--(17.540677\du,14.198003\du)--(17.982567\du,14.478074\du)--(17.540677\du,14.758144\du)--(17.540677\du,14.618109\du)--(17.098788\du,14.618109\du)--cycle;
\definecolor{dialinecolor}{rgb}{0.000000, 0.000000, 0.000000}
\pgfsetstrokecolor{dialinecolor}
\draw (17.098788\du,14.338038\du)--(17.540677\du,14.338038\du)--(17.540677\du,14.198003\du)--(17.982567\du,14.478074\du)--(17.540677\du,14.758144\du)--(17.540677\du,14.618109\du)--(17.098788\du,14.618109\du)--cycle;
\pgfsetbuttcap
\pgfsetmiterjoin
\pgfsetdash{}{0pt}
\definecolor{dialinecolor}{rgb}{0.000000, 0.000000, 0.000000}
\pgfsetstrokecolor{dialinecolor}
\draw (17.098788\du,14.338038\du)--(17.540677\du,14.338038\du)--(17.540677\du,14.198003\du)--(17.982567\du,14.478074\du)--(17.540677\du,14.758144\du)--(17.540677\du,14.618109\du)--(17.098788\du,14.618109\du)--cycle;
\pgfsetlinewidth{0.100000\du}
\pgfsetdash{}{0pt}
\pgfsetdash{}{0pt}
\pgfsetbuttcap
\pgfsetmiterjoin
\pgfsetlinewidth{0.100000\du}
\pgfsetbuttcap
\pgfsetmiterjoin
\pgfsetdash{}{0pt}
\definecolor{dialinecolor}{rgb}{0.000000, 0.000000, 0.000000}
\pgfsetfillcolor{dialinecolor}
\fill (17.082606\du,18.459437\du)--(17.524496\du,18.459437\du)--(17.524496\du,18.319402\du)--(17.966385\du,18.599473\du)--(17.524496\du,18.879544\du)--(17.524496\du,18.739508\du)--(17.082606\du,18.739508\du)--cycle;
\definecolor{dialinecolor}{rgb}{0.000000, 0.000000, 0.000000}
\pgfsetstrokecolor{dialinecolor}
\draw (17.082606\du,18.459437\du)--(17.524496\du,18.459437\du)--(17.524496\du,18.319402\du)--(17.966385\du,18.599473\du)--(17.524496\du,18.879544\du)--(17.524496\du,18.739508\du)--(17.082606\du,18.739508\du)--cycle;
\pgfsetbuttcap
\pgfsetmiterjoin
\pgfsetdash{}{0pt}
\definecolor{dialinecolor}{rgb}{0.000000, 0.000000, 0.000000}
\pgfsetstrokecolor{dialinecolor}
\draw (17.082606\du,18.459437\du)--(17.524496\du,18.459437\du)--(17.524496\du,18.319402\du)--(17.966385\du,18.599473\du)--(17.524496\du,18.879544\du)--(17.524496\du,18.739508\du)--(17.082606\du,18.739508\du)--cycle;
\pgfsetlinewidth{0.100000\du}
\pgfsetdash{}{0pt}
\pgfsetdash{}{0pt}
\pgfsetbuttcap
\pgfsetmiterjoin
\pgfsetlinewidth{0.100000\du}
\pgfsetbuttcap
\pgfsetmiterjoin
\pgfsetdash{}{0pt}
\definecolor{dialinecolor}{rgb}{0.000000, 0.000000, 0.000000}
\pgfsetfillcolor{dialinecolor}
\fill (33.061585\du,13.232692\du)--(33.503475\du,13.232692\du)--(33.503475\du,13.092656\du)--(33.945365\du,13.372727\du)--(33.503475\du,13.652798\du)--(33.503475\du,13.512762\du)--(33.061585\du,13.512762\du)--cycle;
\definecolor{dialinecolor}{rgb}{0.000000, 0.000000, 0.000000}
\pgfsetstrokecolor{dialinecolor}
\draw (33.061585\du,13.232692\du)--(33.503475\du,13.232692\du)--(33.503475\du,13.092656\du)--(33.945365\du,13.372727\du)--(33.503475\du,13.652798\du)--(33.503475\du,13.512762\du)--(33.061585\du,13.512762\du)--cycle;
\pgfsetbuttcap
\pgfsetmiterjoin
\pgfsetdash{}{0pt}
\definecolor{dialinecolor}{rgb}{0.000000, 0.000000, 0.000000}
\pgfsetstrokecolor{dialinecolor}
\draw (33.061585\du,13.232692\du)--(33.503475\du,13.232692\du)--(33.503475\du,13.092656\du)--(33.945365\du,13.372727\du)--(33.503475\du,13.652798\du)--(33.503475\du,13.512762\du)--(33.061585\du,13.512762\du)--cycle;
\pgfsetlinewidth{0.100000\du}
\pgfsetdash{}{0pt}
\pgfsetdash{}{0pt}
\pgfsetbuttcap
\pgfsetmiterjoin
\pgfsetlinewidth{0.100000\du}
\pgfsetbuttcap
\pgfsetmiterjoin
\pgfsetdash{}{0pt}
\definecolor{dialinecolor}{rgb}{0.000000, 0.000000, 0.000000}
\pgfsetfillcolor{dialinecolor}
\fill (33.045404\du,18.399689\du)--(33.487293\du,18.399689\du)--(33.487293\du,18.259653\du)--(33.929183\du,18.539724\du)--(33.487293\du,18.819795\du)--(33.487293\du,18.679760\du)--(33.045404\du,18.679760\du)--cycle;
\definecolor{dialinecolor}{rgb}{0.000000, 0.000000, 0.000000}
\pgfsetstrokecolor{dialinecolor}
\draw (33.045404\du,18.399689\du)--(33.487293\du,18.399689\du)--(33.487293\du,18.259653\du)--(33.929183\du,18.539724\du)--(33.487293\du,18.819795\du)--(33.487293\du,18.679760\du)--(33.045404\du,18.679760\du)--cycle;
\pgfsetbuttcap
\pgfsetmiterjoin
\pgfsetdash{}{0pt}
\definecolor{dialinecolor}{rgb}{0.000000, 0.000000, 0.000000}
\pgfsetstrokecolor{dialinecolor}
\draw (33.045404\du,18.399689\du)--(33.487293\du,18.399689\du)--(33.487293\du,18.259653\du)--(33.929183\du,18.539724\du)--(33.487293\du,18.819795\du)--(33.487293\du,18.679760\du)--(33.045404\du,18.679760\du)--cycle;
\pgfsetlinewidth{0.100000\du}
\pgfsetdash{}{0pt}
\pgfsetdash{}{0pt}
\pgfsetmiterjoin
\definecolor{dialinecolor}{rgb}{0.000000, 0.000000, 0.000000}
\pgfsetstrokecolor{dialinecolor}
\draw (16.658208\du,11.772819\du)--(16.658208\du,19.489021\du)--(34.670289\du,19.489021\du)--(34.670289\du,11.772819\du)--cycle;
% setfont left to latex
\definecolor{dialinecolor}{rgb}{0.000000, 0.000000, 0.000000}
\pgfsetstrokecolor{dialinecolor}
\node at (25.664249\du,15.825920\du){};
\end{tikzpicture}

  \caption{\label{fig:archi}Architecture}
\end{figure}
 
\subsubsection{Hyperparameters}
For the $\lambda$ hyerparameter we use the value defined in \cite{Deap-K-Means}. 
For the different hyerparamters $\alpha_i$, we find them values experimently.

\subsubsection{Clustering without Noise}
We divide the dataset into two corpus $C_1, C_2$. Each corpus contain
ten classes. We generate keywords \ref{algo:gen_kw} and pairwise
constraints \ref{algo:gen_pair} from $C_1$. Then we add noise to the corpus
$C_1$. To add noise, we concatenate document from corpus $C_1$ with document
from corpus $C_2$.
\\The purpose of the experiment is to rediscover the different classes of the
corpus $C_1$ with keywords and background knowledge.

\subsubsection{Clustering with Noise}
We divide the dataset into two corpus $C_1, C_2$. Each corpus contain
ten classes. We generate keywords \ref{algo:gen_kw} and pairwise
constraints \ref{algo:gen_pair} from $C_1$. Then we add noise to the corpus
$C_1$. To add noise, we concatenate document from corpus $C_1$ with document
from corpus $C_2$.
\\The purpose of the experiment is to rediscover the different classes of the
corpus $C_1$ with keywords and background knowledge.

\subsubsection{Results}
\begin{figure}[!h]
  \centering
  \begin{tabular}{| l | l | l | l | l | l | l | l | l | l | l |}
    \hline
    & \multicolumn{3}{|c|}{Without noise} & \multicolumn{3}{|c|}{With noise} & \multicolumn{4}{|c|}{Hyperparameters}  \\
    & ACC &ARI & NMI & ACC & ARI &NMI & $\lambda$ & $\alpha_0$ & $\alpha_1$ & $\alpha_2$\\ \hline
    Constrained Deep K-Means & 0.501& 0.353 & 0.507 & 0.404& 0.19 & 0.287 & $10^{-4}$ & $10^{4}$ & 0 & 0 \\ \hline
    Deep K-Means& 0.505 & 0.361 & 0.512 & 0.381 & 0.183 & 0.290 &   $10^{-4}$ & 0 & 0 & 0\\ \hline
  \end{tabular}
\caption{\label{tab1}Clustering with only lexical constraints}
\end{figure}