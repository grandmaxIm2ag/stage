\section{Proposed Method}
The idea is to learn a latent space taking into account lexical constraints and
background knowledge.
\\We denote $KW = \begin{pmatrix} kw_1 & kw_2 & ... & kw_{k-1} & kw_{k}
\end {pmatrix}$
the set of index of keywords. X a document of size n,
X' the vector showing only informations about keywords such that :
\begin{equation*}
\forall_{i=1,2,..,n}X_i' = \left\{
\begin{array}{ll}
  X_i & \mbox{if } i \in KW \\
  0 & \mbox{Otherwise.}
\end{array}
\right.
\end{equation*}
C the corpus, N the size of C, and K the number of cluster.
\subsection{Auto-Encoder}
We denote $h_X$ the encoder output : 
\begin{equation}\label{eq:h}
  h_X = g(X,\theta)
\end{equation}
\subsubsection{Lexical Constraints}
To learn the latent space we a use sparse penalty such that the representation
of the document X is close of the representation of X'. In this way, in the
latent space, the representation of document will be biased by keywords.
We introduce a sparse penalty $\omega_{KW}$ for lexical constraints and
use squared loss : 
\begin{equation}\label{eq:omega1}
  \omega_{KW} = \sum_{\forall{X\in C}} || h_X - g(X',\theta) ||_2^2 =
  \sum_{\forall{X\in C}} || h_X - h_{X'}||_2^2
\end{equation}
\subsubsection{Backgroung Knowledge}
In addition of lexical constraints we introduce background knowledge, so we need
a representation taking into account these constraints.\\
For must-link constraints we want close representations for each pair $X_1$, $X_2$
in the set ML. We introduce a sparse penalty $\omega_{ML}$ for must-link
constraints and use squared loss :
\begin{equation}\label{eq:omegaML}
  \omega_{ML} = \sum_{\forall{(X_1,X_2)\in ML}} || h_{X_1} - h_{X_2} ||_2^2
\end{equation}
For cannot-link constraints we want distant representations for each pair $X_1$,
$X_2$ in the set CL.
We can use the hinge loss.
We introduce a sparse penalty $\omega_{CL}$ for cannot-link constraints.
\begin{equation}\label{eq:omegaCL}
  \omega_{CL} = \sum_{\forall{(X_1,X_2)\in CL}} max(0,
  \eta - || h_{X_1} - h_{X_2} ||_2^2)
\end{equation}
The sparse penalty is :
\begin{equation}\label{eq:Sparse}
  \Omega(C) = \alpha_0\omega_{KW} + \alpha_1\omega_{CL} + \alpha_2\omega_{ML}  
\end{equation}
With hyperparameters $\alpha_0\geq 0$ ,$\alpha_1\geq 0$, $\alpha_2\geq 0$. We use
hyperparameters because we can imagine there exists document $Y$ such that
$\forall_{i \in KW} Y_i = 0$. With this configuration, Y' is the null vector. If the
number of null vector Y' is high, the representation will be useless.
Hyperparameters allows to change the importance of each sparse penalty
and avoid problems with null vectors.\\
Then the reconstruct loss function is :
\begin{equation}\label{eq:AEDK}
  L_{rec}(C, \theta) = \sum_{X \in C}(||X - f(g(X, \theta))||_2^2
\end{equation}
Finally the loss function for the Auto-Encoder is :
\begin{equation}\label{eq:AE}
  L_{SAE}(C,KW; \theta) = L_{rec}(C, \theta) + \Omega(C)  
\end{equation}

\subsection{Deep K-Means}

\subsubsection{Clustering Loss}

For the clustering loss we can use squared loss with euclidian distance. We
denote $r = \begin{pmatrix} r_1 & r_2 & ... & r_K\end{pmatrix}$ the vector of
centroids, and S the set of clusters.  
\begin{equation}\label{eq:loss_clust}
  L_{clust}(C,K) = \sum_{i=1}^K \sum_{X \in S_i} ||h_X - r_i ||_2^2 
\end{equation}
Finally the loss function is :
\begin{equation}\label{eq:loss_FINALE}
  Min~L(KW, C, K; \theta) = \epsilon_0.L_{SAE}(C, KW; \theta) 
  + \epsilon_1.L_{clust}(C,K)
\end{equation}

with $\epsilon_0 \geq 0$, $\epsilon_1 \geq 0$.
