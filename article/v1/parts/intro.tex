\section{Introduction}\label{sec:intro}

Clustering is one of the most fundamental tasks in data mining and machine
learning. $K$-Means algorithm is a clustering method using centroid models,
it represents each cluster by a single mean vector. $K$-Means clustering sorts
n objects into k clusters in which each observation belongs to
the cluster with the nearest centroid. This problem is computationally
difficult (NP-hard).
\begin{algorithm}
  \SetKwInOut{Input}{input}
  \SetKwInOut{Output}{output}
  \Input{Corpus C, the number of cluster K}
  \Output{Clusters $S_1, S_2, ..., S_k$}
  Let $S_1^{0}, S_2^{0} , ..., S_k^{0}$ be the initial clusters\\
  Let $r_1^{0}, r_2^{0} , ..., r_k^{0}$ be the initial centroids\\
  $t \gets 1$\\
  \Repeat{Convergence}{
    \ForEach{$X \in C$}{
      $S_k^{t} \gets \{X : ||X-S_k^{t-1}||_2^2 \leq ||X-S_l^{t-1}||_2^2, 
\forall l \neq k, 1 \leq l \leq K \} $\\
    }
    \ForEach{Clusters $S_k$}{
      $r_k^t \gets \frac{1}{S_k^t}\sum_{X \in S_k^t}X$.\\
    }
    $t++$\\
  }
  \Return{$S_1, S_2, ..., S_k$}
  \caption{$K$-means}
\end{algorithm}
\\Given a corpus C, where each document X is a 
d-dimensional real vector, k-means clustering aims to partition the n 
documents into K clusters S = {$S_1, S_2, ..., S_K$}, represented by centroids 
r = {$r_1, r_2, ..., r_K$}. Formally, the objective is to minimize :
$$
\sum_{i=1}^K \sum_{X \in S_i} ||X - r_i ||_2^2
$$

In real application domains, users may want to introduce constaints to find
usefull properties for clustering data. The difficulty with integration of
constraints to $K$-Means algorithm  is to find a good representation for data
taking into account constraints. The Deep Learning and Auto-Encoder can be used to
learn this representation. With Auto-Encoder we have to perform the $K$-Means in
the latent space learned.
\\The major contribution to this work, is to propose a constrained Deep $K$-Means
taking account into ML and CL constraints and lexical constraints.
\\In next section, we provide some background on the $K$-Means algorithm and deep
learning. In section 3, we proposed a method to introduce constraints to the
$K$-Means algorithm. And we experiment our method in section 4.  
