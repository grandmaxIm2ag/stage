\section{Introduction}\label{sec:intro}

Clustering is one of the most fundamental tasks in data mining and machine
learning. $K$-Means algorithm is a clustering method using centroid models,
it represents each cluster by a single mean vector. $K$-Means clustering sorts
n objects into k clusters in which each observation belongs to
the cluster with the nearest centroid. This problem is computationally
difficult (NP-hard).
\begin{algorithm}
  \SetKwInOut{Input}{input}
  \SetKwInOut{Output}{output}
  \Input{Corpus C, the number of cluster K}
  \Output{Assignment matrix S}
  Let $r_1^{0}, r_2^{0} , ..., r_k^{0}$ be the initial centroids\\
  $t \gets 1$\\
  \Repeat{Convergence}{
    \For{i = 1 : N}{
      $
        S_{i,j}^t \gets \left\{
        \begin{array}{ll}
          1 & \mbox{if } j = \argmin_{k = \{1 ... K\}}||g(X_i) -
          r_k^{t-1}||_2^2\\
          0 & \mbox{Otherwise.}
        \end{array}
        \right.
      $
    }
    \ForEach{centroids $r_k$}{
      $r_k^t \gets \frac{1}{\sum_{i = 1}^N S_{i,k}}\sum_{i = 1}^N
      {X_i}S_{i,k}$.\\
    }
    $t++$\\
  }
  \Return{S}
  \caption{$K$-means}
\end{algorithm}
\\Given a corpus C, where each document X is a 
d-dimensional real vector, k-means clustering aims to partition the n 
documents into K clusters represented by centroids 
R = {$r_1, r_2, ..., r_K$}. S is the assignment matrix :
\begin{equation*}
  S_{i,j} = \left\{
\begin{array}{ll}
  1 & \mbox{if document i $\in$ cluster j}\\
  0 & \mbox{Otherwise.}
\end{array}
\right.
\end{equation*}
Formally, the objective is to minimize :
$$
\sum_{i =1 }^N ||X_i - RS_i ||_2^2
$$
In real application domains, users may want to introduce constraints to finding 
useful properties for clustering data. The difficulty with integration of 
constraints into $K$-Means algorithm is to find a good representation for data 
taking into account constraints. The Deep Learning and Auto-Encoder can be used 
to learn this representation. With Auto-Encoder we have to perform the $K$-Means 
in the latent space learned.
\\The major contribution to this work, is to propose a constrained Deep $K$-Means 
taking account into ML and CL constraints and lexical constraints.
\\In the next section, we provide some background on the $K$-Means algorithm and 
deep learning. In section 3, we proposed a method to introduce constraints to 
the $K$-Means algorithm. And we are experimenting our method in section 4.
