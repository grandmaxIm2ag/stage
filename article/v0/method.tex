\documentclass{article}
\usepackage[utf8]{inputenc}
\usepackage[english]{babel}
\usepackage[document]{ragged2e}
\usepackage{graphicx}
\usepackage{amsmath,amssymb,amsthm} 
\usepackage{url}
\usepackage{xspace}
\usepackage[left=20mm,top=20mm]{geometry}
\usepackage{subcaption}
\usepackage{mathpazo}
\usepackage{booktabs}
\usepackage{hyperref}
\usepackage{tikz}
\usepackage{dsfont}
\usepackage[]{algorithm2e}
\usepackage[
  style=numeric,
  natbib=true,
  sortcites=true,
  block=space]{biblatex} 
\bibliography{ressources/biblio/biblio}

\newcommand{\ie}{ie}
\newcommand{\eg}{eg}
\newcommand{\reffig}[1]{Figure~\ref{#1}}
\newcommand{\refsec}[1]{Section~\ref{#1}}


\title{Integrate lexical constraints to K-Means : ML \& CL Method}
\author{Grand Maxence}
\date{\today}

\begin{document}

\maketitle
\justify

\section{The proposed Method}

The idea is to produce must-link and cannot-link constraints using
TFIDF on key words and Auto-Encoder.\\ \\
First, we use a siamese architecture  Auto-Encoder to learn a latent
space where each coponents shows the TF-IDF of the Key Words without
taking account of all other words.\\Then, we can use the idea proposed
by Yen-Chang Hsu and Zsolt Kira \cite{2015arXiv151106321H}.
We use the softmax function for having a distribution over key works
and recognise the signficiance of each key word for all documents.
And then, we usethe KL Divergence to produce constraints.\\Finally,
we can use the COP-Kmeans to perform the K-Means with must-link and
cannot-link constraints \cite{Wagstaff:2001:CKC:645530.655669}.
The clustering will be biased by key words because the constraints
will only take into account key words.
\\ \\
We denote $KW = \begin{pmatrix} kw_1 & kw_2 & ... & kw_{k-1} & kw_{k}
\end {pmatrix}$
the set of key words. $W_i$ the word of index i.X a document including
only key word such that $\forall_{i = 1,2,..,n}~x_i = 1_{KW}(W_i)$.
Y the vector of occurence of each key words in document X such that
$\forall_{i=1, 2, .., k}~y_i = occur(kw_i~in~X)$. X' the representation of
the document X in the latent space. C the corpus of document and N the
size of C.

\section{TFIDF}

The term frequency-inversed document frequency (TFIDF) is a method of
weghting depicting the significiance of each word in a document rather
a corpus.
\\
To compute the Term Frequency (tf) of a term t in document c we use
the double normalisation K, with k=0.5 :  
\begin{equation}\label{eq:tf}
  tf(t, c) = 0.5 + 0.5.\frac{f_{t,c}}{max_{\{t' \in c \}}f_{t',c}}
\end{equation}

\begin{equation}\label{eq:idf}
  idf(t, C) = log(\frac{N}{| \{ c \in C : t \in c \}  |})
\end{equation}

\begin{equation}\label{eq:tfidf}
  tfidf(t,c,C) = tf(t,c) - idf(t,C)
\end{equation}


\section{Learn the latent space}

The auto-encoder allows to learn a latent space with significiantly
informations for the clustering without loss of information.
We use a sparse auto-encoder for learn the latent space of document.
\\
We denote $h=f(X, \theta_1)$ the encoder output, $g(f(X, \theta_1))$
the decoder output and $\omega(h)$ the sparsity penalty, $L_{sparse}$
the loss function of the sparse auto-encoder. we use MSE for
reconstruct loss.
\\
We can see the architecture in figure~\ref{fig:archi}.
\begin{equation}\label{eq:omega}
  \omega(h) = || h - Softmax(Y')||_2^2
\end{equation}

\begin{equation}\label{eq:Sparse}
  L_{SAE}(X, \theta_1) = ||X - f(g(X, \theta_1))||_2^2 + \omega(h)  
\end{equation}

\begin{figure}[!t]
  \centering
  \tikzset{every picture/.style={scale=1.5}}
  % Graphic for TeX using PGF
% Title: /media/maxence/SD_MAXENCE/cours/m1Informatique/S2/stage/article/v0/ressources/neural_network_autoencoder.dia
% Creator: Dia v0.97.3
% CreationDate: Wed Mar 28 15:41:42 2018
% For: maxence
% \usepackage{tikz}
% The following commands are not supported in PSTricks at present
% We define them conditionally, so when they are implemented,
% this pgf file will use them.
\ifx\du\undefined
  \newlength{\du}
\fi
\setlength{\du}{15\unitlength}
\begin{tikzpicture}
\pgftransformxscale{1.000000}
\pgftransformyscale{-1.000000}
\definecolor{dialinecolor}{rgb}{0.000000, 0.000000, 0.000000}
\pgfsetstrokecolor{dialinecolor}
\definecolor{dialinecolor}{rgb}{1.000000, 1.000000, 1.000000}
\pgfsetfillcolor{dialinecolor}
\pgfsetlinewidth{0.050000\du}
\pgfsetdash{}{0pt}
\pgfsetdash{}{0pt}
\pgfsetbuttcap
\pgfsetmiterjoin
\pgfsetlinewidth{0.050000\du}
\pgfsetbuttcap
\pgfsetmiterjoin
\pgfsetdash{}{0pt}
\definecolor{dialinecolor}{rgb}{1.000000, 1.000000, 1.000000}
\pgfsetfillcolor{dialinecolor}
\pgfpathellipse{\pgfpoint{0.017617\du}{-190.709859\du}}{\pgfpoint{0.975015\du}{0\du}}{\pgfpoint{0\du}{0.975015\du}}
\pgfusepath{fill}
\definecolor{dialinecolor}{rgb}{0.000000, 0.000000, 0.000000}
\pgfsetstrokecolor{dialinecolor}
\pgfpathellipse{\pgfpoint{0.017617\du}{-190.709859\du}}{\pgfpoint{0.975015\du}{0\du}}{\pgfpoint{0\du}{0.975015\du}}
\pgfusepath{stroke}
\pgfsetbuttcap
\pgfsetmiterjoin
\pgfsetdash{}{0pt}
\definecolor{dialinecolor}{rgb}{0.000000, 0.000000, 0.000000}
\pgfsetstrokecolor{dialinecolor}
\pgfpathellipse{\pgfpoint{0.017617\du}{-190.709859\du}}{\pgfpoint{0.975015\du}{0\du}}{\pgfpoint{0\du}{0.975015\du}}
\pgfusepath{stroke}
\pgfsetlinewidth{0.050000\du}
\pgfsetdash{}{0pt}
\pgfsetdash{}{0pt}
\pgfsetbuttcap
\pgfsetmiterjoin
\pgfsetlinewidth{0.050000\du}
\pgfsetbuttcap
\pgfsetmiterjoin
\pgfsetdash{}{0pt}
\definecolor{dialinecolor}{rgb}{1.000000, 1.000000, 1.000000}
\pgfsetfillcolor{dialinecolor}
\pgfpathellipse{\pgfpoint{-0.005712\du}{-185.477681\du}}{\pgfpoint{0.975015\du}{0\du}}{\pgfpoint{0\du}{0.975015\du}}
\pgfusepath{fill}
\definecolor{dialinecolor}{rgb}{0.000000, 0.000000, 0.000000}
\pgfsetstrokecolor{dialinecolor}
\pgfpathellipse{\pgfpoint{-0.005712\du}{-185.477681\du}}{\pgfpoint{0.975015\du}{0\du}}{\pgfpoint{0\du}{0.975015\du}}
\pgfusepath{stroke}
\pgfsetbuttcap
\pgfsetmiterjoin
\pgfsetdash{}{0pt}
\definecolor{dialinecolor}{rgb}{0.000000, 0.000000, 0.000000}
\pgfsetstrokecolor{dialinecolor}
\pgfpathellipse{\pgfpoint{-0.005712\du}{-185.477681\du}}{\pgfpoint{0.975015\du}{0\du}}{\pgfpoint{0\du}{0.975015\du}}
\pgfusepath{stroke}
\pgfsetlinewidth{0.080000\du}
\pgfsetdash{}{0pt}
\pgfsetdash{}{0pt}
\pgfsetbuttcap
{
\definecolor{dialinecolor}{rgb}{0.000000, 0.000000, 0.000000}
\pgfsetfillcolor{dialinecolor}
% was here!!!
\pgfsetarrowsend{stealth}
\definecolor{dialinecolor}{rgb}{0.000000, 0.000000, 0.000000}
\pgfsetstrokecolor{dialinecolor}
\draw (-2.894676\du,-190.634796\du)--(-0.981519\du,-190.684106\du);
}
\pgfsetlinewidth{0.050000\du}
\pgfsetdash{}{0pt}
\pgfsetdash{}{0pt}
\pgfsetbuttcap
\pgfsetmiterjoin
\pgfsetlinewidth{0.050000\du}
\pgfsetbuttcap
\pgfsetmiterjoin
\pgfsetdash{}{0pt}
\definecolor{dialinecolor}{rgb}{1.000000, 1.000000, 1.000000}
\pgfsetfillcolor{dialinecolor}
\pgfpathellipse{\pgfpoint{0.021277\du}{-193.493278\du}}{\pgfpoint{0.975015\du}{0\du}}{\pgfpoint{0\du}{0.975015\du}}
\pgfusepath{fill}
\definecolor{dialinecolor}{rgb}{0.000000, 0.000000, 0.000000}
\pgfsetstrokecolor{dialinecolor}
\pgfpathellipse{\pgfpoint{0.021277\du}{-193.493278\du}}{\pgfpoint{0.975015\du}{0\du}}{\pgfpoint{0\du}{0.975015\du}}
\pgfusepath{stroke}
\pgfsetbuttcap
\pgfsetmiterjoin
\pgfsetdash{}{0pt}
\definecolor{dialinecolor}{rgb}{0.000000, 0.000000, 0.000000}
\pgfsetstrokecolor{dialinecolor}
\pgfpathellipse{\pgfpoint{0.021277\du}{-193.493278\du}}{\pgfpoint{0.975015\du}{0\du}}{\pgfpoint{0\du}{0.975015\du}}
\pgfusepath{stroke}
\pgfsetlinewidth{0.050000\du}
\pgfsetdash{}{0pt}
\pgfsetdash{}{0pt}
\pgfsetbuttcap
\pgfsetmiterjoin
\pgfsetlinewidth{0.050000\du}
\pgfsetbuttcap
\pgfsetmiterjoin
\pgfsetdash{}{0pt}
\definecolor{dialinecolor}{rgb}{1.000000, 1.000000, 1.000000}
\pgfsetfillcolor{dialinecolor}
\pgfpathellipse{\pgfpoint{8.189937\du}{-189.828452\du}}{\pgfpoint{0.975015\du}{0\du}}{\pgfpoint{0\du}{0.975015\du}}
\pgfusepath{fill}
\definecolor{dialinecolor}{rgb}{0.000000, 0.000000, 0.000000}
\pgfsetstrokecolor{dialinecolor}
\pgfpathellipse{\pgfpoint{8.189937\du}{-189.828452\du}}{\pgfpoint{0.975015\du}{0\du}}{\pgfpoint{0\du}{0.975015\du}}
\pgfusepath{stroke}
\pgfsetbuttcap
\pgfsetmiterjoin
\pgfsetdash{}{0pt}
\definecolor{dialinecolor}{rgb}{0.000000, 0.000000, 0.000000}
\pgfsetstrokecolor{dialinecolor}
\pgfpathellipse{\pgfpoint{8.189937\du}{-189.828452\du}}{\pgfpoint{0.975015\du}{0\du}}{\pgfpoint{0\du}{0.975015\du}}
\pgfusepath{stroke}
\pgfsetlinewidth{0.050000\du}
\pgfsetdash{}{0pt}
\pgfsetdash{}{0pt}
\pgfsetbuttcap
\pgfsetmiterjoin
\pgfsetlinewidth{0.050000\du}
\pgfsetbuttcap
\pgfsetmiterjoin
\pgfsetdash{}{0pt}
\definecolor{dialinecolor}{rgb}{1.000000, 1.000000, 1.000000}
\pgfsetfillcolor{dialinecolor}
\pgfpathellipse{\pgfpoint{8.134822\du}{-186.557197\du}}{\pgfpoint{0.975015\du}{0\du}}{\pgfpoint{0\du}{0.975015\du}}
\pgfusepath{fill}
\definecolor{dialinecolor}{rgb}{0.000000, 0.000000, 0.000000}
\pgfsetstrokecolor{dialinecolor}
\pgfpathellipse{\pgfpoint{8.134822\du}{-186.557197\du}}{\pgfpoint{0.975015\du}{0\du}}{\pgfpoint{0\du}{0.975015\du}}
\pgfusepath{stroke}
\pgfsetbuttcap
\pgfsetmiterjoin
\pgfsetdash{}{0pt}
\definecolor{dialinecolor}{rgb}{0.000000, 0.000000, 0.000000}
\pgfsetstrokecolor{dialinecolor}
\pgfpathellipse{\pgfpoint{8.134822\du}{-186.557197\du}}{\pgfpoint{0.975015\du}{0\du}}{\pgfpoint{0\du}{0.975015\du}}
\pgfusepath{stroke}
\pgfsetlinewidth{0.080000\du}
\pgfsetdash{}{0pt}
\pgfsetdash{}{0pt}
\pgfsetbuttcap
{
\definecolor{dialinecolor}{rgb}{0.000000, 0.000000, 0.000000}
\pgfsetfillcolor{dialinecolor}
% was here!!!
\pgfsetarrowsend{stealth}
\definecolor{dialinecolor}{rgb}{0.000000, 0.000000, 0.000000}
\pgfsetstrokecolor{dialinecolor}
\draw (-3.019651\du,-193.464706\du)--(-0.978057\du,-193.483889\du);
}
\pgfsetlinewidth{0.080000\du}
\pgfsetdash{}{0pt}
\pgfsetdash{}{0pt}
\pgfsetbuttcap
{
\definecolor{dialinecolor}{rgb}{0.000000, 0.000000, 0.000000}
\pgfsetfillcolor{dialinecolor}
% was here!!!
\pgfsetarrowsend{stealth}
\definecolor{dialinecolor}{rgb}{0.000000, 0.000000, 0.000000}
\pgfsetstrokecolor{dialinecolor}
\draw (-2.992058\du,-185.459325\du)--(-1.005266\du,-185.471538\du);
}
\pgfsetlinewidth{0.080000\du}
\pgfsetdash{}{0pt}
\pgfsetdash{}{0pt}
\pgfsetbuttcap
{
\definecolor{dialinecolor}{rgb}{0.000000, 0.000000, 0.000000}
\pgfsetfillcolor{dialinecolor}
% was here!!!
\pgfsetarrowsend{stealth}
\definecolor{dialinecolor}{rgb}{0.000000, 0.000000, 0.000000}
\pgfsetstrokecolor{dialinecolor}
\draw (1.010269\du,-193.351253\du)--(7.191219\du,-192.463633\du);
}
\pgfsetlinewidth{0.080000\du}
\pgfsetdash{}{0pt}
\pgfsetdash{}{0pt}
\pgfsetbuttcap
{
\definecolor{dialinecolor}{rgb}{0.000000, 0.000000, 0.000000}
\pgfsetfillcolor{dialinecolor}
% was here!!!
\pgfsetarrowsend{stealth}
\definecolor{dialinecolor}{rgb}{0.000000, 0.000000, 0.000000}
\pgfsetstrokecolor{dialinecolor}
\draw (0.933919\du,-193.083826\du)--(7.277295\du,-190.237904\du);
}
\pgfsetlinewidth{0.080000\du}
\pgfsetdash{}{0pt}
\pgfsetdash{}{0pt}
\pgfsetbuttcap
{
\definecolor{dialinecolor}{rgb}{0.000000, 0.000000, 0.000000}
\pgfsetfillcolor{dialinecolor}
% was here!!!
\pgfsetarrowsend{stealth}
\definecolor{dialinecolor}{rgb}{0.000000, 0.000000, 0.000000}
\pgfsetstrokecolor{dialinecolor}
\draw (0.933919\du,-193.083826\du)--(7.277295\du,-190.237904\du);
}
\pgfsetlinewidth{0.080000\du}
\pgfsetdash{}{0pt}
\pgfsetdash{}{0pt}
\pgfsetbuttcap
{
\definecolor{dialinecolor}{rgb}{0.000000, 0.000000, 0.000000}
\pgfsetfillcolor{dialinecolor}
% was here!!!
\pgfsetarrowsend{stealth}
\definecolor{dialinecolor}{rgb}{0.000000, 0.000000, 0.000000}
\pgfsetstrokecolor{dialinecolor}
\draw (0.998583\du,-190.903556\du)--(7.199245\du,-192.127911\du);
}
\pgfsetlinewidth{0.080000\du}
\pgfsetdash{}{0pt}
\pgfsetdash{}{0pt}
\pgfsetbuttcap
{
\definecolor{dialinecolor}{rgb}{0.000000, 0.000000, 0.000000}
\pgfsetfillcolor{dialinecolor}
% was here!!!
\pgfsetarrowsend{stealth}
\definecolor{dialinecolor}{rgb}{0.000000, 0.000000, 0.000000}
\pgfsetstrokecolor{dialinecolor}
\draw (1.011723\du,-190.602642\du)--(7.195831\du,-189.935669\du);
}
\pgfsetlinewidth{0.080000\du}
\pgfsetdash{}{0pt}
\pgfsetdash{}{0pt}
\pgfsetbuttcap
{
\definecolor{dialinecolor}{rgb}{0.000000, 0.000000, 0.000000}
\pgfsetfillcolor{dialinecolor}
% was here!!!
\pgfsetarrowsend{stealth}
\definecolor{dialinecolor}{rgb}{0.000000, 0.000000, 0.000000}
\pgfsetstrokecolor{dialinecolor}
\draw (0.907914\du,-190.254394\du)--(7.244525\du,-187.012661\du);
}
\pgfsetlinewidth{0.080000\du}
\pgfsetdash{}{0pt}
\pgfsetdash{}{0pt}
\pgfsetbuttcap
{
\definecolor{dialinecolor}{rgb}{0.000000, 0.000000, 0.000000}
\pgfsetfillcolor{dialinecolor}
% was here!!!
\pgfsetarrowsend{stealth}
\definecolor{dialinecolor}{rgb}{0.000000, 0.000000, 0.000000}
\pgfsetstrokecolor{dialinecolor}
\draw (0.985024\du,-185.609063\du)--(7.144086\du,-186.425815\du);
}
\pgfsetlinewidth{0.080000\du}
\pgfsetdash{}{0pt}
\pgfsetdash{}{0pt}
\pgfsetbuttcap
{
\definecolor{dialinecolor}{rgb}{0.000000, 0.000000, 0.000000}
\pgfsetfillcolor{dialinecolor}
% was here!!!
\pgfsetarrowsend{stealth}
\definecolor{dialinecolor}{rgb}{0.000000, 0.000000, 0.000000}
\pgfsetstrokecolor{dialinecolor}
\draw (0.877181\du,-185.946377\du)--(7.307044\du,-189.359756\du);
}
\pgfsetlinewidth{0.080000\du}
\pgfsetdash{}{0pt}
\pgfsetdash{}{0pt}
\pgfsetbuttcap
{
\definecolor{dialinecolor}{rgb}{0.000000, 0.000000, 0.000000}
\pgfsetfillcolor{dialinecolor}
% was here!!!
\pgfsetarrowsend{stealth}
\definecolor{dialinecolor}{rgb}{0.000000, 0.000000, 0.000000}
\pgfsetstrokecolor{dialinecolor}
\draw (0.633813\du,-186.012363\du)--(7.540686\du,-191.786926\du);
}
\pgfsetlinewidth{0.050000\du}
\pgfsetdash{}{0pt}
\pgfsetdash{}{0pt}
\pgfsetbuttcap
\pgfsetmiterjoin
\pgfsetlinewidth{0.050000\du}
\pgfsetbuttcap
\pgfsetmiterjoin
\pgfsetdash{}{0pt}
\definecolor{dialinecolor}{rgb}{1.000000, 1.000000, 1.000000}
\pgfsetfillcolor{dialinecolor}
\pgfpathellipse{\pgfpoint{8.180211\du}{-192.321608\du}}{\pgfpoint{0.975015\du}{0\du}}{\pgfpoint{0\du}{0.975015\du}}
\pgfusepath{fill}
\definecolor{dialinecolor}{rgb}{0.000000, 0.000000, 0.000000}
\pgfsetstrokecolor{dialinecolor}
\pgfpathellipse{\pgfpoint{8.180211\du}{-192.321608\du}}{\pgfpoint{0.975015\du}{0\du}}{\pgfpoint{0\du}{0.975015\du}}
\pgfusepath{stroke}
\pgfsetbuttcap
\pgfsetmiterjoin
\pgfsetdash{}{0pt}
\definecolor{dialinecolor}{rgb}{0.000000, 0.000000, 0.000000}
\pgfsetstrokecolor{dialinecolor}
\pgfpathellipse{\pgfpoint{8.180211\du}{-192.321608\du}}{\pgfpoint{0.975015\du}{0\du}}{\pgfpoint{0\du}{0.975015\du}}
\pgfusepath{stroke}
\pgfsetlinewidth{0.050000\du}
\pgfsetdash{}{0pt}
\pgfsetdash{}{0pt}
\pgfsetbuttcap
\pgfsetmiterjoin
\pgfsetlinewidth{0.050000\du}
\pgfsetbuttcap
\pgfsetmiterjoin
\pgfsetdash{}{0pt}
\definecolor{dialinecolor}{rgb}{1.000000, 1.000000, 1.000000}
\pgfsetfillcolor{dialinecolor}
\pgfpathellipse{\pgfpoint{15.361418\du}{-193.498046\du}}{\pgfpoint{0.975015\du}{0\du}}{\pgfpoint{0\du}{0.975015\du}}
\pgfusepath{fill}
\definecolor{dialinecolor}{rgb}{0.000000, 0.000000, 0.000000}
\pgfsetstrokecolor{dialinecolor}
\pgfpathellipse{\pgfpoint{15.361418\du}{-193.498046\du}}{\pgfpoint{0.975015\du}{0\du}}{\pgfpoint{0\du}{0.975015\du}}
\pgfusepath{stroke}
\pgfsetbuttcap
\pgfsetmiterjoin
\pgfsetdash{}{0pt}
\definecolor{dialinecolor}{rgb}{0.000000, 0.000000, 0.000000}
\pgfsetstrokecolor{dialinecolor}
\pgfpathellipse{\pgfpoint{15.361418\du}{-193.498046\du}}{\pgfpoint{0.975015\du}{0\du}}{\pgfpoint{0\du}{0.975015\du}}
\pgfusepath{stroke}
\pgfsetlinewidth{0.050000\du}
\pgfsetdash{}{0pt}
\pgfsetdash{}{0pt}
\pgfsetbuttcap
\pgfsetmiterjoin
\pgfsetlinewidth{0.050000\du}
\pgfsetbuttcap
\pgfsetmiterjoin
\pgfsetdash{}{0pt}
\definecolor{dialinecolor}{rgb}{1.000000, 1.000000, 1.000000}
\pgfsetfillcolor{dialinecolor}
\pgfpathellipse{\pgfpoint{15.403565\du}{-190.680688\du}}{\pgfpoint{0.975015\du}{0\du}}{\pgfpoint{0\du}{0.975015\du}}
\pgfusepath{fill}
\definecolor{dialinecolor}{rgb}{0.000000, 0.000000, 0.000000}
\pgfsetstrokecolor{dialinecolor}
\pgfpathellipse{\pgfpoint{15.403565\du}{-190.680688\du}}{\pgfpoint{0.975015\du}{0\du}}{\pgfpoint{0\du}{0.975015\du}}
\pgfusepath{stroke}
\pgfsetbuttcap
\pgfsetmiterjoin
\pgfsetdash{}{0pt}
\definecolor{dialinecolor}{rgb}{0.000000, 0.000000, 0.000000}
\pgfsetstrokecolor{dialinecolor}
\pgfpathellipse{\pgfpoint{15.403565\du}{-190.680688\du}}{\pgfpoint{0.975015\du}{0\du}}{\pgfpoint{0\du}{0.975015\du}}
\pgfusepath{stroke}
\pgfsetlinewidth{0.050000\du}
\pgfsetdash{}{0pt}
\pgfsetdash{}{0pt}
\pgfsetbuttcap
\pgfsetmiterjoin
\pgfsetlinewidth{0.050000\du}
\pgfsetbuttcap
\pgfsetmiterjoin
\pgfsetdash{}{0pt}
\definecolor{dialinecolor}{rgb}{1.000000, 1.000000, 1.000000}
\pgfsetfillcolor{dialinecolor}
\pgfpathellipse{\pgfpoint{15.348450\du}{-185.366936\du}}{\pgfpoint{0.975015\du}{0\du}}{\pgfpoint{0\du}{0.975015\du}}
\pgfusepath{fill}
\definecolor{dialinecolor}{rgb}{0.000000, 0.000000, 0.000000}
\pgfsetstrokecolor{dialinecolor}
\pgfpathellipse{\pgfpoint{15.348450\du}{-185.366936\du}}{\pgfpoint{0.975015\du}{0\du}}{\pgfpoint{0\du}{0.975015\du}}
\pgfusepath{stroke}
\pgfsetbuttcap
\pgfsetmiterjoin
\pgfsetdash{}{0pt}
\definecolor{dialinecolor}{rgb}{0.000000, 0.000000, 0.000000}
\pgfsetstrokecolor{dialinecolor}
\pgfpathellipse{\pgfpoint{15.348450\du}{-185.366936\du}}{\pgfpoint{0.975015\du}{0\du}}{\pgfpoint{0\du}{0.975015\du}}
\pgfusepath{stroke}
\pgfsetlinewidth{0.080000\du}
\pgfsetdash{}{0pt}
\pgfsetdash{}{0pt}
\pgfsetbuttcap
{
\definecolor{dialinecolor}{rgb}{0.000000, 0.000000, 0.000000}
\pgfsetfillcolor{dialinecolor}
% was here!!!
\pgfsetarrowsend{stealth}
\definecolor{dialinecolor}{rgb}{0.000000, 0.000000, 0.000000}
\pgfsetstrokecolor{dialinecolor}
\draw (9.166400\du,-192.483167\du)--(14.375230\du,-193.336487\du);
}
\pgfsetlinewidth{0.080000\du}
\pgfsetdash{}{0pt}
\pgfsetdash{}{0pt}
\pgfsetbuttcap
{
\definecolor{dialinecolor}{rgb}{0.000000, 0.000000, 0.000000}
\pgfsetfillcolor{dialinecolor}
% was here!!!
\pgfsetarrowsend{stealth}
\definecolor{dialinecolor}{rgb}{0.000000, 0.000000, 0.000000}
\pgfsetstrokecolor{dialinecolor}
\draw (9.079807\du,-190.283791\du)--(14.471549\du,-193.042706\du);
}
\pgfsetlinewidth{0.080000\du}
\pgfsetdash{}{0pt}
\pgfsetdash{}{0pt}
\pgfsetbuttcap
{
\definecolor{dialinecolor}{rgb}{0.000000, 0.000000, 0.000000}
\pgfsetfillcolor{dialinecolor}
% was here!!!
\pgfsetarrowsend{stealth}
\definecolor{dialinecolor}{rgb}{0.000000, 0.000000, 0.000000}
\pgfsetstrokecolor{dialinecolor}
\draw (8.855541\du,-187.249418\du)--(14.640699\du,-192.805825\du);
}
\pgfsetlinewidth{0.080000\du}
\pgfsetdash{}{0pt}
\pgfsetdash{}{0pt}
\pgfsetbuttcap
{
\definecolor{dialinecolor}{rgb}{0.000000, 0.000000, 0.000000}
\pgfsetfillcolor{dialinecolor}
% was here!!!
\pgfsetarrowsend{stealth}
\definecolor{dialinecolor}{rgb}{0.000000, 0.000000, 0.000000}
\pgfsetstrokecolor{dialinecolor}
\draw (9.154553\du,-192.100268\du)--(14.429224\du,-190.902028\du);
}
\pgfsetlinewidth{0.080000\du}
\pgfsetdash{}{0pt}
\pgfsetdash{}{0pt}
\pgfsetbuttcap
{
\definecolor{dialinecolor}{rgb}{0.000000, 0.000000, 0.000000}
\pgfsetfillcolor{dialinecolor}
% was here!!!
\pgfsetarrowsend{stealth}
\definecolor{dialinecolor}{rgb}{0.000000, 0.000000, 0.000000}
\pgfsetstrokecolor{dialinecolor}
\draw (9.179698\du,-189.945384\du)--(14.413805\du,-190.563755\du);
}
\pgfsetlinewidth{0.080000\du}
\pgfsetdash{}{0pt}
\pgfsetdash{}{0pt}
\pgfsetbuttcap
{
\definecolor{dialinecolor}{rgb}{0.000000, 0.000000, 0.000000}
\pgfsetfillcolor{dialinecolor}
% was here!!!
\pgfsetarrowsend{stealth}
\definecolor{dialinecolor}{rgb}{0.000000, 0.000000, 0.000000}
\pgfsetstrokecolor{dialinecolor}
\draw (9.004817\du,-187.050737\du)--(14.533570\du,-190.187147\du);
}
\pgfsetlinewidth{0.080000\du}
\pgfsetdash{}{0pt}
\pgfsetdash{}{0pt}
\pgfsetbuttcap
{
\definecolor{dialinecolor}{rgb}{0.000000, 0.000000, 0.000000}
\pgfsetfillcolor{dialinecolor}
% was here!!!
\pgfsetarrowsend{stealth}
\definecolor{dialinecolor}{rgb}{0.000000, 0.000000, 0.000000}
\pgfsetstrokecolor{dialinecolor}
\draw (8.898172\du,-191.625037\du)--(14.630489\du,-186.063507\du);
}
\pgfsetlinewidth{0.080000\du}
\pgfsetdash{}{0pt}
\pgfsetdash{}{0pt}
\pgfsetbuttcap
{
\definecolor{dialinecolor}{rgb}{0.000000, 0.000000, 0.000000}
\pgfsetfillcolor{dialinecolor}
% was here!!!
\pgfsetarrowsend{stealth}
\definecolor{dialinecolor}{rgb}{0.000000, 0.000000, 0.000000}
\pgfsetstrokecolor{dialinecolor}
\draw (9.037564\du,-189.300172\du)--(14.500823\du,-185.895216\du);
}
\pgfsetlinewidth{0.080000\du}
\pgfsetdash{}{0pt}
\pgfsetdash{}{0pt}
\pgfsetbuttcap
{
\definecolor{dialinecolor}{rgb}{0.000000, 0.000000, 0.000000}
\pgfsetfillcolor{dialinecolor}
% was here!!!
\pgfsetarrowsend{stealth}
\definecolor{dialinecolor}{rgb}{0.000000, 0.000000, 0.000000}
\pgfsetstrokecolor{dialinecolor}
\draw (9.121500\du,-186.394393\du)--(14.361771\du,-185.529740\du);
}
\definecolor{dialinecolor}{rgb}{1.000000, 1.000000, 1.000000}
\pgfsetfillcolor{dialinecolor}
\fill (-1.394217\du,-197.574197\du)--(-1.394217\du,-194.784908\du)--(1.393283\du,-194.784908\du)--(1.393283\du,-197.574197\du)--cycle;
\pgfsetlinewidth{0.000000\du}
\pgfsetdash{}{0pt}
\pgfsetdash{}{0pt}
\pgfsetmiterjoin
\definecolor{dialinecolor}{rgb}{1.000000, 1.000000, 1.000000}
\pgfsetstrokecolor{dialinecolor}
\draw (-1.394217\du,-197.574197\du)--(-1.394217\du,-194.784908\du)--(1.393283\du,-194.784908\du)--(1.393283\du,-197.574197\du)--cycle;
% setfont left to latex
\definecolor{dialinecolor}{rgb}{0.000000, 0.000000, 0.000000}
\pgfsetstrokecolor{dialinecolor}
\node at (-0.000467\du,-196.741697\du){Input};
% setfont left to latex
\definecolor{dialinecolor}{rgb}{0.000000, 0.000000, 0.000000}
\pgfsetstrokecolor{dialinecolor}
\node at (-0.000467\du,-196.294374\du){};
% setfont left to latex
\definecolor{dialinecolor}{rgb}{0.000000, 0.000000, 0.000000}
\pgfsetstrokecolor{dialinecolor}
\node at (-0.000467\du,-195.847052\du){};
% setfont left to latex
\definecolor{dialinecolor}{rgb}{0.000000, 0.000000, 0.000000}
\pgfsetstrokecolor{dialinecolor}
\node at (-0.000467\du,-195.399730\du){Layer};
\definecolor{dialinecolor}{rgb}{1.000000, 1.000000, 1.000000}
\pgfsetfillcolor{dialinecolor}
\fill (6.658195\du,-197.587703\du)--(6.658195\du,-194.798414\du)--(9.408882\du,-194.798414\du)--(9.408882\du,-197.587703\du)--cycle;
\pgfsetlinewidth{0.000000\du}
\pgfsetdash{}{0pt}
\pgfsetdash{}{0pt}
\pgfsetmiterjoin
\definecolor{dialinecolor}{rgb}{1.000000, 1.000000, 1.000000}
\pgfsetstrokecolor{dialinecolor}
\draw (6.658195\du,-197.587703\du)--(6.658195\du,-194.798414\du)--(9.408882\du,-194.798414\du)--(9.408882\du,-197.587703\du)--cycle;
% setfont left to latex
\definecolor{dialinecolor}{rgb}{0.000000, 0.000000, 0.000000}
\pgfsetstrokecolor{dialinecolor}
\node at (8.033539\du,-196.755203\du){Latent};
% setfont left to latex
\definecolor{dialinecolor}{rgb}{0.000000, 0.000000, 0.000000}
\pgfsetstrokecolor{dialinecolor}
\node at (8.033539\du,-196.307881\du){};
% setfont left to latex
\definecolor{dialinecolor}{rgb}{0.000000, 0.000000, 0.000000}
\pgfsetstrokecolor{dialinecolor}
\node at (8.033539\du,-195.860559\du){};
% setfont left to latex
\definecolor{dialinecolor}{rgb}{0.000000, 0.000000, 0.000000}
\pgfsetstrokecolor{dialinecolor}
\node at (8.033539\du,-195.413237\du){Space};
\definecolor{dialinecolor}{rgb}{1.000000, 1.000000, 1.000000}
\pgfsetfillcolor{dialinecolor}
\fill (13.668071\du,-197.541559\du)--(13.668071\du,-194.752271\du)--(16.455571\du,-194.752271\du)--(16.455571\du,-197.541559\du)--cycle;
\pgfsetlinewidth{0.000000\du}
\pgfsetdash{}{0pt}
\pgfsetdash{}{0pt}
\pgfsetmiterjoin
\definecolor{dialinecolor}{rgb}{1.000000, 1.000000, 1.000000}
\pgfsetstrokecolor{dialinecolor}
\draw (13.668071\du,-197.541559\du)--(13.668071\du,-194.752271\du)--(16.455571\du,-194.752271\du)--(16.455571\du,-197.541559\du)--cycle;
% setfont left to latex
\definecolor{dialinecolor}{rgb}{0.000000, 0.000000, 0.000000}
\pgfsetstrokecolor{dialinecolor}
\node at (15.061821\du,-196.709059\du){Output};
% setfont left to latex
\definecolor{dialinecolor}{rgb}{0.000000, 0.000000, 0.000000}
\pgfsetstrokecolor{dialinecolor}
\node at (15.061821\du,-196.261737\du){ };
% setfont left to latex
\definecolor{dialinecolor}{rgb}{0.000000, 0.000000, 0.000000}
\pgfsetstrokecolor{dialinecolor}
\node at (15.061821\du,-195.814415\du){};
% setfont left to latex
\definecolor{dialinecolor}{rgb}{0.000000, 0.000000, 0.000000}
\pgfsetstrokecolor{dialinecolor}
\node at (15.061821\du,-195.367093\du){Layer};
\pgfsetlinewidth{0.080000\du}
\pgfsetdash{}{0pt}
\pgfsetdash{}{0pt}
\pgfsetbuttcap
{
\definecolor{dialinecolor}{rgb}{0.000000, 0.000000, 0.000000}
\pgfsetfillcolor{dialinecolor}
% was here!!!
\pgfsetarrowsend{stealth}
\definecolor{dialinecolor}{rgb}{0.000000, 0.000000, 0.000000}
\pgfsetstrokecolor{dialinecolor}
\draw (16.336433\du,-193.498046\du)--(18.509790\du,-193.444114\du);
}
\pgfsetlinewidth{0.080000\du}
\pgfsetdash{}{0pt}
\pgfsetdash{}{0pt}
\pgfsetbuttcap
{
\definecolor{dialinecolor}{rgb}{0.000000, 0.000000, 0.000000}
\pgfsetfillcolor{dialinecolor}
% was here!!!
\pgfsetarrowsend{stealth}
\definecolor{dialinecolor}{rgb}{0.000000, 0.000000, 0.000000}
\pgfsetstrokecolor{dialinecolor}
\draw (16.401506\du,-190.691097\du)--(18.532845\du,-190.713328\du);
}
\pgfsetlinewidth{0.080000\du}
\pgfsetdash{}{0pt}
\pgfsetdash{}{0pt}
\pgfsetbuttcap
{
\definecolor{dialinecolor}{rgb}{0.000000, 0.000000, 0.000000}
\pgfsetfillcolor{dialinecolor}
% was here!!!
\pgfsetarrowsend{stealth}
\definecolor{dialinecolor}{rgb}{0.000000, 0.000000, 0.000000}
\pgfsetstrokecolor{dialinecolor}
\draw (16.346412\du,-185.367421\du)--(18.585028\du,-185.368508\du);
}
\pgfsetlinewidth{0.100000\du}
\pgfsetdash{}{0pt}
\pgfsetdash{}{0pt}
\pgfsetbuttcap
\pgfsetmiterjoin
\pgfsetlinewidth{0.100000\du}
\pgfsetbuttcap
\pgfsetmiterjoin
\pgfsetdash{}{0pt}
\definecolor{dialinecolor}{rgb}{0.000000, 0.000000, 0.000000}
\pgfsetfillcolor{dialinecolor}
\pgfpathellipse{\pgfpoint{8.059748\du}{-188.488165\du}}{\pgfpoint{0.054256\du}{0\du}}{\pgfpoint{0\du}{0.054256\du}}
\pgfusepath{fill}
\definecolor{dialinecolor}{rgb}{0.000000, 0.000000, 0.000000}
\pgfsetstrokecolor{dialinecolor}
\pgfpathellipse{\pgfpoint{8.059748\du}{-188.488165\du}}{\pgfpoint{0.054256\du}{0\du}}{\pgfpoint{0\du}{0.054256\du}}
\pgfusepath{stroke}
\pgfsetbuttcap
\pgfsetmiterjoin
\pgfsetdash{}{0pt}
\definecolor{dialinecolor}{rgb}{0.000000, 0.000000, 0.000000}
\pgfsetstrokecolor{dialinecolor}
\pgfpathellipse{\pgfpoint{8.059748\du}{-188.488165\du}}{\pgfpoint{0.054256\du}{0\du}}{\pgfpoint{0\du}{0.054256\du}}
\pgfusepath{stroke}
\pgfsetlinewidth{0.100000\du}
\pgfsetdash{}{0pt}
\pgfsetdash{}{0pt}
\pgfsetbuttcap
\pgfsetmiterjoin
\pgfsetlinewidth{0.100000\du}
\pgfsetbuttcap
\pgfsetmiterjoin
\pgfsetdash{}{0pt}
\definecolor{dialinecolor}{rgb}{0.000000, 0.000000, 0.000000}
\pgfsetfillcolor{dialinecolor}
\pgfpathellipse{\pgfpoint{8.067707\du}{-187.809990\du}}{\pgfpoint{0.054256\du}{0\du}}{\pgfpoint{0\du}{0.054256\du}}
\pgfusepath{fill}
\definecolor{dialinecolor}{rgb}{0.000000, 0.000000, 0.000000}
\pgfsetstrokecolor{dialinecolor}
\pgfpathellipse{\pgfpoint{8.067707\du}{-187.809990\du}}{\pgfpoint{0.054256\du}{0\du}}{\pgfpoint{0\du}{0.054256\du}}
\pgfusepath{stroke}
\pgfsetbuttcap
\pgfsetmiterjoin
\pgfsetdash{}{0pt}
\definecolor{dialinecolor}{rgb}{0.000000, 0.000000, 0.000000}
\pgfsetstrokecolor{dialinecolor}
\pgfpathellipse{\pgfpoint{8.067707\du}{-187.809990\du}}{\pgfpoint{0.054256\du}{0\du}}{\pgfpoint{0\du}{0.054256\du}}
\pgfusepath{stroke}
\pgfsetlinewidth{0.100000\du}
\pgfsetdash{}{0pt}
\pgfsetdash{}{0pt}
\pgfsetbuttcap
\pgfsetmiterjoin
\pgfsetlinewidth{0.100000\du}
\pgfsetbuttcap
\pgfsetmiterjoin
\pgfsetdash{}{0pt}
\definecolor{dialinecolor}{rgb}{0.000000, 0.000000, 0.000000}
\pgfsetfillcolor{dialinecolor}
\pgfpathellipse{\pgfpoint{15.354433\du}{-189.239925\du}}{\pgfpoint{0.054256\du}{0\du}}{\pgfpoint{0\du}{0.054256\du}}
\pgfusepath{fill}
\definecolor{dialinecolor}{rgb}{0.000000, 0.000000, 0.000000}
\pgfsetstrokecolor{dialinecolor}
\pgfpathellipse{\pgfpoint{15.354433\du}{-189.239925\du}}{\pgfpoint{0.054256\du}{0\du}}{\pgfpoint{0\du}{0.054256\du}}
\pgfusepath{stroke}
\pgfsetbuttcap
\pgfsetmiterjoin
\pgfsetdash{}{0pt}
\definecolor{dialinecolor}{rgb}{0.000000, 0.000000, 0.000000}
\pgfsetstrokecolor{dialinecolor}
\pgfpathellipse{\pgfpoint{15.354433\du}{-189.239925\du}}{\pgfpoint{0.054256\du}{0\du}}{\pgfpoint{0\du}{0.054256\du}}
\pgfusepath{stroke}
\pgfsetlinewidth{0.100000\du}
\pgfsetdash{}{0pt}
\pgfsetdash{}{0pt}
\pgfsetbuttcap
\pgfsetmiterjoin
\pgfsetlinewidth{0.100000\du}
\pgfsetbuttcap
\pgfsetmiterjoin
\pgfsetdash{}{0pt}
\definecolor{dialinecolor}{rgb}{0.000000, 0.000000, 0.000000}
\pgfsetfillcolor{dialinecolor}
\pgfpathellipse{\pgfpoint{15.356153\du}{-188.529835\du}}{\pgfpoint{0.054256\du}{0\du}}{\pgfpoint{0\du}{0.054256\du}}
\pgfusepath{fill}
\definecolor{dialinecolor}{rgb}{0.000000, 0.000000, 0.000000}
\pgfsetstrokecolor{dialinecolor}
\pgfpathellipse{\pgfpoint{15.356153\du}{-188.529835\du}}{\pgfpoint{0.054256\du}{0\du}}{\pgfpoint{0\du}{0.054256\du}}
\pgfusepath{stroke}
\pgfsetbuttcap
\pgfsetmiterjoin
\pgfsetdash{}{0pt}
\definecolor{dialinecolor}{rgb}{0.000000, 0.000000, 0.000000}
\pgfsetstrokecolor{dialinecolor}
\pgfpathellipse{\pgfpoint{15.356153\du}{-188.529835\du}}{\pgfpoint{0.054256\du}{0\du}}{\pgfpoint{0\du}{0.054256\du}}
\pgfusepath{stroke}
\pgfsetlinewidth{0.100000\du}
\pgfsetdash{}{0pt}
\pgfsetdash{}{0pt}
\pgfsetbuttcap
\pgfsetmiterjoin
\pgfsetlinewidth{0.100000\du}
\pgfsetbuttcap
\pgfsetmiterjoin
\pgfsetdash{}{0pt}
\definecolor{dialinecolor}{rgb}{0.000000, 0.000000, 0.000000}
\pgfsetfillcolor{dialinecolor}
\pgfpathellipse{\pgfpoint{15.357872\du}{-187.848401\du}}{\pgfpoint{0.054256\du}{0\du}}{\pgfpoint{0\du}{0.054256\du}}
\pgfusepath{fill}
\definecolor{dialinecolor}{rgb}{0.000000, 0.000000, 0.000000}
\pgfsetstrokecolor{dialinecolor}
\pgfpathellipse{\pgfpoint{15.357872\du}{-187.848401\du}}{\pgfpoint{0.054256\du}{0\du}}{\pgfpoint{0\du}{0.054256\du}}
\pgfusepath{stroke}
\pgfsetbuttcap
\pgfsetmiterjoin
\pgfsetdash{}{0pt}
\definecolor{dialinecolor}{rgb}{0.000000, 0.000000, 0.000000}
\pgfsetstrokecolor{dialinecolor}
\pgfpathellipse{\pgfpoint{15.357872\du}{-187.848401\du}}{\pgfpoint{0.054256\du}{0\du}}{\pgfpoint{0\du}{0.054256\du}}
\pgfusepath{stroke}
\pgfsetlinewidth{0.100000\du}
\pgfsetdash{}{0pt}
\pgfsetdash{}{0pt}
\pgfsetbuttcap
\pgfsetmiterjoin
\pgfsetlinewidth{0.100000\du}
\pgfsetbuttcap
\pgfsetmiterjoin
\pgfsetdash{}{0pt}
\definecolor{dialinecolor}{rgb}{0.000000, 0.000000, 0.000000}
\pgfsetfillcolor{dialinecolor}
\pgfpathellipse{\pgfpoint{15.353860\du}{-187.109655\du}}{\pgfpoint{0.054256\du}{0\du}}{\pgfpoint{0\du}{0.054256\du}}
\pgfusepath{fill}
\definecolor{dialinecolor}{rgb}{0.000000, 0.000000, 0.000000}
\pgfsetstrokecolor{dialinecolor}
\pgfpathellipse{\pgfpoint{15.353860\du}{-187.109655\du}}{\pgfpoint{0.054256\du}{0\du}}{\pgfpoint{0\du}{0.054256\du}}
\pgfusepath{stroke}
\pgfsetbuttcap
\pgfsetmiterjoin
\pgfsetdash{}{0pt}
\definecolor{dialinecolor}{rgb}{0.000000, 0.000000, 0.000000}
\pgfsetstrokecolor{dialinecolor}
\pgfpathellipse{\pgfpoint{15.353860\du}{-187.109655\du}}{\pgfpoint{0.054256\du}{0\du}}{\pgfpoint{0\du}{0.054256\du}}
\pgfusepath{stroke}
\pgfsetlinewidth{0.100000\du}
\pgfsetdash{}{0pt}
\pgfsetdash{}{0pt}
\pgfsetbuttcap
\pgfsetmiterjoin
\pgfsetlinewidth{0.100000\du}
\pgfsetbuttcap
\pgfsetmiterjoin
\pgfsetdash{}{0pt}
\definecolor{dialinecolor}{rgb}{0.000000, 0.000000, 0.000000}
\pgfsetfillcolor{dialinecolor}
\pgfpathellipse{\pgfpoint{-0.003305\du}{-187.033939\du}}{\pgfpoint{0.054256\du}{0\du}}{\pgfpoint{0\du}{0.054256\du}}
\pgfusepath{fill}
\definecolor{dialinecolor}{rgb}{0.000000, 0.000000, 0.000000}
\pgfsetstrokecolor{dialinecolor}
\pgfpathellipse{\pgfpoint{-0.003305\du}{-187.033939\du}}{\pgfpoint{0.054256\du}{0\du}}{\pgfpoint{0\du}{0.054256\du}}
\pgfusepath{stroke}
\pgfsetbuttcap
\pgfsetmiterjoin
\pgfsetdash{}{0pt}
\definecolor{dialinecolor}{rgb}{0.000000, 0.000000, 0.000000}
\pgfsetstrokecolor{dialinecolor}
\pgfpathellipse{\pgfpoint{-0.003305\du}{-187.033939\du}}{\pgfpoint{0.054256\du}{0\du}}{\pgfpoint{0\du}{0.054256\du}}
\pgfusepath{stroke}
\pgfsetlinewidth{0.100000\du}
\pgfsetdash{}{0pt}
\pgfsetdash{}{0pt}
\pgfsetbuttcap
\pgfsetmiterjoin
\pgfsetlinewidth{0.100000\du}
\pgfsetbuttcap
\pgfsetmiterjoin
\pgfsetdash{}{0pt}
\definecolor{dialinecolor}{rgb}{0.000000, 0.000000, 0.000000}
\pgfsetfillcolor{dialinecolor}
\pgfpathellipse{\pgfpoint{0.003557\du}{-187.765386\du}}{\pgfpoint{0.054256\du}{0\du}}{\pgfpoint{0\du}{0.054256\du}}
\pgfusepath{fill}
\definecolor{dialinecolor}{rgb}{0.000000, 0.000000, 0.000000}
\pgfsetstrokecolor{dialinecolor}
\pgfpathellipse{\pgfpoint{0.003557\du}{-187.765386\du}}{\pgfpoint{0.054256\du}{0\du}}{\pgfpoint{0\du}{0.054256\du}}
\pgfusepath{stroke}
\pgfsetbuttcap
\pgfsetmiterjoin
\pgfsetdash{}{0pt}
\definecolor{dialinecolor}{rgb}{0.000000, 0.000000, 0.000000}
\pgfsetstrokecolor{dialinecolor}
\pgfpathellipse{\pgfpoint{0.003557\du}{-187.765386\du}}{\pgfpoint{0.054256\du}{0\du}}{\pgfpoint{0\du}{0.054256\du}}
\pgfusepath{stroke}
\pgfsetlinewidth{0.100000\du}
\pgfsetdash{}{0pt}
\pgfsetdash{}{0pt}
\pgfsetbuttcap
\pgfsetmiterjoin
\pgfsetlinewidth{0.100000\du}
\pgfsetbuttcap
\pgfsetmiterjoin
\pgfsetdash{}{0pt}
\definecolor{dialinecolor}{rgb}{0.000000, 0.000000, 0.000000}
\pgfsetfillcolor{dialinecolor}
\pgfpathellipse{\pgfpoint{0.005602\du}{-189.260357\du}}{\pgfpoint{0.054256\du}{0\du}}{\pgfpoint{0\du}{0.054256\du}}
\pgfusepath{fill}
\definecolor{dialinecolor}{rgb}{0.000000, 0.000000, 0.000000}
\pgfsetstrokecolor{dialinecolor}
\pgfpathellipse{\pgfpoint{0.005602\du}{-189.260357\du}}{\pgfpoint{0.054256\du}{0\du}}{\pgfpoint{0\du}{0.054256\du}}
\pgfusepath{stroke}
\pgfsetbuttcap
\pgfsetmiterjoin
\pgfsetdash{}{0pt}
\definecolor{dialinecolor}{rgb}{0.000000, 0.000000, 0.000000}
\pgfsetstrokecolor{dialinecolor}
\pgfpathellipse{\pgfpoint{0.005602\du}{-189.260357\du}}{\pgfpoint{0.054256\du}{0\du}}{\pgfpoint{0\du}{0.054256\du}}
\pgfusepath{stroke}
\pgfsetlinewidth{0.100000\du}
\pgfsetdash{}{0pt}
\pgfsetdash{}{0pt}
\pgfsetbuttcap
\pgfsetmiterjoin
\pgfsetlinewidth{0.100000\du}
\pgfsetbuttcap
\pgfsetmiterjoin
\pgfsetdash{}{0pt}
\definecolor{dialinecolor}{rgb}{0.000000, 0.000000, 0.000000}
\pgfsetfillcolor{dialinecolor}
\pgfpathellipse{\pgfpoint{0.000831\du}{-188.593123\du}}{\pgfpoint{0.054256\du}{0\du}}{\pgfpoint{0\du}{0.054256\du}}
\pgfusepath{fill}
\definecolor{dialinecolor}{rgb}{0.000000, 0.000000, 0.000000}
\pgfsetstrokecolor{dialinecolor}
\pgfpathellipse{\pgfpoint{0.000831\du}{-188.593123\du}}{\pgfpoint{0.054256\du}{0\du}}{\pgfpoint{0\du}{0.054256\du}}
\pgfusepath{stroke}
\pgfsetbuttcap
\pgfsetmiterjoin
\pgfsetdash{}{0pt}
\definecolor{dialinecolor}{rgb}{0.000000, 0.000000, 0.000000}
\pgfsetstrokecolor{dialinecolor}
\pgfpathellipse{\pgfpoint{0.000831\du}{-188.593123\du}}{\pgfpoint{0.054256\du}{0\du}}{\pgfpoint{0\du}{0.054256\du}}
\pgfusepath{stroke}
\definecolor{dialinecolor}{rgb}{1.000000, 1.000000, 1.000000}
\pgfsetfillcolor{dialinecolor}
\fill (3.958726\du,-184.167113\du)--(3.958726\du,-182.267113\du)--(11.988493\du,-182.267113\du)--(11.988493\du,-184.167113\du)--cycle;
\pgfsetlinewidth{0.050000\du}
\pgfsetdash{}{0pt}
\pgfsetdash{}{0pt}
\pgfsetmiterjoin
\definecolor{dialinecolor}{rgb}{0.000000, 0.000000, 0.000000}
\pgfsetstrokecolor{dialinecolor}
\draw (3.958726\du,-184.167113\du)--(3.958726\du,-182.267113\du)--(11.988493\du,-182.267113\du)--(11.988493\du,-184.167113\du)--cycle;
% setfont left to latex
\definecolor{dialinecolor}{rgb}{0.000000, 0.000000, 0.000000}
\pgfsetstrokecolor{dialinecolor}
\node at (7.973609\du,-183.022113\du){Softmax};
\definecolor{dialinecolor}{rgb}{1.000000, 1.000000, 1.000000}
\pgfsetfillcolor{dialinecolor}
\fill (3.996255\du,-180.794707\du)--(3.996255\du,-178.894707\du)--(12.026022\du,-178.894707\du)--(12.026022\du,-180.794707\du)--cycle;
\pgfsetlinewidth{0.050000\du}
\pgfsetdash{}{0pt}
\pgfsetdash{}{0pt}
\pgfsetmiterjoin
\definecolor{dialinecolor}{rgb}{0.000000, 0.000000, 0.000000}
\pgfsetstrokecolor{dialinecolor}
\draw (3.996255\du,-180.794707\du)--(3.996255\du,-178.894707\du)--(12.026022\du,-178.894707\du)--(12.026022\du,-180.794707\du)--cycle;
% setfont left to latex
\definecolor{dialinecolor}{rgb}{0.000000, 0.000000, 0.000000}
\pgfsetstrokecolor{dialinecolor}
\node at (8.011138\du,-179.649707\du){KL Divergence};
\pgfsetlinewidth{0.050000\du}
\pgfsetdash{}{0pt}
\pgfsetdash{}{0pt}
\pgfsetbuttcap
\pgfsetmiterjoin
\pgfsetlinewidth{0.050000\du}
\pgfsetbuttcap
\pgfsetmiterjoin
\pgfsetdash{}{0pt}
\definecolor{dialinecolor}{rgb}{1.000000, 1.000000, 1.000000}
\pgfsetfillcolor{dialinecolor}
\fill (7.876151\du,-185.442041\du)--(7.876151\du,-184.867076\du)--(7.588669\du,-184.867076\du)--(8.163634\du,-184.292111\du)--(8.738599\du,-184.867076\du)--(8.451116\du,-184.867076\du)--(8.451116\du,-185.442041\du)--cycle;
\definecolor{dialinecolor}{rgb}{0.000000, 0.000000, 0.000000}
\pgfsetstrokecolor{dialinecolor}
\draw (7.876151\du,-185.442041\du)--(7.876151\du,-184.867076\du)--(7.588669\du,-184.867076\du)--(8.163634\du,-184.292111\du)--(8.738599\du,-184.867076\du)--(8.451116\du,-184.867076\du)--(8.451116\du,-185.442041\du)--cycle;
\pgfsetbuttcap
\pgfsetmiterjoin
\pgfsetdash{}{0pt}
\definecolor{dialinecolor}{rgb}{0.000000, 0.000000, 0.000000}
\pgfsetstrokecolor{dialinecolor}
\draw (7.876151\du,-185.442041\du)--(7.876151\du,-184.867076\du)--(7.588669\du,-184.867076\du)--(8.163634\du,-184.292111\du)--(8.738599\du,-184.867076\du)--(8.451116\du,-184.867076\du)--(8.451116\du,-185.442041\du)--cycle;
\pgfsetlinewidth{0.050000\du}
\pgfsetdash{}{0pt}
\pgfsetdash{}{0pt}
\pgfsetbuttcap
\pgfsetmiterjoin
\pgfsetlinewidth{0.050000\du}
\pgfsetbuttcap
\pgfsetmiterjoin
\pgfsetdash{}{0pt}
\definecolor{dialinecolor}{rgb}{1.000000, 1.000000, 1.000000}
\pgfsetfillcolor{dialinecolor}
\fill (7.805193\du,-182.104665\du)--(7.805193\du,-181.529700\du)--(7.517710\du,-181.529700\du)--(8.092675\du,-180.954735\du)--(8.667640\du,-181.529700\du)--(8.380158\du,-181.529700\du)--(8.380158\du,-182.104665\du)--cycle;
\definecolor{dialinecolor}{rgb}{0.000000, 0.000000, 0.000000}
\pgfsetstrokecolor{dialinecolor}
\draw (7.805193\du,-182.104665\du)--(7.805193\du,-181.529700\du)--(7.517710\du,-181.529700\du)--(8.092675\du,-180.954735\du)--(8.667640\du,-181.529700\du)--(8.380158\du,-181.529700\du)--(8.380158\du,-182.104665\du)--cycle;
\pgfsetbuttcap
\pgfsetmiterjoin
\pgfsetdash{}{0pt}
\definecolor{dialinecolor}{rgb}{0.000000, 0.000000, 0.000000}
\pgfsetstrokecolor{dialinecolor}
\draw (7.805193\du,-182.104665\du)--(7.805193\du,-181.529700\du)--(7.517710\du,-181.529700\du)--(8.092675\du,-180.954735\du)--(8.667640\du,-181.529700\du)--(8.380158\du,-181.529700\du)--(8.380158\du,-182.104665\du)--cycle;
\pgfsetlinewidth{0.100000\du}
\pgfsetdash{}{0pt}
\pgfsetdash{}{0pt}
\pgfsetmiterjoin
\definecolor{dialinecolor}{rgb}{0.000000, 0.000000, 0.000000}
\pgfsetstrokecolor{dialinecolor}
\draw (-4.106773\du,-198.006085\du)--(-4.106773\du,-178.384799\du)--(19.141478\du,-178.384799\du)--(19.141478\du,-198.006085\du)--cycle;
\end{tikzpicture}

  \caption{Auto-Encoder}
  \label{fig:archi}
\end{figure}
\section{Produce constraints}

We consider that two documents are in the same cluster only if the
importance of key words are similare, and two documents are not in the
same cluster only if the importance of each key words are dissimilar.
\\
We can use the method proposed by Yen-Chang Hsu and Zsolt Kira
\cite{2015arXiv151106321H}. We apply the softmax function to the
latent space. The outputs of the whole sofmax layer could be viewed as
the distribution of the importance of each key words. We can use the
Kullback-Leibler (KL) divergence to ealuate the similarity between
distributions.
\\
To produce constraints we use a multi layer perceptron taking two
documents $x_p'$ and $x_q'$ for input and the functions $I_s$ and
$I_{ds}$ for output.
\\
We can see the architecture in figure~\ref{fig:final_archi}.
\begin{equation}
  x_p' = g(x_p)
\end{equation}
\begin{equation}
  x_q' = g(x_q)
\end{equation}

\begin{equation}\label{eq:Is}
I_s(x_p', x_q'; \theta_2) = \left\{
    \begin{array}{ll}
        1~if~x_p'~and~x_q'~are~similare \\
        0~otherwise
    \end{array}
\right.
\end{equation}
\begin{equation}\label{eq:Ids}
I_{ds}(x_p', x_q'; \theta_2) = \left\{
    \begin{array}{ll}
        1~if~x_p'~and~x_q'~are~disimilare \\
        0~otherwise
    \end{array}
    \right.
\end{equation}

\begin{equation}\label{eq:KL}
KL(x_p' || x_q') = \sum_{i=1}^k {x_p'}_i . log(\frac{{x_p'}_i}{{x_q'}_i}) 
\end{equation}

We denote $L_{KL}$ the loss function for this network :
\begin{equation}\label{eq:LossKL}
L_{KL}(x_p', x_q') = L(x_p' || x_q') + L(x_q' || x_p') 
\end{equation}
\begin{equation}\label{eq:KLpq}
  L(x_p' || x_q') = I_s(x_p', x_q'; \theta_2) . KL(x_p' || x_q') +
  I_{ds}(x_p', x_q''; \theta_2) . min(0, margin - KL(x_p' || x_q'))
\end{equation}
\\ \\
Finally the loss function is :

\begin{equation}\label{eq:loss_FINALE}
  L = L_{SAE}(X_P, X_q) + L_{KL}(X_P, X_q)
\end{equation}

\begin{figure}[!t]
  \centering
  \tikzset{every picture/.style={scale=1.5}}
  % Graphic for TeX using PGF
% Title: /home/maxence/Documents/stage/article/v0/ressources/dia_file/net_prod.dia
% Creator: Dia v0.97.3
% CreationDate: Sat Apr  7 01:12:42 2018
% For: maxence
% \usepackage{tikz}
% The following commands are not supported in PSTricks at present
% We define them conditionally, so when they are implemented,
% this pgf file will use them.
\ifx\du\undefined
  \newlength{\du}
\fi
\setlength{\du}{15\unitlength}
\begin{tikzpicture}
\pgftransformxscale{1.000000}
\pgftransformyscale{-1.000000}
\definecolor{dialinecolor}{rgb}{0.000000, 0.000000, 0.000000}
\pgfsetstrokecolor{dialinecolor}
\definecolor{dialinecolor}{rgb}{1.000000, 1.000000, 1.000000}
\pgfsetfillcolor{dialinecolor}
\pgfsetlinewidth{0.000000\du}
\pgfsetdash{}{0pt}
\pgfsetdash{}{0pt}
\pgfsetmiterjoin
\definecolor{dialinecolor}{rgb}{1.000000, 1.000000, 1.000000}
\pgfsetstrokecolor{dialinecolor}
\draw (17.904685\du,9.320014\du)--(17.904685\du,10.661503\du)--(19.479685\du,10.661503\du)--(19.479685\du,9.320014\du)--cycle;
% setfont left to latex
\definecolor{dialinecolor}{rgb}{0.000000, 0.000000, 0.000000}
\pgfsetstrokecolor{dialinecolor}
\node at (18.692185\du,10.072514\du){Xp};
\pgfsetlinewidth{0.050000\du}
\pgfsetdash{}{0pt}
\pgfsetdash{}{0pt}
\pgfsetbuttcap
\pgfsetmiterjoin
\pgfsetlinewidth{0.050000\du}
\pgfsetbuttcap
\pgfsetmiterjoin
\pgfsetdash{}{0pt}
\definecolor{dialinecolor}{rgb}{1.000000, 1.000000, 1.000000}
\pgfsetfillcolor{dialinecolor}
\pgfpathellipse{\pgfpoint{20.270582\du}{9.239892\du}}{\pgfpoint{0.284372\du}{0\du}}{\pgfpoint{0\du}{0.284372\du}}
\pgfusepath{fill}
\definecolor{dialinecolor}{rgb}{0.000000, 0.000000, 0.000000}
\pgfsetstrokecolor{dialinecolor}
\pgfpathellipse{\pgfpoint{20.270582\du}{9.239892\du}}{\pgfpoint{0.284372\du}{0\du}}{\pgfpoint{0\du}{0.284372\du}}
\pgfusepath{stroke}
\pgfsetbuttcap
\pgfsetmiterjoin
\pgfsetdash{}{0pt}
\definecolor{dialinecolor}{rgb}{0.000000, 0.000000, 0.000000}
\pgfsetstrokecolor{dialinecolor}
\pgfpathellipse{\pgfpoint{20.270582\du}{9.239892\du}}{\pgfpoint{0.284372\du}{0\du}}{\pgfpoint{0\du}{0.284372\du}}
\pgfusepath{stroke}
\pgfsetlinewidth{0.000000\du}
\pgfsetdash{}{0pt}
\pgfsetdash{}{0pt}
\pgfsetmiterjoin
\definecolor{dialinecolor}{rgb}{1.000000, 1.000000, 1.000000}
\pgfsetstrokecolor{dialinecolor}
\draw (17.915936\du,12.008508\du)--(17.915936\du,13.349997\du)--(19.490936\du,13.349997\du)--(19.490936\du,12.008508\du)--cycle;
% setfont left to latex
\definecolor{dialinecolor}{rgb}{0.000000, 0.000000, 0.000000}
\pgfsetstrokecolor{dialinecolor}
\node at (18.703436\du,12.761008\du){Xq};
\pgfsetlinewidth{0.000000\du}
\pgfsetdash{}{0pt}
\pgfsetdash{}{0pt}
\pgfsetmiterjoin
\definecolor{dialinecolor}{rgb}{1.000000, 1.000000, 1.000000}
\pgfsetstrokecolor{dialinecolor}
\draw (26.661550\du,9.970407\du)--(26.661550\du,11.311896\du)--(28.236550\du,11.311896\du)--(28.236550\du,9.970407\du)--cycle;
% setfont left to latex
\definecolor{dialinecolor}{rgb}{0.000000, 0.000000, 0.000000}
\pgfsetstrokecolor{dialinecolor}
\node at (27.449050\du,10.722907\du){Is};
\pgfsetlinewidth{0.000000\du}
\pgfsetdash{}{0pt}
\pgfsetdash{}{0pt}
\pgfsetmiterjoin
\definecolor{dialinecolor}{rgb}{1.000000, 1.000000, 1.000000}
\pgfsetstrokecolor{dialinecolor}
\draw (26.782824\du,11.708711\du)--(26.782824\du,13.050200\du)--(28.357824\du,13.050200\du)--(28.357824\du,11.708711\du)--cycle;
% setfont left to latex
\definecolor{dialinecolor}{rgb}{0.000000, 0.000000, 0.000000}
\pgfsetstrokecolor{dialinecolor}
\node at (27.570324\du,12.461211\du){Ids};
\pgfsetlinewidth{0.050000\du}
\pgfsetdash{}{0pt}
\pgfsetdash{}{0pt}
\pgfsetbuttcap
\pgfsetmiterjoin
\pgfsetlinewidth{0.050000\du}
\pgfsetbuttcap
\pgfsetmiterjoin
\pgfsetdash{}{0pt}
\definecolor{dialinecolor}{rgb}{1.000000, 1.000000, 1.000000}
\pgfsetfillcolor{dialinecolor}
\pgfpathellipse{\pgfpoint{20.275294\du}{10.976219\du}}{\pgfpoint{0.284372\du}{0\du}}{\pgfpoint{0\du}{0.284372\du}}
\pgfusepath{fill}
\definecolor{dialinecolor}{rgb}{0.000000, 0.000000, 0.000000}
\pgfsetstrokecolor{dialinecolor}
\pgfpathellipse{\pgfpoint{20.275294\du}{10.976219\du}}{\pgfpoint{0.284372\du}{0\du}}{\pgfpoint{0\du}{0.284372\du}}
\pgfusepath{stroke}
\pgfsetbuttcap
\pgfsetmiterjoin
\pgfsetdash{}{0pt}
\definecolor{dialinecolor}{rgb}{0.000000, 0.000000, 0.000000}
\pgfsetstrokecolor{dialinecolor}
\pgfpathellipse{\pgfpoint{20.275294\du}{10.976219\du}}{\pgfpoint{0.284372\du}{0\du}}{\pgfpoint{0\du}{0.284372\du}}
\pgfusepath{stroke}
\pgfsetlinewidth{0.050000\du}
\pgfsetdash{}{0pt}
\pgfsetdash{}{0pt}
\pgfsetbuttcap
\pgfsetmiterjoin
\pgfsetlinewidth{0.050000\du}
\pgfsetbuttcap
\pgfsetmiterjoin
\pgfsetdash{}{0pt}
\definecolor{dialinecolor}{rgb}{1.000000, 1.000000, 1.000000}
\pgfsetfillcolor{dialinecolor}
\pgfpathellipse{\pgfpoint{20.282439\du}{11.987809\du}}{\pgfpoint{0.284372\du}{0\du}}{\pgfpoint{0\du}{0.284372\du}}
\pgfusepath{fill}
\definecolor{dialinecolor}{rgb}{0.000000, 0.000000, 0.000000}
\pgfsetstrokecolor{dialinecolor}
\pgfpathellipse{\pgfpoint{20.282439\du}{11.987809\du}}{\pgfpoint{0.284372\du}{0\du}}{\pgfpoint{0\du}{0.284372\du}}
\pgfusepath{stroke}
\pgfsetbuttcap
\pgfsetmiterjoin
\pgfsetdash{}{0pt}
\definecolor{dialinecolor}{rgb}{0.000000, 0.000000, 0.000000}
\pgfsetstrokecolor{dialinecolor}
\pgfpathellipse{\pgfpoint{20.282439\du}{11.987809\du}}{\pgfpoint{0.284372\du}{0\du}}{\pgfpoint{0\du}{0.284372\du}}
\pgfusepath{stroke}
\pgfsetlinewidth{0.050000\du}
\pgfsetdash{}{0pt}
\pgfsetdash{}{0pt}
\pgfsetbuttcap
\pgfsetmiterjoin
\pgfsetlinewidth{0.050000\du}
\pgfsetbuttcap
\pgfsetmiterjoin
\pgfsetdash{}{0pt}
\definecolor{dialinecolor}{rgb}{1.000000, 1.000000, 1.000000}
\pgfsetfillcolor{dialinecolor}
\pgfpathellipse{\pgfpoint{20.324939\du}{13.611117\du}}{\pgfpoint{0.284372\du}{0\du}}{\pgfpoint{0\du}{0.284372\du}}
\pgfusepath{fill}
\definecolor{dialinecolor}{rgb}{0.000000, 0.000000, 0.000000}
\pgfsetstrokecolor{dialinecolor}
\pgfpathellipse{\pgfpoint{20.324939\du}{13.611117\du}}{\pgfpoint{0.284372\du}{0\du}}{\pgfpoint{0\du}{0.284372\du}}
\pgfusepath{stroke}
\pgfsetbuttcap
\pgfsetmiterjoin
\pgfsetdash{}{0pt}
\definecolor{dialinecolor}{rgb}{0.000000, 0.000000, 0.000000}
\pgfsetstrokecolor{dialinecolor}
\pgfpathellipse{\pgfpoint{20.324939\du}{13.611117\du}}{\pgfpoint{0.284372\du}{0\du}}{\pgfpoint{0\du}{0.284372\du}}
\pgfusepath{stroke}
\pgfsetlinewidth{0.050000\du}
\pgfsetdash{}{0pt}
\pgfsetdash{}{0pt}
\pgfsetbuttcap
\pgfsetmiterjoin
\pgfsetlinewidth{0.050000\du}
\pgfsetbuttcap
\pgfsetmiterjoin
\pgfsetdash{}{0pt}
\definecolor{dialinecolor}{rgb}{1.000000, 1.000000, 1.000000}
\pgfsetfillcolor{dialinecolor}
\pgfpathellipse{\pgfpoint{24.946524\du}{10.006140\du}}{\pgfpoint{0.284372\du}{0\du}}{\pgfpoint{0\du}{0.284372\du}}
\pgfusepath{fill}
\definecolor{dialinecolor}{rgb}{0.000000, 0.000000, 0.000000}
\pgfsetstrokecolor{dialinecolor}
\pgfpathellipse{\pgfpoint{24.946524\du}{10.006140\du}}{\pgfpoint{0.284372\du}{0\du}}{\pgfpoint{0\du}{0.284372\du}}
\pgfusepath{stroke}
\pgfsetbuttcap
\pgfsetmiterjoin
\pgfsetdash{}{0pt}
\definecolor{dialinecolor}{rgb}{0.000000, 0.000000, 0.000000}
\pgfsetstrokecolor{dialinecolor}
\pgfpathellipse{\pgfpoint{24.946524\du}{10.006140\du}}{\pgfpoint{0.284372\du}{0\du}}{\pgfpoint{0\du}{0.284372\du}}
\pgfusepath{stroke}
\pgfsetlinewidth{0.050000\du}
\pgfsetdash{}{0pt}
\pgfsetdash{}{0pt}
\pgfsetbuttcap
\pgfsetmiterjoin
\pgfsetlinewidth{0.050000\du}
\pgfsetbuttcap
\pgfsetmiterjoin
\pgfsetdash{}{0pt}
\definecolor{dialinecolor}{rgb}{1.000000, 1.000000, 1.000000}
\pgfsetfillcolor{dialinecolor}
\pgfpathellipse{\pgfpoint{25.007411\du}{13.366454\du}}{\pgfpoint{0.284372\du}{0\du}}{\pgfpoint{0\du}{0.284372\du}}
\pgfusepath{fill}
\definecolor{dialinecolor}{rgb}{0.000000, 0.000000, 0.000000}
\pgfsetstrokecolor{dialinecolor}
\pgfpathellipse{\pgfpoint{25.007411\du}{13.366454\du}}{\pgfpoint{0.284372\du}{0\du}}{\pgfpoint{0\du}{0.284372\du}}
\pgfusepath{stroke}
\pgfsetbuttcap
\pgfsetmiterjoin
\pgfsetdash{}{0pt}
\definecolor{dialinecolor}{rgb}{0.000000, 0.000000, 0.000000}
\pgfsetstrokecolor{dialinecolor}
\pgfpathellipse{\pgfpoint{25.007411\du}{13.366454\du}}{\pgfpoint{0.284372\du}{0\du}}{\pgfpoint{0\du}{0.284372\du}}
\pgfusepath{stroke}
\pgfsetlinewidth{0.050000\du}
\pgfsetdash{}{0pt}
\pgfsetdash{}{0pt}
\pgfsetbuttcap
\pgfsetmiterjoin
\pgfsetlinewidth{0.050000\du}
\pgfsetbuttcap
\pgfsetmiterjoin
\pgfsetdash{}{0pt}
\definecolor{dialinecolor}{rgb}{1.000000, 1.000000, 1.000000}
\pgfsetfillcolor{dialinecolor}
\pgfpathellipse{\pgfpoint{26.319991\du}{10.703337\du}}{\pgfpoint{0.284372\du}{0\du}}{\pgfpoint{0\du}{0.284372\du}}
\pgfusepath{fill}
\definecolor{dialinecolor}{rgb}{0.000000, 0.000000, 0.000000}
\pgfsetstrokecolor{dialinecolor}
\pgfpathellipse{\pgfpoint{26.319991\du}{10.703337\du}}{\pgfpoint{0.284372\du}{0\du}}{\pgfpoint{0\du}{0.284372\du}}
\pgfusepath{stroke}
\pgfsetbuttcap
\pgfsetmiterjoin
\pgfsetdash{}{0pt}
\definecolor{dialinecolor}{rgb}{0.000000, 0.000000, 0.000000}
\pgfsetstrokecolor{dialinecolor}
\pgfpathellipse{\pgfpoint{26.319991\du}{10.703337\du}}{\pgfpoint{0.284372\du}{0\du}}{\pgfpoint{0\du}{0.284372\du}}
\pgfusepath{stroke}
\pgfsetlinewidth{0.050000\du}
\pgfsetdash{}{0pt}
\pgfsetdash{}{0pt}
\pgfsetbuttcap
\pgfsetmiterjoin
\pgfsetlinewidth{0.050000\du}
\pgfsetbuttcap
\pgfsetmiterjoin
\pgfsetdash{}{0pt}
\definecolor{dialinecolor}{rgb}{1.000000, 1.000000, 1.000000}
\pgfsetfillcolor{dialinecolor}
\pgfpathellipse{\pgfpoint{26.347219\du}{12.332866\du}}{\pgfpoint{0.284372\du}{0\du}}{\pgfpoint{0\du}{0.284372\du}}
\pgfusepath{fill}
\definecolor{dialinecolor}{rgb}{0.000000, 0.000000, 0.000000}
\pgfsetstrokecolor{dialinecolor}
\pgfpathellipse{\pgfpoint{26.347219\du}{12.332866\du}}{\pgfpoint{0.284372\du}{0\du}}{\pgfpoint{0\du}{0.284372\du}}
\pgfusepath{stroke}
\pgfsetbuttcap
\pgfsetmiterjoin
\pgfsetdash{}{0pt}
\definecolor{dialinecolor}{rgb}{0.000000, 0.000000, 0.000000}
\pgfsetstrokecolor{dialinecolor}
\pgfpathellipse{\pgfpoint{26.347219\du}{12.332866\du}}{\pgfpoint{0.284372\du}{0\du}}{\pgfpoint{0\du}{0.284372\du}}
\pgfusepath{stroke}
\pgfsetlinewidth{0.001000\du}
\pgfsetdash{}{0pt}
\pgfsetdash{}{0pt}
\pgfsetbuttcap
{
\definecolor{dialinecolor}{rgb}{0.000000, 0.000000, 0.000000}
\pgfsetfillcolor{dialinecolor}
% was here!!!
\pgfsetarrowsend{stealth}
\definecolor{dialinecolor}{rgb}{0.000000, 0.000000, 0.000000}
\pgfsetstrokecolor{dialinecolor}
\draw (20.577510\du,10.913458\du)--(24.644308\du,10.068902\du);
}
\pgfsetlinewidth{0.001000\du}
\pgfsetdash{}{0pt}
\pgfsetdash{}{0pt}
\pgfsetbuttcap
{
\definecolor{dialinecolor}{rgb}{0.000000, 0.000000, 0.000000}
\pgfsetfillcolor{dialinecolor}
% was here!!!
\pgfsetarrowsend{stealth}
\definecolor{dialinecolor}{rgb}{0.000000, 0.000000, 0.000000}
\pgfsetstrokecolor{dialinecolor}
\draw (20.574244\du,9.289654\du)--(24.642862\du,9.956379\du);
}
\pgfsetlinewidth{0.001000\du}
\pgfsetdash{}{0pt}
\pgfsetdash{}{0pt}
\pgfsetbuttcap
{
\definecolor{dialinecolor}{rgb}{0.000000, 0.000000, 0.000000}
\pgfsetfillcolor{dialinecolor}
% was here!!!
\pgfsetarrowsend{stealth}
\definecolor{dialinecolor}{rgb}{0.000000, 0.000000, 0.000000}
\pgfsetstrokecolor{dialinecolor}
\draw (20.567397\du,11.866737\du)--(24.661566\du,10.127213\du);
}
\pgfsetlinewidth{0.001000\du}
\pgfsetdash{}{0pt}
\pgfsetdash{}{0pt}
\pgfsetbuttcap
{
\definecolor{dialinecolor}{rgb}{0.000000, 0.000000, 0.000000}
\pgfsetfillcolor{dialinecolor}
% was here!!!
\pgfsetarrowsend{stealth}
\definecolor{dialinecolor}{rgb}{0.000000, 0.000000, 0.000000}
\pgfsetstrokecolor{dialinecolor}
\draw (20.562660\du,13.413530\du)--(24.662152\du,10.006140\du);
}
\pgfsetlinewidth{0.001000\du}
\pgfsetdash{}{0pt}
\pgfsetdash{}{0pt}
\pgfsetbuttcap
{
\definecolor{dialinecolor}{rgb}{0.000000, 0.000000, 0.000000}
\pgfsetfillcolor{dialinecolor}
% was here!!!
\pgfsetarrowsend{stealth}
\definecolor{dialinecolor}{rgb}{0.000000, 0.000000, 0.000000}
\pgfsetstrokecolor{dialinecolor}
\draw (20.634170\du,13.594960\du)--(24.698180\du,13.382612\du);
}
\pgfsetlinewidth{0.001000\du}
\pgfsetdash{}{0pt}
\pgfsetdash{}{0pt}
\pgfsetbuttcap
{
\definecolor{dialinecolor}{rgb}{0.000000, 0.000000, 0.000000}
\pgfsetfillcolor{dialinecolor}
% was here!!!
\pgfsetarrowsend{stealth}
\definecolor{dialinecolor}{rgb}{0.000000, 0.000000, 0.000000}
\pgfsetstrokecolor{dialinecolor}
\draw (20.578903\du,12.074311\du)--(24.710947\du,13.279952\du);
}
\pgfsetlinewidth{0.001000\du}
\pgfsetdash{}{0pt}
\pgfsetdash{}{0pt}
\pgfsetbuttcap
{
\definecolor{dialinecolor}{rgb}{0.000000, 0.000000, 0.000000}
\pgfsetfillcolor{dialinecolor}
% was here!!!
\pgfsetarrowsend{stealth}
\definecolor{dialinecolor}{rgb}{0.000000, 0.000000, 0.000000}
\pgfsetstrokecolor{dialinecolor}
\draw (20.551700\du,11.115835\du)--(24.731005\du,13.226839\du);
}
\pgfsetlinewidth{0.001000\du}
\pgfsetdash{}{0pt}
\pgfsetdash{}{0pt}
\pgfsetbuttcap
{
\definecolor{dialinecolor}{rgb}{0.000000, 0.000000, 0.000000}
\pgfsetfillcolor{dialinecolor}
% was here!!!
\pgfsetarrowsend{stealth}
\definecolor{dialinecolor}{rgb}{0.000000, 0.000000, 0.000000}
\pgfsetstrokecolor{dialinecolor}
\draw (20.503029\du,9.442392\du)--(24.774964\du,13.163955\du);
}
\pgfsetlinewidth{0.001000\du}
\pgfsetdash{}{0pt}
\pgfsetdash{}{0pt}
\pgfsetbuttcap
{
\definecolor{dialinecolor}{rgb}{0.000000, 0.000000, 0.000000}
\pgfsetfillcolor{dialinecolor}
% was here!!!
\pgfsetarrowsend{stealth}
\definecolor{dialinecolor}{rgb}{0.000000, 0.000000, 0.000000}
\pgfsetstrokecolor{dialinecolor}
\draw (25.252410\du,13.177451\du)--(26.102220\du,12.521869\du);
}
\pgfsetlinewidth{0.001000\du}
\pgfsetdash{}{0pt}
\pgfsetdash{}{0pt}
\pgfsetbuttcap
{
\definecolor{dialinecolor}{rgb}{0.000000, 0.000000, 0.000000}
\pgfsetfillcolor{dialinecolor}
% was here!!!
\pgfsetarrowsend{stealth}
\definecolor{dialinecolor}{rgb}{0.000000, 0.000000, 0.000000}
\pgfsetstrokecolor{dialinecolor}
\draw (25.080069\du,10.284484\du)--(26.062847\du,12.332866\du);
}
\pgfsetlinewidth{0.001000\du}
\pgfsetdash{}{0pt}
\pgfsetdash{}{0pt}
\pgfsetbuttcap
{
\definecolor{dialinecolor}{rgb}{0.000000, 0.000000, 0.000000}
\pgfsetfillcolor{dialinecolor}
% was here!!!
\pgfsetarrowsend{stealth}
\definecolor{dialinecolor}{rgb}{0.000000, 0.000000, 0.000000}
\pgfsetstrokecolor{dialinecolor}
\draw (25.143924\du,13.089480\du)--(26.183477\du,10.980312\du);
}
\pgfsetlinewidth{0.001000\du}
\pgfsetdash{}{0pt}
\pgfsetdash{}{0pt}
\pgfsetbuttcap
{
\definecolor{dialinecolor}{rgb}{0.000000, 0.000000, 0.000000}
\pgfsetfillcolor{dialinecolor}
% was here!!!
\pgfsetarrowsend{stealth}
\definecolor{dialinecolor}{rgb}{0.000000, 0.000000, 0.000000}
\pgfsetstrokecolor{dialinecolor}
\draw (25.221486\du,10.145716\du)--(26.045029\du,10.563762\du);
}
\pgfsetlinewidth{0.100000\du}
\pgfsetdash{}{0pt}
\pgfsetdash{}{0pt}
\pgfsetbuttcap
\pgfsetmiterjoin
\pgfsetlinewidth{0.100000\du}
\pgfsetbuttcap
\pgfsetmiterjoin
\pgfsetdash{}{0pt}
\definecolor{dialinecolor}{rgb}{0.000000, 0.000000, 0.000000}
\pgfsetfillcolor{dialinecolor}
\fill (19.465209\du,9.153654\du)--(19.683958\du,9.153654\du)--(19.683958\du,9.084904\du)--(19.902706\du,9.222403\du)--(19.683958\du,9.359902\du)--(19.683958\du,9.291153\du)--(19.465209\du,9.291153\du)--cycle;
\definecolor{dialinecolor}{rgb}{0.000000, 0.000000, 0.000000}
\pgfsetstrokecolor{dialinecolor}
\draw (19.465209\du,9.153654\du)--(19.683958\du,9.153654\du)--(19.683958\du,9.084904\du)--(19.902706\du,9.222403\du)--(19.683958\du,9.359902\du)--(19.683958\du,9.291153\du)--(19.465209\du,9.291153\du)--cycle;
\pgfsetbuttcap
\pgfsetmiterjoin
\pgfsetdash{}{0pt}
\definecolor{dialinecolor}{rgb}{0.000000, 0.000000, 0.000000}
\pgfsetstrokecolor{dialinecolor}
\draw (19.465209\du,9.153654\du)--(19.683958\du,9.153654\du)--(19.683958\du,9.084904\du)--(19.902706\du,9.222403\du)--(19.683958\du,9.359902\du)--(19.683958\du,9.291153\du)--(19.465209\du,9.291153\du)--cycle;
\pgfsetlinewidth{0.100000\du}
\pgfsetdash{}{0pt}
\pgfsetdash{}{0pt}
\pgfsetbuttcap
\pgfsetmiterjoin
\pgfsetlinewidth{0.100000\du}
\pgfsetbuttcap
\pgfsetmiterjoin
\pgfsetdash{}{0pt}
\definecolor{dialinecolor}{rgb}{0.000000, 0.000000, 0.000000}
\pgfsetfillcolor{dialinecolor}
\fill (19.512977\du,10.913055\du)--(19.731726\du,10.913055\du)--(19.731726\du,10.844306\du)--(19.950474\du,10.981805\du)--(19.731726\du,11.119304\du)--(19.731726\du,11.050554\du)--(19.512977\du,11.050554\du)--cycle;
\definecolor{dialinecolor}{rgb}{0.000000, 0.000000, 0.000000}
\pgfsetstrokecolor{dialinecolor}
\draw (19.512977\du,10.913055\du)--(19.731726\du,10.913055\du)--(19.731726\du,10.844306\du)--(19.950474\du,10.981805\du)--(19.731726\du,11.119304\du)--(19.731726\du,11.050554\du)--(19.512977\du,11.050554\du)--cycle;
\pgfsetbuttcap
\pgfsetmiterjoin
\pgfsetdash{}{0pt}
\definecolor{dialinecolor}{rgb}{0.000000, 0.000000, 0.000000}
\pgfsetstrokecolor{dialinecolor}
\draw (19.512977\du,10.913055\du)--(19.731726\du,10.913055\du)--(19.731726\du,10.844306\du)--(19.950474\du,10.981805\du)--(19.731726\du,11.119304\du)--(19.731726\du,11.050554\du)--(19.512977\du,11.050554\du)--cycle;
\pgfsetlinewidth{0.100000\du}
\pgfsetdash{}{0pt}
\pgfsetdash{}{0pt}
\pgfsetbuttcap
\pgfsetmiterjoin
\pgfsetlinewidth{0.100000\du}
\pgfsetbuttcap
\pgfsetmiterjoin
\pgfsetdash{}{0pt}
\definecolor{dialinecolor}{rgb}{0.000000, 0.000000, 0.000000}
\pgfsetfillcolor{dialinecolor}
\fill (19.518473\du,11.949947\du)--(19.737221\du,11.949947\du)--(19.737221\du,11.881197\du)--(19.955970\du,12.018697\du)--(19.737221\du,12.156196\du)--(19.737221\du,12.087446\du)--(19.518473\du,12.087446\du)--cycle;
\definecolor{dialinecolor}{rgb}{0.000000, 0.000000, 0.000000}
\pgfsetstrokecolor{dialinecolor}
\draw (19.518473\du,11.949947\du)--(19.737221\du,11.949947\du)--(19.737221\du,11.881197\du)--(19.955970\du,12.018697\du)--(19.737221\du,12.156196\du)--(19.737221\du,12.087446\du)--(19.518473\du,12.087446\du)--cycle;
\pgfsetbuttcap
\pgfsetmiterjoin
\pgfsetdash{}{0pt}
\definecolor{dialinecolor}{rgb}{0.000000, 0.000000, 0.000000}
\pgfsetstrokecolor{dialinecolor}
\draw (19.518473\du,11.949947\du)--(19.737221\du,11.949947\du)--(19.737221\du,11.881197\du)--(19.955970\du,12.018697\du)--(19.737221\du,12.156196\du)--(19.737221\du,12.087446\du)--(19.518473\du,12.087446\du)--cycle;
\pgfsetlinewidth{0.100000\du}
\pgfsetdash{}{0pt}
\pgfsetdash{}{0pt}
\pgfsetbuttcap
\pgfsetmiterjoin
\pgfsetlinewidth{0.100000\du}
\pgfsetbuttcap
\pgfsetmiterjoin
\pgfsetdash{}{0pt}
\definecolor{dialinecolor}{rgb}{0.000000, 0.000000, 0.000000}
\pgfsetfillcolor{dialinecolor}
\fill (19.515876\du,13.557551\du)--(19.734624\du,13.557551\du)--(19.734624\du,13.488801\du)--(19.953373\du,13.626300\du)--(19.734624\du,13.763799\du)--(19.734624\du,13.695050\du)--(19.515876\du,13.695050\du)--cycle;
\definecolor{dialinecolor}{rgb}{0.000000, 0.000000, 0.000000}
\pgfsetstrokecolor{dialinecolor}
\draw (19.515876\du,13.557551\du)--(19.734624\du,13.557551\du)--(19.734624\du,13.488801\du)--(19.953373\du,13.626300\du)--(19.734624\du,13.763799\du)--(19.734624\du,13.695050\du)--(19.515876\du,13.695050\du)--cycle;
\pgfsetbuttcap
\pgfsetmiterjoin
\pgfsetdash{}{0pt}
\definecolor{dialinecolor}{rgb}{0.000000, 0.000000, 0.000000}
\pgfsetstrokecolor{dialinecolor}
\draw (19.515876\du,13.557551\du)--(19.734624\du,13.557551\du)--(19.734624\du,13.488801\du)--(19.953373\du,13.626300\du)--(19.734624\du,13.763799\du)--(19.734624\du,13.695050\du)--(19.515876\du,13.695050\du)--cycle;
\pgfsetlinewidth{0.100000\du}
\pgfsetdash{}{0pt}
\pgfsetdash{}{0pt}
\pgfsetbuttcap
\pgfsetmiterjoin
\pgfsetlinewidth{0.100000\du}
\pgfsetbuttcap
\pgfsetmiterjoin
\pgfsetdash{}{0pt}
\definecolor{dialinecolor}{rgb}{0.000000, 0.000000, 0.000000}
\pgfsetfillcolor{dialinecolor}
\fill (26.715722\du,12.277408\du)--(26.934471\du,12.277408\du)--(26.934471\du,12.208659\du)--(27.153219\du,12.346158\du)--(26.934471\du,12.483657\du)--(26.934471\du,12.414907\du)--(26.715722\du,12.414907\du)--cycle;
\definecolor{dialinecolor}{rgb}{0.000000, 0.000000, 0.000000}
\pgfsetstrokecolor{dialinecolor}
\draw (26.715722\du,12.277408\du)--(26.934471\du,12.277408\du)--(26.934471\du,12.208659\du)--(27.153219\du,12.346158\du)--(26.934471\du,12.483657\du)--(26.934471\du,12.414907\du)--(26.715722\du,12.414907\du)--cycle;
\pgfsetbuttcap
\pgfsetmiterjoin
\pgfsetdash{}{0pt}
\definecolor{dialinecolor}{rgb}{0.000000, 0.000000, 0.000000}
\pgfsetstrokecolor{dialinecolor}
\draw (26.715722\du,12.277408\du)--(26.934471\du,12.277408\du)--(26.934471\du,12.208659\du)--(27.153219\du,12.346158\du)--(26.934471\du,12.483657\du)--(26.934471\du,12.414907\du)--(26.715722\du,12.414907\du)--cycle;
\pgfsetlinewidth{0.100000\du}
\pgfsetdash{}{0pt}
\pgfsetdash{}{0pt}
\pgfsetbuttcap
\pgfsetmiterjoin
\pgfsetlinewidth{0.100000\du}
\pgfsetbuttcap
\pgfsetmiterjoin
\pgfsetdash{}{0pt}
\definecolor{dialinecolor}{rgb}{0.000000, 0.000000, 0.000000}
\pgfsetfillcolor{dialinecolor}
\fill (26.683866\du,10.551395\du)--(26.902615\du,10.551395\du)--(26.902615\du,10.482645\du)--(27.121363\du,10.620144\du)--(26.902615\du,10.757643\du)--(26.902615\du,10.688894\du)--(26.683866\du,10.688894\du)--cycle;
\definecolor{dialinecolor}{rgb}{0.000000, 0.000000, 0.000000}
\pgfsetstrokecolor{dialinecolor}
\draw (26.683866\du,10.551395\du)--(26.902615\du,10.551395\du)--(26.902615\du,10.482645\du)--(27.121363\du,10.620144\du)--(26.902615\du,10.757643\du)--(26.902615\du,10.688894\du)--(26.683866\du,10.688894\du)--cycle;
\pgfsetbuttcap
\pgfsetmiterjoin
\pgfsetdash{}{0pt}
\definecolor{dialinecolor}{rgb}{0.000000, 0.000000, 0.000000}
\pgfsetstrokecolor{dialinecolor}
\draw (26.683866\du,10.551395\du)--(26.902615\du,10.551395\du)--(26.902615\du,10.482645\du)--(27.121363\du,10.620144\du)--(26.902615\du,10.757643\du)--(26.902615\du,10.688894\du)--(26.683866\du,10.688894\du)--cycle;
\pgfsetlinewidth{0.000000\du}
\pgfsetdash{}{0pt}
\pgfsetdash{}{0pt}
\pgfsetmiterjoin
\pgfsetbuttcap
{
\definecolor{dialinecolor}{rgb}{0.000000, 0.000000, 0.000000}
\pgfsetfillcolor{dialinecolor}
% was here!!!
\definecolor{dialinecolor}{rgb}{0.000000, 0.000000, 0.000000}
\pgfsetstrokecolor{dialinecolor}
\pgfpathmoveto{\pgfpoint{19.412105\du}{9.071014\du}}
\pgfpathcurveto{\pgfpoint{19.102034\du}{9.082838\du}}{\pgfpoint{19.333274\du}{9.848820\du}}{\pgfpoint{19.165099\du}{9.959185\du}}
\pgfpathcurveto{\pgfpoint{18.996925\du}{10.069549\du}}{\pgfpoint{19.059990\du}{9.938163\du}}{\pgfpoint{19.154588\du}{10.043272\du}}
\pgfpathcurveto{\pgfpoint{19.249187\du}{10.148381\du}}{\pgfpoint{19.182944\du}{11.173198\du}}{\pgfpoint{19.481744\du}{11.173198\du}}
\pgfusepath{stroke}
}
\pgfsetlinewidth{0.000000\du}
\pgfsetdash{}{0pt}
\pgfsetdash{}{0pt}
\pgfsetmiterjoin
\pgfsetbuttcap
{
\definecolor{dialinecolor}{rgb}{0.000000, 0.000000, 0.000000}
\pgfsetfillcolor{dialinecolor}
% was here!!!
\definecolor{dialinecolor}{rgb}{0.000000, 0.000000, 0.000000}
\pgfsetstrokecolor{dialinecolor}
\pgfpathmoveto{\pgfpoint{19.471858\du}{11.801868\du}}
\pgfpathcurveto{\pgfpoint{19.161786\du}{11.813692\du}}{\pgfpoint{19.393026\du}{12.579674\du}}{\pgfpoint{19.224851\du}{12.690039\du}}
\pgfpathcurveto{\pgfpoint{19.056677\du}{12.800403\du}}{\pgfpoint{19.119743\du}{12.669017\du}}{\pgfpoint{19.214341\du}{12.774126\du}}
\pgfpathcurveto{\pgfpoint{19.308939\du}{12.879235\du}}{\pgfpoint{19.242697\du}{13.904052\du}}{\pgfpoint{19.541497\du}{13.904052\du}}
\pgfusepath{stroke}
}
\pgfsetlinewidth{0.020000\du}
\pgfsetdash{}{0pt}
\pgfsetdash{}{0pt}
\pgfsetmiterjoin
\definecolor{dialinecolor}{rgb}{0.000000, 0.000000, 0.000000}
\pgfsetstrokecolor{dialinecolor}
\draw (17.835710\du,8.735258\du)--(17.835710\du,13.983485\du)--(28.441211\du,13.983485\du)--(28.441211\du,8.735258\du)--cycle;
% setfont left to latex
\definecolor{dialinecolor}{rgb}{0.000000, 0.000000, 0.000000}
\pgfsetstrokecolor{dialinecolor}
\node at (23.138461\du,11.554371\du){};
\end{tikzpicture}

  \caption{Produce Architecture}
  \label{fig:final_archi}
\end{figure}
\section{Integrate ML \& CL to K-Means}

We can use the COPKmeans algorithme proposed by Kiri Wagstaff, Claire
Cardie, Seth Rogers and Stephan Schr\"odl
\cite{Wagstaff:2001:CKC:645530.655669} 

\begin{algorithm}[H]
  \KwData{Data Set D, must-link consytraints $Con_=$, cannot-
    link constraints $Con_\neq$}
  Let $C_1$ ... $C_k$ be the initial clusters centers;\\
  \Repeat{convergence}{
    \ForEach{$d_i \in D$}{
      
    }
    \ForEach{$C_i$}{
      update($C_i$)
    }
  }
  return \{$C_1$...$C_k$\}
 \caption{COP-Kmeans Algorithm}
\end{algorithm} 
\nocite{*}
\printbibliography[title=References]
\end{document}
