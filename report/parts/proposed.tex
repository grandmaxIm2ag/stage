\section{Proposed Method}
The idea is to learn a latent space taking into account lexical constraints and
background knowledge.
\\We denote  X a document of size d,
C the corpus, N the size of C, and K the number of cluster.
\subsection{Pairwise Constraints}
We introduce background knowledge, so we need
a representation taking into account these constraints.\\
For must-link constraints we want close representations for each pair $X_i$, $X_j$
in the set ML. We introduce a penalty $\omega_{ML}$ for must-link
constraints and use squared loss :
\begin{equation}\label{eq:omegaML}
  \omega_{ML} = \sum_{\forall{(X_i,X_j)\in ML}} || h_\theta(X_i) - h_\theta(X_j) ||_2^2
\end{equation}
For cannot-link constraints we want distant representations for each pair $X_i$,
$X_j$ in the set CL.
We can use the hinge loss.
We introduce a penalty $\omega_{CL}$ for cannot-link constraints.
\begin{equation}\label{eq:omegaCL}
  \omega_{CL} = \sum\limits_{\forall{(X_i,X_j)\in CL}} max(0,
  \eta - ||h_\theta(X_i) - h_\theta(X_j) ||_2^2)
\end{equation}
where $\eta \ge 0$ is a hyperparameter.\\
\subsection{Lexical Constraints}
There are two ways to see this problem, the first is that the user gives a set 
of keywords and it's up to our algorithm to find the classes that best fit our 
set of keywords. A second way of seeing, is that the user gives keywords with 
classes, in other word he gives for each class a set of keywords.
\subsubsection{First Method}
We begin with the method where the user gives the keywords without information
about classes. In this method we want a latent space where the representation
of the document X is biasing by keywords.\\
We denote $KW = \begin{pmatrix} kw_1 & kw_2 & ... & kw_{N_{kw}-1} & kw_{N_{kw}}
\end {pmatrix}$
the set of index of keywords, X' the vector showing only information about 
keywords such that :
\begin{equation}
\forall_{i=1,2,..,d}X_i' = \left\{
\begin{array}{ll}
  X_i & \mbox{if } i \in KW \\
  0 & \mbox{Otherwise.}
\end{array}
\right.
\end{equation}
To learn the latent space we add a penalty such that the representation
of the document X is close of the representation of X'. In this way, in the
latent space, the representation of document will be biased by keywords.
We introduce a penalty $\omega_{KW}$ for lexical constraints and
use squared loss : 
\begin{equation}\label{eq:omega1}
  \omega_{KW} = \sum\limits_{X \in C} || h_\theta(X) - h_\theta(X')||_2^2
\end{equation}
The plus of this method is that we have as many constraints as document. 
However, it is possible, in practice, to have a large number of null vector X', 
in other word a large number of  document X without any keywords.
It might be interesting to capture these documents in a new class having
for representative the embedding for null document $\vv{0}$. We denote    
$R=(r_0, r_1, ...., r_K)$ our K+1 representatives, we denote 
$r_0$ the representative of the adding class as follows : 
\begin{equation}
r_0 = h_\theta(\vv{0})
\end{equation}
\subsubsection{Second Method}
It is possible that the user will know more than
information as simple keywords. Indeed, the user can know the 
classes of keywords.
We denote 
\begin{equation}
KW = \begin{pmatrix}KW_1  \\ ... \\ KW_k \\ ...\\ ... \\ KW_{K}\end{pmatrix}
\end{equation}
where $KW_k$ the set of keyword for the $k^{th}$ class. Also, we denote X' such that :
\begin{equation}
X' = \begin{pmatrix}mask_1(X)  \\ ... \\ mask_k(X) \\ ...\\ ... \\ mask_K(X)\end{pmatrix}
\end{equation}
where $mask_k(X)$ is the document X  masked by keywords from the $k^{th}$ class. 
For this method we want representation where document X is close of the closest
$X'_k$ : 
\begin{equation}
|| h_\theta(X) - \argmin\limits_{X'_k \in X'}(|| h_\theta(X) - 
h_\theta(X'_k)||_2^2)  ||_2^2
\end{equation}
However, the argmin function is not differentiable, and we cant use SGD algorithm
to learn this function. To approximate the $\argmin$ function we can use the 
parameterized softmax function \cite{doi:10.1117/1.2819119}. The softmax 
function can be used as a differentiable substitute to $\argmin$. 
\\The penalty takes the form : 
\begin{equation}\label{eq:omega_kw_soft}
\begin{array}{l}
  \omega_{KW} = \\ \sum\limits_{X \in C} \sum\limits_{k = 1}^K \frac{e^{-\beta|| h_\theta(X) - 
h_\theta(X'_k)||_2^2}}{\sum\limits_{k' = 1}^K e^{-\beta|| h_\theta(X) - 
h_\theta(X'_{k'})||_2^2}}|| h_\theta(X) - h_\theta(X'_{k})||_2^2
\end{array}
\end{equation}
With this method, we can have the same problem as in the previous version.
We can add $X'_0$ such $X'_0 = \vv{0}$, the penalty takes the form :
\begin{equation}\label{eq:k_add}
\begin{array}{l}
  \omega_{KW} = \\ \sum\limits_{X \in C} \sum\limits_{k = 0}^K \frac{e^{|| h_\theta(X) - 
h_\theta(X'_k)||_2^2}}{\sum\limits_{k' = 0}^K e^{|| h_\theta(X) - 
h_\theta(X'_{k'})||_2^2}}|| h_\theta(X) - h_\theta(X'_{k})||_2^2
\end{array}
\end{equation} 
%%%%%%%%%%%%%%%%%%%%%%
%%%%%%%%%%%%%%%%%%%%%%
%%%%%%%%%%%%%%%%%%%%%%
\subsection{Non Clustering Loss}
The penalty is :
\begin{equation}\label{eq:Sparse}
  \Omega(C,KW;\theta) = \alpha_0\omega_{KW} + \alpha_1\omega_{CL} + \alpha_2\omega_{ML}  
\end{equation}
With hyperparameters $\alpha_0\geq 0$ ,$\alpha_1\geq 0$, $\alpha_2\geq 0$.
\\Then the reconstruct loss function is :
\begin{equation}\label{eq:AEDK}
  L_{rec}(C, \theta) = \sum\limits_{X \in C}(||X - f(h_\theta(X))||_2^2
\end{equation}
Finally the Non clustering loss function is :
\begin{equation}\label{eq:AE}
  L_{NonClust}(C,KW; \theta) = L_{rec}(C, \theta) + \Omega(C)  
\end{equation}

\subsection{Deep $K$-Means}

For the Deep $K$-Means, we can use the approach proposed by Moradi Fard, Thonet and Gaussier 
\cite{Deap-K-Means} see in section 2.3.\\
For the clustering loss we can use squared loss with euclidean distance. We
denote $R = \begin{pmatrix} r_1 & r_2 & ... & r_K\end{pmatrix}$ the vector of
centroids :  
\begin{equation}\label{eq:loss_clust}
  L_{Clust}(C, K;R, \theta) = \sum\limits_{X \in C}\sum\limits_{k=1}^K F(h_X, r_k) G_{k, F}(h_X, \beta; R) 
\end{equation}
Finally the loss function is :
\begin{equation}\label{eq:loss_FINALE}
\begin{array}{l l}
  Min~L(KW, C, K; \theta) = & L_{NonClust}(C, KW; \theta) \\
  & + \lambda.L_{Clust}(C, K;R, \theta)
\end{array}
\end{equation}
with $\lambda \geq 0$.\\
For the learning algorithm we can use the Deep $K$-Means algorithm with 
pretraining~\ref{algo:dkm} see in section \ref{seq:DeepClust}. We use two 
methods for pretraining, a simple pretraining minimizing only reconstruct loss :
$\sum\limits_{X \in X} || X - f(h_\theta(X))||_2^2$
And a lexical pretrain, with the second method we learn a representation 
minimizing $\sum\limits_{X \in X} || X - f(h_\theta(X))||_2^2 + 
\alpha_0 \sum\limits_{X \in C} || h_\theta(X) - h_\theta(X')||_2^2$. 
It allows to initialize centroïds with bias  representation.
