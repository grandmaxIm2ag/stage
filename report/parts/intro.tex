\section{Introduction}\label{sec:intro}

Clustering is one of the most fundamental tasks in data mining and machine
learning. $K$-Means algorithm is a clustering method using centroid models,
it represents each cluster by a single mean vector. $K$-Means clustering sorts
n objects into k clusters in which each observation belongs to
the cluster with the nearest centroid.
Then, $K$-Means is often used in practice and it is easy to interpret. 
\\In real application domains, users may want to introduce constraints to finding 
useful properties for clustering data. Traditional $K$-Means algorithms 
have no way to take advantage of this information.
\\The difficulty with integration of constraints into $K$-Means
algorithm is to find a good representation for data  taking into
account constraints. The Deep Learning and Auto-Encoder can be used to
learn this representation. With Auto-Encoder we have to perform the
$K$-Means in the latent space learned, and this latent space must be
$K$-Means friendly.
\\In this study, we specifically focus on the k-Means algorithm with lexical biases. Lexical biases are represented
by a set of keywords given by the user. We use an Autoencoder to learn a latent space taking
into account biases. The loss of the Autoencoder is divided in two parts, (a) 
the reconstruct loss $L_{rec}$ and (b) different penalties skewing the
representation. Then, the representation must be $K$-Means friendly, 
to do this, we use the Deep $K$-Means model \cite{Deap-K-Means}.  
\\In the next section, we provide some background on the $K$-Means algorithm and
deep learning. In section \ref{seq:proposed}, we proposed a method to introduce constraints to 
the $K$-Means algorithm. And we are experimenting our method in section \ref{seq:exp}.
