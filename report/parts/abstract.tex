\begin{abstract}
  We study in this paper the problem of thematic clustering. 
  Thematic clustering is the set of lexical constraints skewing the text 
  clustering towards different themes. These lexical constraints are represented
  by a set of keywords given by the user. 
  To perform $K$-Means with these constraints, we need a 
  representation where data about documents and constraints are present. For this 
  reason, deep $K$-Means can be used to learn a latent space showing all constraints,
  and perform $K$-Means in the latent space. We propose here an approach to 
  integrate constraints to $K$-Means algorithm by introducing penalties in the
  Non Clustering loss of the Deep $K$-Means model\cite{Deap-K-Means}.
\end{abstract}

\keywords{Clustering, $K$-means, DNN, Autoencoder, Deep Clustering}