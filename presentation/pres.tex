\documentclass{beamer}

\usepackage[utf8]{inputenc}
\usepackage[francais]{babel}
\usepackage[T1]{fontenc}
\usepackage{lmodern}
\usepackage{graphicx}
\usepackage{commath}
\usetheme{Warsaw}

\begin{document}

\title{Ajout de contraintes lexicales et de connaissances \`a priori \`a l'algorithme  des K-Moyennes \`a l'aide de Deep Learning}
\author{Grand Maxence}
\institute{Laboratoire d'Informatique de Grenoble\\\'Equipe AMA\\Supervis\'e par \'Eric Gaussier}

\maketitle

\begin{frame}
  \frametitle{K-Moyenne}

  L'algorithme des K-Moyennes est un algorithme de clustering non supervis\'e.
  \pause
  \\
  Soit D une collection de n documents, K le nombre de clusters souhait\'e, $r_k$
  le rep\'esentant du cluster $C_k$, l 'algorithme des  K-Moyennes maximise la fonction : 
\\
\[
\sum_{k=1}^K ~ \sum_{\forall d_i \in C_k} sim(d_i, r_k)
\]

\end{frame}

\begin{frame}
  \frametitle{Contraintes}
  Les contraintes lexicales sont les vues d\'efinies par un ensemble de mot cl\'es
  donn\'e par l'utilisateur.
  \pause
  \\
  Les connaissances \`a priori sont un ensemble de contraintes "must-link" et "cannot-link".
\end{frame}

\begin{frame}
  \frametitle{Travail Demand\'e}
  \begin{itemize}
  \item \'Etat de l'Art
  \item Rechercher comment repr\'esenter les contraintes
  \item Implantation de l'algorithme
  \item Test de l'algorithme
  \end{itemize}
\end{frame}

\end{document}
