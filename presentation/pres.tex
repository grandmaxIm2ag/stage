\documentclass{beamer}

\usepackage[utf8]{inputenc} 
\usepackage[T1]{fontenc}
\usepackage{lmodern}
\usepackage{graphicx}
\usepackage{commath}
\usetheme{Warsaw}

\begin{document}

\title{Ajout de contraintes lexicales aux k-moyenne}
\author{Grand Maxence}
\maketitle

\begin{frame}
  \frametitle{K-Moyenne}

  L'algorithme K-Moyenne est un algorithme de clustering non supervis\'e.
\\
  Soit C une collection de n documents, k le nombre de clusters souhait\'e, $r_i$
  le rep\'esentant du cluster i, la K-Moyenne cherche \`a minimiser la fonction : 
\\
\[
Loss = \sum_{i=1}^n \norm{C_i - r_i}_2^2
\]
\end{frame}

\begin{frame}
  \frametitle{Les contraintes Lexicales}
  Les contraintes lexicales sont les vues d\'efinits par un ensemble de mot cl\'ees
  donn\'e par l'utilisateur.
\end{frame}

\begin{frame}
  \frametitle{Travail Demand\'e}
  \begin{itemize}
  \item Recherche documentaire
  \item Rechercher comment repr\'esenter les contraintes lexicales
  \item Implantation de l'algotithme
  \item Test de l'algorithme
  \end{itemize}
\end{frame}

\end{document}
